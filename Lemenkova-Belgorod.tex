\documentclass[pdflatex,compress,8pt,
	xcolor={dvipsnames,dvipsnames,svgnames,x11names,table},
	hyperref={colorlinks = true,breaklinks = true, urlcolor = NavyBlue, breaklinks = true}]{beamer}
\usetheme{AnnArbor}

\usepackage{gensymb} % degree symbol
% ----------------------------------------------------------------------------
% *** START BIBLIOGRAPHY <<<
% ----------------------------------------------------------------------------
\usepackage[
	backend=biber, 
%	style=authortitle,
	style = numeric,
%	style=ieee,
%	style=nature,
%	style=science,
%	style=apa,
%	style=mla,
%	style=phys, 
	maxbibnames=99,
%	citestyle=authoryear,
	citestyle=numeric,
	giveninits=true,
	isbn=true,
	url=true,
	natbib=true,
%	sorting=nydt,
%	sorting=ydnt, 
	sorting=ndymdt,
%	sorting=ydnt, % Sort by year (descending), name, title.
% 	sorting=nty (Sort by name, title, year.)
% 	sorting=nyt,
%	sorting=nyvt (Sort by name, year, volume, title).
%	sorting=ydnt, 
	bibencoding=utf8,
	useprefix=false,
	language=auto, 
	autolang=other,
	backref=true,
	backrefstyle=none,
	indexing=cite,
]{biblatex}
\DeclareSortingTemplate{ndymdt}{
  \sort{
    \field{presort}
  }
  \sort[final]{
    \field{sortkey}
  }
  \sort{
    \field{sortname}
    \field{author}
    \field{editor}
    \field{translator}
    \field{sorttitle}
    \field{title}
  }
  \sort[direction=descending]{
    \field{sortyear}
    \field{year}
    \literal{9999}
  }
  \sort[direction=descending]{
    \field[padside=left,padwidth=2,padchar=0]{month}
    \literal{99}
  }
  \sort[direction=descending]{
    \field[padside=left,padwidth=2,padchar=0]{day}
    \literal{99}
  }
  \sort{
    \field{sorttitle}
  }
  \sort[direction=descending]{
    \field[padside=left,padwidth=4,padchar=0]{volume}
    \literal{9999}
  }
}

\addbibresource{Belgorod.bib}
\renewcommand*{\bibfont}{\scriptsize} %\footnotesize

\setbeamertemplate{bibliography item}{\insertbiblabel}

% ----------------------------------------------------------------------------
% *** END BIBLIOGRAPHY <<<
% ----------------------------------------------------------------------------

%%%%%%%%%%%%%%%%%%%%%%%%%%%%%%%%%%%%%%%

\title{Analysis of Landsat NDVI Time Series for Detecting Degradation of Vegetation}
\author{Polina Lemenkova}
         \date{April 6, 2015}

\begin{document}

\begin{frame}
           \titlepage
\end{frame}

\section*{Outline}
        \begin{frame}
           \tableofcontents
         \end{frame}

\section{Introduction}
\subsection{Brief Summary of Research Aims}
       
 \begin{itemize}
        	\item GIS and RS application for environmental studies of Yamal
	\item Calculation of NDVI 
	\item Monitoring vegetation changes in tundra landscapes
	\item Analysis of the vegetation dynamics in the past two decades (1988-2011).
	\item Data: Landsat TM scenes for 1988, 2001 and 2011
	\item Originality: Application of ILWIS
	\item GIS spatial analysis tools and Landsat imagery 
	\item Area: Bovanenkovo region in Yamal Peninsula, Russian Extreme North
\end{itemize}

\begin{frame}\frametitle{Google Earth Image}
\vspace{1em}
\begin{figure}[H]
	\centering
		\includegraphics[width=8cm]{f01.jpg}
	\caption{Study area}\label{fig:f01}
\end{figure}
\end{frame}

\subsection{Study Area}
\begin{frame}\frametitle{Study Area}
\begin{minipage}[0.4\textheight]{\textwidth}
\begin{columns}[T]
\begin{column}{0.5\textwidth}
\vspace{2em}
\begin{figure}[H]
	\centering
		\includegraphics[width=4.0cm]{f02.png}
\end{figure}
Source: B. Forbes
\end{column}
\begin{column}{0.5\textwidth}
\vspace{2em}
\begin{figure}[H]
	\centering
		\includegraphics[width=4.0cm]{f03.png}
\end{figure}
Geographic location: Yamal Peninsula, north Russia.
(a) Geographic location of Yamal
(b) Location of the study Peninsula Map source: google.com area on Yamal (western coast). 
\end{column}
\end{columns}
\end{minipage}
\end{frame}

\subsection{Specific climatic-environmental settings of Yamal}

\begin{frame}\frametitle{Environmental setting}
\begin{minipage}[0.4\textheight]{\textwidth}
\begin{columns}[T]
\begin{column}{0.5\textwidth}
Yamal Peninsula: geomorphology : flat geomorphology, elevations lower than 90 m
Processes:
 \begin{itemize}
        	\item seasonal flooding,
	\item active erosion processing,
	\item permafrost distribution,
	\item cryogenic landslides formation
\end{itemize}
Landslides affect local ecosystem structure. \\
Landslides change vegetation types recovering after the disaster.
\end{column}
\begin{column}{0.5\textwidth}
Landscapes of Yamal. 
\begin{figure}[H]
	\centering
		\includegraphics[width=5.0cm]{f04.png}
\end{figure}
Source: http://pixtale.net/
\end{column}
\end{columns}
\end{minipage}
\end{frame}

\subsection{Landscapes of the Yamal Peninsula}

\begin{frame}\frametitle{Landscapes of the Yamal Peninsula - I}
\begin{figure}[H]
	\centering
		\includegraphics[width=10cm]{f05.png}
\end{figure}
\end{frame}

\begin{frame}\frametitle{Landscapes of the Yamal Peninsula - II}
\begin{figure}[H]
	\centering
		\includegraphics[width=11.0cm]{f06.png}
\end{figure}
Dry grass heath tundra (left). Sedge grass tundra (center). Dry short shrub tundra (right)
\begin{figure}[H]
	\centering
		\includegraphics[width=7.0cm]{f07.png}
\end{figure}
Landscapes of Yamal (left). Sphagnum moss (right)
\end{frame}

\begin{frame}\frametitle{Landscapes of the Yamal Peninsula - III}
\begin{figure}[H]
	\centering
		\includegraphics[width=11.0cm]{f08.png}
\end{figure}
Dry short shrub sedge tundra (left). Wetlands (right)
\begin{figure}[H]
	\centering
		\includegraphics[width=7.0cm]{f09.png}
\end{figure}
Short shrub tundra
\end{frame}

\section{Methods} 
\subsection{Methodology: ILWIS GIS}
\begin{frame}\frametitle{Methodology: ILWIS GIS}
\begin{minipage}[0.4\textheight]{\textwidth}
\begin{columns}[T]
\begin{column}{0.5\textwidth}
Technical tools: The RS data processing was performed in ILWIS GIS software.
Research Methods: 
\begin{itemize}
	\item Image interpretation ( Landsat TM scenes).
	\item NDVI calculation.
	\item Producing vegetation indices has been done in this research using mathematic calculation of the channels
	\item The formulae are: (NIR-VIS) / (NIR+VIS), or a ratio for channels: (Band4-Band3) / (Band4 +Band3).
	\item NDVI values lie in the range of 0 – 1 and never become negative or extend over 1, since NDVI is a linear algebraic function of these bands
\end{itemize}
\end{column}
\begin{column}{0.5\textwidth}
\begin{figure}[H]
	\centering
		\includegraphics[width=5.0cm]{f10.png}
\end{figure}
ILWIS GIS: https://52north.org/software/software-projects/ilwis/
\end{column}
\end{columns}
\end{minipage}
\end{frame}

\subsection{Advantages of the NDVI}
\begin{frame}\frametitle{Advantages of the NDVI}
Calculation of vegetation indices, especially and in this case Normalized Difference Vegetation Index (NDVI), has become one of the most successful, popular and traditional attempts in biogeographical research methods.
\begin{itemize}
            \item NDVI has certain advantages over other vegetation indices or band combinations.
            \item NDVI is less depending on soil properties of study area
            \item NDVI is less depending on the daytime illumination comparing to simple red-infrared bands combination
            \item NDVI is well adjusted specially for the analysis of vegetation properties
            \item NDVI can be indirectly interpreted from the objects colors, as shown on the raster image
\end{itemize}
\end{frame}

\subsection{Workflow}
\begin{frame}\frametitle{Workflow}
Data pre-processing
\begin{itemize}
            \item Import .img file into ASCII raster format (GDAL). 
            \item After converting, each image contained collection of 7 raster bands
            \item Pre-processing (visual color and contrast enhancement)
            \item Geographic referencing of Landsat scenes, initially based on WGS 1984 datum.
            \item Georeference Corner Editor
            \item UTM (Universal Transverse Mercator) Projection, Eastern Zone 42, Northern Zone W.
            \item Crop of study area: the area of interest (AOI) was identified and cropped on the raw images. 
            \item Selected area shows Bovanenkovo region in a large scale 
            \item Environmentally, AOI best represents typical tundra landscapes.
            \item NDVI Calculation
            \item GIS visualization and mapping
\end{itemize}
\end{frame}

\subsection{Import and data conversion}
\begin{frame}\frametitle{Import and data conversion}
\begin{itemize}
        \item Test area selection (Mask): 67\degree 00' - 72\degree 00' E - 70\degree 00' - 71\degree 00' N. 
	\item3 selected  Landsat TM satellite images show Yamal region in 1988, 2001, 2011.
	\item Time span: 23 years (1988, 2001, 2011). 
	\item Summer months selected for vegetation assessment. 
	\item Data conversion / original images in format .TIFF converted to Erdas Imagine .img.
\end{itemize}
\begin{figure}[H]
	\centering
		\includegraphics[width=10.0cm]{f11.png}
\end{figure}
Initial remote sensing data, left to right: Landsat TM 1988, bands 7-3-1; Landsat TM 2011, pseudo natural colors composite; Landsat ETM + 2001 bands 6-3-1.
\end{frame}

\subsection{Georeferencing}
\begin{frame}\frametitle{Georeferencing: Google Earth}
\begin{figure}[H]
	\centering
		\includegraphics[width=10.0cm]{f19.png}
\end{figure}
\end{frame}

\subsection{NDVI Calculation}
\begin{frame}\frametitle{NDVI Calculation}
\begin{itemize}
	\item To model the NDVI I used Map Calculation tool in command line of the Raster Operations menu in ILWIS GIS. 
	\item ILWIS GIS enables to perform spatial analysis and modeling by combination of queries, arithmetic expressions and overlays of selected raster images. 
	\item The NDVI was calculated using following equation: NDVI = (Band4 - Band3) / (Band4 + Band3),
	\item Band 4 is DN values of spectral reflectance in NIR (near infra-red) and Band 3 is DN values of spectral reflectance in VIS (visual).
	\item The NDVI was calculated automatically in ILWIS GIS using arguments of images: VegIndex = NDVI (Band3, Band4). 
	\item Two Landsat bands have been used: Band4, containing red reflectance and Band3 with infra-red reflectance: VegIndex=NDVI(TM\_3,TM\_4).
	\item Other Landsat scenes were classified individually using the same method scheme in ILWIS GIS in the same way described above.
\end{itemize}
\end{frame}

\section{Results} 

\begin{frame}\frametitle{Results}
The resulting images shows distribution of the vegetation over the Bovanenkovo region within three (3) years: 1988, 2001 and 2011.
\begin{figure}[H]
	\centering
		\includegraphics[width=10.0cm]{f20.png}
\end{figure}
 \begin{itemize}
	\item The results for 2001 show that vegetation has very moderate overall index, reaching NDVI value 0,50 as a maximal.
	\item The maximal NDVI values in year 2011 are 0.49,
	\item The maximal NDVI values in year 1988 was 0.76.
\end{itemize}
Comparison of these results shows gradual decrease in the biomass values during the past two decades.
\end{frame}

\section{Discussion} 
\begin{frame}\frametitle{Conclusion and Discussion}
 \begin{itemize}
        	\item This research presented GIS based studies of environmental of Yamal
	\item The study is technically supported by means of ILWIS GIS which proved to be effective tool for spatial analysis
	\item Performed analysis demonstrated changes in the NDVI during period of 1988 - 2011, calculated on the on the Landsat TM images
	\item The results show decrease in overall NDVI values for the study area, which is caused by the environmental and anthropogenic factors
	\item The results of spatial analysis are presented as 3 GIS maps illustrating changes in vegetation based on the image analysis using NDVI.
	\item The calculated NDVI indicated biomass and can be also used as indicator of “greenness” of the vegetation.
	\item Application of RS data is especially important for studies of northern ecosystems, as it enables study of remotely located areas of Arctic
	\item GIS-based processing of the RS data ( Landsat TM) improves technical aspects of the landscape studies and monitoring
\end{itemize}
\end{frame}

\section{Thanks}
\begin{frame}{Thanks}
  	\centering \Huge 
  	\emph{Thank you for attention !}
\end{frame}

%%%%%%%%%%% Bibliography %%%%%%%
\section{Bibliography}
\Huge{Bibliography}
\nocite{*}
\printbibliography

	
%%%%%%%%%%% Bibliography %%%%%%%	
\end{document}