\documentclass[pdflatex,compress,8pt,
	xcolor={dvipsnames,dvipsnames,svgnames,x11names,table},
	hyperref={colorlinks = true,
	breaklinks = true, urlcolor = NavyBlue, breaklinks = true}]{beamer}	
%\usetheme{Ilmenau}%Copenhagen
\usetheme{Klope}
%\usecolortheme[named=DarkOliveGreen3]{structure}
%\usefonttheme{professionalfonts}
%\usepackage{palatino}
%\usepackage{courier}
%\usepackage{lmodern}
%\usepackage{textcomp}
\usefonttheme{structureitalicserif}

% ----------------------------------------------------------------------------
% *** START BIBLIOGRAPHY <<<
% ----------------------------------------------------------------------------
\usepackage[
	backend=biber, 
%	style = numeric,
	style = phys,
	maxbibnames=99,
	citestyle=numeric,
	giveninits=true,
	isbn=true,
	url=true,
	natbib=true,
	sorting=ndymdt,
	bibencoding=utf8,
	useprefix=false,
	language=auto, 
	autolang=other,
	backref=true,
	backrefstyle=none,
	indexing=cite,
]{biblatex}
\DeclareSortingTemplate{ndymdt}{
  \sort{
    \field{presort}
  }
  \sort[final]{
    \field{sortkey}
  }
  \sort{
    \field{sortname}
    \field{author}
    \field{editor}
    \field{translator}
    \field{sorttitle}
    \field{title}
  }
  \sort[direction=descending]{
    \field{sortyear}
    \field{year}
    \literal{9999}
  }
  \sort[direction=descending]{
    \field[padside=left,padwidth=2,padchar=0]{month}
    \literal{99}
  }
  \sort[direction=descending]{
    \field[padside=left,padwidth=2,padchar=0]{day}
    \literal{99}
  }
  \sort{
    \field{sorttitle}
  }
  \sort[direction=descending]{
    \field[padside=left,padwidth=4,padchar=0]{volume}
    \literal{9999}
  }
}

\addbibresource{Bialowieza.bib}%  \scriptsize \footnotesize
\renewcommand*{\bibfont}{\tiny} % 

\setbeamertemplate{bibliography item}{\insertbiblabel}

% ----------------------------------------------------------------------------
% *** END BIBLIOGRAPHY <<<
% ----------------------------------------------------------------------------

% ----------------------------------------------------------------------------
% делать footnote \title[Short Title]{Long Title}
\makeatletter
\setbeamertemplate{footline}{%
\leavevmode%
\hbox{\begin{beamercolorbox}[wd=.24 \paperwidth,ht=2.5ex,dp=1.125ex,leftskip=.01cm plus1fill,rightskip=.05cm]{author in head/foot}%
            \usebeamerfont{title in head/foot}\insertshortauthor
    \end{beamercolorbox}%
    \begin{beamercolorbox}[wd=.76\paperwidth,ht=2.5ex,dp=1.125ex,leftskip=.05cm,rightskip=.15cm plus1fil]{title in head/foot}%
        \usebeamerfont{title in head/foot}\insertshorttitle{}
        \insertframenumber{} / \inserttotalframenumber \ \hspace*{2ex} 
    \end{beamercolorbox}}%
    \vskip0pt%
}
\makeatother

% ----------------------------------------------------------------------------

% Путь к файлам с иллюстрациями
\graphicspath{{fig/}} % path to folder with Figures

\usepackage{gensymb} % degree symbol
\usepackage[super]{nth}
\usepackage{amsmath}
\usepackage{subfig}

%%%%%%%%%%%%%%%%%%%%%%%%%%%%

\title[The Environment of the Białowieza National Park and Sudety Mountains]{The Environment of the Białowieza National Park and Sudety Mountains}
\subtitle{Presentation at GEM MSc Course, \\
Environmental Management and Policy in Central Europe\\
Erasmus Mundus Scholarship}
\institute{University of Warsaw}

\author{Polina Lemenkova}
         \date{June 20, 2010}

\begin{document}
\begin{frame}
           \titlepage
\end{frame}

\section*{Outline}
\begin{frame}
\vspace{5em}
           \tableofcontents
\end{frame}
         
\section{Białowieza National Park}
\subsection{Geographic Location}
\begin{frame}\frametitle{Introduction}
\begin{minipage}[0.4\textheight]{\textwidth}
\begin{columns}[T]
\begin{column}{0.6\textwidth}

\vspace{4em}
\begin{alertblock}{Location}
Białowieza National Park (BNP) location: 52\degree 41' 52\degree 59'N 23\degree 43' 23\degree 56' E. \\
BNP is a World Heritage Site located in north-east Poland on the border with Belarus.\\
BNP belongs to the Bialostockie administrative region. Altitude: 145m to 202m.\\
BNP is in UNESCO World Heritage List, 2008. 
\end{alertblock}

\begin{block}{Boundaries}
\begin{itemize}
	\item North: BNP is bounded by the Hwozna
	\item West: and Narewka Rivers
	\item East: Belovezhskaja Puscha National Park, Belarus
	\item South: national forests
\end{itemize} 
\end{block}

\begin{examples}{Situation}
 BNP lies between western Belarus about 60 km north-northwest of Brest and eastern Poland, 62 km south-east of Bialystok, 190 km north-east of Warsaw.
\end{examples}   

\end{column}
\begin{column}{0.4\textwidth}
\vspace{5em}
\begin{figure}[H]
	\centering
		\includegraphics[width=4.0cm]{F1.jpg}
\end{figure}
\emph{Photos: author.}
\end{column}
\end{columns}
\end{minipage}
\end{frame}

\subsection{History}
\begin{frame}\frametitle{History}
\vspace{3em}
\small{\begin{itemize}
	\item 14th century: limited hunting rights were granted throughout the forest;
	\item 1538: first record of legal protection;
	\item 1541: declared as a hunting reserve;
	\item 1557: establishing a board governing the forest usage;
	\item 1639: another record of legal protection under "The Białowieza Royal Forest Decree" (‘‘Ordynacja Puszczy J.K. Mości leśnictwa Białowieskiego'');
	\item 1801: reintroduction of the forest protection (after ca. 15 years of intensive hunting);
	\item 1860: reintroduction of the protection of European bison (\emph{Bison bonasus});
	\item 1921: Puszcza Białowieska was declared a National Reserve, as a forestry reservation;
	\item 1929: reintroduction of European bison (\emph{Bison bonasus});
	\item 1932: designated a national park;
	\item 1944/47: re-establishing the Belovezhskaya Pushcha National Park and BNP;
	\item 1976: internationally recognized as a Biosphere Reserve under UNESCO \emph{Man and the Biosphere} Program (191,300 ha);
	\item 1979: inscribed on the World Heritage List and extended in 1999.
	\item 2002: Belovezhskaya Pushcha National Park (BPNP) in Belarus enlarged to 100,312 ha with a buffer zone of 92,000 ha
	\item 2004: The forest area in Belarus enlarged to 152,200 ha with a core area	
\end{itemize}}
\end{frame}

\subsection{Governance and Protection}
\begin{frame}\frametitle{Governance and Protection of BNP}

\begin{minipage}[0.4\textheight]{\textwidth}
\begin{columns}[T]
\begin{column}{0.7\textwidth}
\vspace{4em} 
\begin{alertblock}{Poland}
The site is in the voivodship of Podlasie, administered by the BNP Management under the General Board of National Parks within the Ministry of the Environment.
\end{alertblock}

\begin{block}{Belorussia}
Forest and Game Hunting Department of the Ministry of Natural Resources \& Environmental Protection is responsible for conservation of biodiversity and natural resources. 
\end{block}

\begin{examples}{Cooperation:}
Since 1991, Poland and Belarus work together on BNP management. The director of the Belarusian Park was nominated as a member of the Scientific Council of the Polish BNP.
\end{examples}

\begin{block}{UNESCO World Heritage List}
In Poland 4,747 ha of the World Heritage site (11\% of the Polish forest) are a Strict Nature Reserve and adjoin 12,000 ha of protected areas. In 1992 the whole transboundary forest was 150,000 ha (ca. 87,600 ha in Belarus and 62,400 ha in Poland).
\end{block}

\end{column}
\begin{column}{0.3\textwidth}
\vspace{4em}
Land Tenure
\begin{figure}[H]
	\centering
		\subfloat {\includegraphics[width=3.0cm]{F2.jpg}}
			\vspace{2mm}
		\subfloat {\includegraphics[width=3.0cm]{F3a.jpg}}
			\vspace{2mm}
		\subfloat {\includegraphics[width=3.0cm]{F3b.jpg}}
\end{figure}
\scriptsize{\emph{Photos: author.}}
\end{column}
\end{columns}
\end{minipage}
\end{frame}

\subsection{Physical Settings}
\begin{frame}\frametitle{Physical Settings}
\begin{minipage}[0.4\textheight]{\textwidth}
\begin{columns}[T]
\begin{column}{0.5\textwidth}
\vspace{2em}
\begin{figure}[H]
	\centering
		\subfloat {\includegraphics[width=2.0cm]{F4.jpg}}
			\hspace{2mm}
		\subfloat {\includegraphics[width=2.0cm]{F5.jpg}}
			\vspace{2mm}
		\subfloat {\includegraphics[width=3.0cm]{F6.jpg}}
\end{figure}
\small{\emph{Photos: author.}}
\begin{alertblock}{Geomorphology}
Relief flat to rolling lowland plain on the hydrological divide between the Baltic and Black Seas forming a mosaic of peat bogs, stream and river valleys.
\end{alertblock}
\end{column}

\begin{column}{0.5\textwidth}
\vspace{4em} 

\begin{alertblock}{Soils and Hydrology}
 It is covered by glacial formations with deposits of deep sands overlying clays and loams, podsols and bog soils. The forest is drained by the River Orlowka, a tributary of the northward-flowing Narewka river which crosses northern part of the BNP in Belarus. The Pavaya Lisnaya river drains southern part of BNP. 
\end{alertblock}

\begin{block}{Climate}
Cool-temperate continental climate. Mean annual precipitation: 620mm, 2/3 falls between April and October.
Mean annual temperature: 7\degree C, average January and July T -5\degree C and 18\degree C. 
\end{block}

\begin{examples}{Snow}
Snow cover persists for an average of 92 days/yr between mid-October and end of April. Conditions favouring plant growth occurs for 205 days/yr.
\end{examples}
\end{column}
\end{columns}
\end{minipage}
\end{frame}

\subsection{Vegetation}
\begin{frame}\frametitle{Vegetation}
\begin{minipage}[0.4\textheight]{\textwidth}
\begin{columns}[T]
\begin{column}{0.4\textwidth}
\vspace{2em}
\begin{figure}[H]
	\centering
		\subfloat {\includegraphics[width=3.0cm]{F7b.jpg}}
			\vspace{2mm}
		\subfloat {\includegraphics[width=3.0cm]{F8a.jpg}}
			\vspace{2mm}
		\subfloat {\includegraphics[width=3.0cm]{F8b.jpg}}
\end{figure}
\small{\emph{Photos: author.}}
\end{column}

\begin{column}{0.6\textwidth}
\vspace{4em} 
\begin{block}{Vegetation}
Humid western European type with intermixed northern and southern elements. 12 major forest associations. 
\end{block}

\begin{alertblock}{Dominant communities}
Typical east European linden-hornbeam Tilio-Carpinetum and typical central European oak-hornbeam Querco-Carpinetum.
\end{alertblock}

\begin{examples}{Species}
Principal forest species include following species:
\end{examples}
\begin{itemize}
	\item Scots pine Pinus silvestris
	\item Norway spruce Picea abies
	\item oak Quercus robur
	\item sycamore Acer platanoides
	\item maple Acer spp
	\item ash Fraxinus excelsior
	\item white birch Betula pubescens
	\item aspen Populus tremula
	\item black alder Alnus glutinosa
\end{itemize}

\end{column}
\end{columns}
\end{minipage}
\end{frame}

\subsection{Fauna of Białowieza}
\begin{frame}\frametitle{Fauna of Białowieza}

\begin{minipage}[0.4\textheight]{\textwidth}
\begin{columns}[T]
\begin{column}{0.5\textwidth}
\vspace{5em}
\begin{figure}[H]
	\centering
		\subfloat {\includegraphics[width=2.0cm]{F7a.jpg}}
			\hspace{2mm}
		\subfloat {\includegraphics[width=2.0cm]{F9.jpg}}
			\vspace{2mm}
		\subfloat {\includegraphics[width=2.0cm]{F10.jpg}}
			\hspace{2mm}
		\subfloat {\includegraphics[width=2.0cm]{F11.jpg}}
\end{figure}
\emph{Photos: author.}
\end{column}
\begin{column}{0.5\textwidth}
\vspace{4em} 
\begin{block}{Animals}
There are over 11,500 different species of animals in Polish part of Białowieza (Society for Conservation Biology, 2003) 
68 species of mammals: European bison, grey wolf and lynx, common otter, European beaver. 300 animals, along with the reintroduced tarpan. 
\end{block}

\begin{alertblock}{Common species}
Wild boar, elk, red deer and roe deer. The bison, exterminated in the forest in 1919, was re-established in 1929. 
\end{alertblock}

\begin{examples}{\textcolor{DarkGreen}{Birds}:}
251 species are noted in BNP. 120 birds include following species: black stork, crane, pygmy owl and eagle owl, a large number of raptors such as Pomeranian eagle, greater spotted eagle, booted eagle, three-toed woodpecker
\end{examples}
\end{column}
\end{columns}
\end{minipage}
\end{frame}

\subsection{People}
\begin{frame}\frametitle{People}

\begin{alertblock}{Activities}
Human activity include clearings, hunting grounds, riverside meadows, road systems and trails, forest settlements, narrow-gauge railways, felling sites, and gravel-pits.
\end{alertblock}

\begin{examples}{Visitors}
In the Belarusian sector visitors averaged 50-60,000 a year, during the 1990s, and 83,000 visitors were recorded in the first nine months of 2004.
In the Polish section there were about 80,000 visitors in 1997, 30\% of whom visit the strict preservation area where access is limited to guided groups. 
\end{examples}

\begin{block}{Settlements}
Living in the strict preservation area is prohibited but some 3,000 people live in villages nearby. Within the extended forest there are 22 villages with some 30,000 inhabitants (UESCO/IUCN, 2004). They are mostly farmers, growing potatoes, rye, wheat, oats, barley, rape and sugar-beet. The nearest town is Kamenyuki (MAB-Belarus, 1993). In Poland the village of Bialowieza is located 1 km from the Park.
\end{block}

\end{frame}

\subsection{Research}
\begin{frame}\frametitle{Research}

\vspace{3em}
\begin{block}{Cultural Heritage}
184 old Slav burial tumuli from the \nth{10} and \nth{11} AD have been found. Slav tribes are settled in this area between \nth{10} and \nth{13} AD.
\end{block}

\begin{block}{Archaeology}
Traces of \nth{18} to \nth{19} AD forest bee-keeping are visible on some 100 pines in the core area. Palace Park, a former imperial Russian hunting lodge near the village of Bialowieza, dates from the 1890s, where a monument commemorates the hunting by Augustus.
\end{block}

\begin{alertblock}{Development}
The Belovezhskaya Pushcha was first covered with a net of exploitation tracks, and subsequently studied scientifically, in the mid 19th century.
Ongoing research includes natural ecosystems and their restoration, natural succession, forest management, agricultural research, and floral and faunal surveys.
\end{alertblock}

\begin{block}{Research Centre}
Research is also planned for the social sciences, in particular ethno-biology, cultural anthropology, rural technology and traditional land-use systems.
There is now a scientific research centre laboratory situated near the Park headquarters at Kamieniuki.
\end{block}

\end{frame}

\section{Sudety Mountains}
\begin{frame}\frametitle{Sudety Mountains}
\vspace{2em}
\begin{alertblock}{Linguistics}
The name Sudetes has been derived from Sudeti montes.
Sudetes consist of 3 parts: Western Sudetes, Central Sudetes, Eastern Sudetes
\end{alertblock}

\begin{block}{Location}
The \nth{2} part of the research lies in Western Sudetes, Karkonosze Mountains and Izera Mountains. \\
Karpacz - one of the most notable small towns along the border of the Czech Republic and Poland, extending ca.185 mi (300 km) between the Elbe and Oder rivers, Erzgebirge and Carpathians
\end{block}

\begin{figure}[H]
	\centering
		\subfloat {\includegraphics[width=5.2cm]{F12.jpg}}
			\hspace{5mm}
		\subfloat {\includegraphics[width=5.0cm]{F13.jpg}}
\end{figure}

\end{frame}

\subsection{Geographic Settings}
\begin{frame}\frametitle{Geographic Settings}
\begin{minipage}[0.4\textheight]{\textwidth}
\begin{columns}[T]
\begin{column}{0.3\textwidth}
\vspace{6em}
\begin{figure}[H]
	\centering
		\subfloat {\includegraphics[width=3.0cm]{S1.jpg}}
			\vspace{2mm}
		\subfloat {\includegraphics[width=3.0cm]{S2.jpg}}
\end{figure}
\small{Sudety mountains. \\Source: Web}
\end{column}
\begin{column}{0.7\textwidth}
\vspace{3em} 
\begin{alertblock}{Karkonosze National Park}
Nature Protection (Karkonoski Park Narodowy, KPN), created in 1959.\\
Area: 55.8 $km^{2}$. Highly sensitive higher parts of the mountain range (altitude of 900– 1000m) and some special nature reserves below this zone
\end{alertblock}

\begin{block}{Geology}
Granite, schist, shale and calcite
\end{block}

\begin{block}{Tectonics}
Caledonian, Varescan. Period: Neoproterozoic, Palaeozoic.
\end{block}

\begin{examples}{Vegetation}
\begin{itemize}
	\item Alpine vegetation zone - 1,400 m: large rocky deserts
	\item Subalpine zone above timber line 1,250 - 1,350 m: knee timber, mountain grass meadows, subarctic highmoor, alpine grasslands
	\item Boreal zone: spruce, mixed forest
\end{itemize}
\end{examples}
\end{column}
\end{columns}
\end{minipage}
\end{frame}

\subsection{Karkonosze Mountains}
\begin{frame}\frametitle{Karkonosze Mountains}
\begin{minipage}[0.4\textheight]{\textwidth}
\begin{columns}[T]
\begin{column}{0.4\textwidth}
\vspace{8em}
\begin{figure}[H]
	\centering
		\includegraphics[width=4.0cm]{F14.jpg}
\end{figure}
\end{column}
\begin{column}{0.6\textwidth}
\vspace{6em} 
Karkonosze Mountains is a part of the Sudety Mountains. 
\begin{itemize}
	\item Karkonosze Mountains – the highest and largest area in Sudety
	\item Karkonosze National Park was established in 1959.
	\item This mountainous region belongs to most valuable landscapes and natural regions in Central Europe 
	\item Area: 5,575 ha
	\item 70\% of area (3,828 ha) are covered by forests
	\item The area under strict protection: alpine and sub-alpine zones (1,717 ha, ca 30\%)
	\item The area under partial protection\
	\item All other areas aimed on restoration of damaged or destroyed forests 
	\item Around the park – protected zone of 11,266 
\end{itemize}
\end{column}
\end{columns}
\end{minipage}
\end{frame}

\subsection{Environmental Protection}
\begin{frame}\frametitle{Environmental Protection}
\begin{minipage}[0.4\textheight]{\textwidth}
\begin{columns}[T]
\begin{column}{0.5\textwidth}
%\vspace{2em}
\begin{figure}[H]
	\centering
		\subfloat {\includegraphics[width=2.0cm]{F15.jpg}}
			\hspace{5mm}
		\subfloat {\includegraphics[width=2.0cm]{F16.jpg}}
			\vspace{3mm}
		\subfloat {\includegraphics[width=2.0cm]{F17.jpg}}
			\hspace{5mm}
		\subfloat {\includegraphics[width=2.0cm]{F18.jpg}}

\end{figure}
\end{column}
\begin{column}{0.5\textwidth}
\vspace{4em} 
Objectives of the Environmental Protection:
\begin{itemize}
	\item Maintenance of natural ecosystems 
	\item Protection of biodiversity 
	\item Restoration of damaged and destroyed phyto- and zoocenoses
	\item Reconstruction of forests and natural ecological processes
	\item Limitation of anthropogenic threats
	\item Protection of unique beauty of Karkonosze Mountains
\end{itemize}
\end{column}
\end{columns}
\end{minipage}
\end{frame}

\subsection{Flora and Fauna}
\begin{frame}\frametitle{Flora and Fauna}
\vspace{3em}
Vertical vegetation zonation:
\begin{enumerate}
	\item Foothill zone (up to 500 m): oak forest, coniferous forest 
	\item Lower sub-alpine forest (500-1000 m): beech forest
	\item Upper sub-alpine forest (1000-1250 m): Sudetic spruce forest 
	\item Sub-alpine shrub (1250-1450 m): Sudetic dwarf pine shrub
	\item Alpine zone (1450-1603 m): mountain meadows
\end{enumerate}

Vegetation uniqueness:
\begin{itemize}
	\item [*]The specific feature of the Karkonosze fauna: proportion of Carpathian species
	\item [*]Nearly total absence of heat-loving species from Mediterranean, Balkans and Black Sea due to the climatic conditions of the Karkonosze Mountains
\end{itemize}

\begin{alertblock}{Animal species}
Wolves, bears, lynxes, wild cats, red deer, roe-deer, fox.
\end{alertblock}

\begin{block}{Birds}
House sparrow, magpie, pygmy owl, hazel hen, white-tailed eagle
\end{block}

\begin{block}{Reptiles}
Common lizard, sand lizard, grass snake
\end{block}

\end{frame}

\section{Sources}
\begin{frame}\frametitle{Sources}
\begin{figure}[H]
	\centering
		\includegraphics[width=8.0cm]{F19.jpg}
\end{figure}
\end{frame}

\section{Thanks}
\begin{frame}{Thanks}
  	\centering \LARGE 
  	\emph{Thank you for attention !}\\
\end{frame}

%%%%%%%%%%% Bibliography %%%%%%%
\section{Bibliography}
\begin{frame}\frametitle{Bibliography}
\vspace{4em}
\scriptsize{Author's publications on geography, geoscience and environment: \cite{Lemenkova2006e}, \cite{Lemenkova2006b}, \cite{Lemenkova2006a}, \cite{Lemenkova2007b}, \cite{Lemenkova2004a}, \cite{Lemenkova2008b},  \cite{Lemenkova2005b1}, \cite{Lemenkova2005a}, \cite{Lemenkova2002b}.}
%\nocite{*}
\printbibliography[heading=none]
\end{frame}

%%%%%%%%%%% Bibliography %%%%%%%	

%Changing the font size locally (from biggest to smallest):	
%\Huge
%\huge
%\LARGE
%\Large
%\large
%\normalsize (default)
%\small
%\footnotesize
%\scriptsize
%\tiny


\end{document}