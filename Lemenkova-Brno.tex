\documentclass[pdflatex,compress,8pt,
	xcolor={dvipsnames,dvipsnames,svgnames,x11names,table},
	hyperref={colorlinks = true,breaklinks = true, urlcolor = NavyBlue, breaklinks = true}]{beamer}
%\usetheme{Szeged}

\usetheme[progressbar=frametitle]{metropolis}
%\useoutertheme{split}
%\usecolortheme{albatross}
\usecolortheme[named=SeaGreen]{structure}

%\setbeamertemplate{background}[grid][step=1cm]

\usepackage[utf8]{inputenc}
\usepackage[T2A,T1]{fontenc}
%\usefonttheme{structuresmallcapsserif}
%\usefonttheme{serif}
\usepackage{bookman}

% \setbeamertemplate{itemize item}{$\Rightarrow$}
 %   \setbeamertemplate{itemize item}[double arrow]
 
 \setbeamertemplate{itemize items}[circle]
 
 %%%%%%%%%%%%%%%%%%%%%%%%%%%%%%%%%%%

\usepackage{gensymb} % degree symbol
\usepackage[super]{nth}
%%%%%%%%%%%%%%%%%%%%%%%%%%%%%%%%%%%
% ----------------------------------------------------------------------------
% *** START BIBLIOGRAPHY <<<
% ----------------------------------------------------------------------------
\usepackage[
	backend=biber, 
%	style = numeric,
	style=apa,
	maxbibnames=99,
%	citestyle=authoryear,
	citestyle=numeric,
	giveninits=true,
	isbn=true,
	url=true,
	natbib=true,
	sorting=ndymdt,
	bibencoding=utf8,
	useprefix=false,
	language=auto, 
	autolang=other,
	backref=true,
	backrefstyle=none,
	indexing=cite,
]{biblatex}
\DeclareSortingTemplate{ndymdt}{
  \sort{
    \field{presort}
  }
  \sort[final]{
    \field{sortkey}
  }
  \sort{
    \field{sortname}
    \field{author}
    \field{editor}
    \field{translator}
    \field{sorttitle}
    \field{title}
  }
  \sort[direction=descending]{
    \field{sortyear}
    \field{year}
    \literal{9999}
  }
  \sort[direction=descending]{
    \field[padside=left,padwidth=2,padchar=0]{month}
    \literal{99}
  }
  \sort[direction=descending]{
    \field[padside=left,padwidth=2,padchar=0]{day}
    \literal{99}
  }
  \sort{
    \field{sorttitle}
  }
  \sort[direction=descending]{
    \field[padside=left,padwidth=4,padchar=0]{volume}
    \literal{9999}
  }
}

\addbibresource{Brno.bib}
\renewcommand*{\bibfont}{\tiny} % \tiny \scriptsize \footnotesize

\setbeamertemplate{bibliography item}{\insertbiblabel}

% ----------------------------------------------------------------------------
% *** END BIBLIOGRAPHY <<<
% ----------------------------------------------------------------------------
% Путь к файлам с иллюстрациями
\graphicspath{{fig/}} % path to folder with Figures

% ----------------------------------------------------------------------------
% делать footnote \title[Short Title]{Long Title}
\makeatletter
\setbeamertemplate{footline}{%
\leavevmode%
\hbox{\begin{beamercolorbox}[wd=.24 \paperwidth,ht=2.5ex,dp=1.125ex,leftskip=.01cm plus1fill,rightskip=.05cm]{author in head/foot}%
            \usebeamerfont{title in head/foot}\insertshortauthor
    \end{beamercolorbox}%
    \begin{beamercolorbox}[wd=.76\paperwidth,ht=2.5ex,dp=1.125ex,leftskip=.05cm,rightskip=.15cm plus1fil]{title in head/foot}%
        \usebeamerfont{title in head/foot}\insertshorttitle{}
        \insertframenumber{} / \inserttotalframenumber \ \hspace*{2ex} 
    \end{beamercolorbox}}%
    \vskip0pt%
}
\makeatother

% ----------------------------------------------------------------------------

%%%%%%%%%%%%%%%%%%%%%%%%%%%%%%%%%


\title[Measuring regional difference in urban growth of Taipei city (Taiwan, China) by ENVI GIS]{Measuring regional difference in urban growth of \\
Taipei city (Taiwan, China) by means of \\ENVI GIS and
remote sensing data}

\subtitle{\vspace*{0.5cm}Presented at \\
Post Graduate Students (PGS) Conference \\
CzechGlobe Research Institute CAS\\
Brno, Czech Republic}

\author{Polina Lemenkova}
\date{February 6, 2014}

\begin{document}
\begin{frame}
           \titlepage
\end{frame}

\section*{Outline}
\begin{frame}
        \scriptsize{   \tableofcontents}
\end{frame}

\section{Techniques}
\begin{frame}\frametitle{Techniques}

\begin{alertblock}{ENVI GIS}
The research was performed using ENVI GIS software using Landsat TM images for years 1990 and 2005.
\end{alertblock}

\begin{block}{K-means Algorithm}
The landscapes in the study area at two multi-temporal Landsat TM images were classified using "K-means" algorithm. Different land use types were identified and classified. The area covered by each land cover class is compared for years 1990 and 2005 and dynamics analyzed.
\end{block}

\begin{block}{Land Cover Types}
Changes in the selected land cover types were analyzed and human impacts on the natural landscapes detected. Classified land cover types were compared at both maps of land cover types for the years 1990 and 2005.
\end{block}

\end{frame}

\section{Data}
\begin{frame}\frametitle{Data}
\begin{figure}[H]
	\centering
		\includegraphics[width=9.0cm]{F1.jpg}
\end{figure}

\begin{alertblock}{Source}
Data: two Landsat TM imagery taken from the USGS website. 
\end{alertblock}

\begin{block}{Temporal comparison}
Study period: imagery for summer periods of 1999-2005 
\end{block}

\begin{examples}{Spatial Comparison}
Selected territory focused on \alert{3 different areas}: comparison of three districts of southern area of Taipei, Taiwan, along the Tamsui river
\end{examples} 
\end{frame}

\subsection{Landsat TM Images}
\begin{frame}\frametitle{Landsat TM 1990}
Landsat TM image (1990). Bands 1-7 (fragment). Color composites (ENVI)
\begin{figure}[H]
	\centering
		\includegraphics[width=9.0cm]{F4.jpg}
\end{figure}
\end{frame}

\begin{frame}\frametitle{Landsat TM 2005}
Landsat TM image (2005). Bands 1-7 (fragment). Color composites (ENVI)
\begin{figure}[H]
	\centering
		\includegraphics[width=9.0cm]{F5.jpg}
\end{figure}
\end{frame}

\subsection{Preliminary Data Processing}
\begin{frame}\frametitle{Preliminary Data Processing}
\begin{itemize}
	\item Preliminary data processing: image contrast stretching. 
	\item By default, ENVI displays images with a 2\% linear contrast stretch.
	\item Histogram equalization contrast stretch was applied to the images in order to enhance the visual quality (better contrast)
\end{itemize}
\begin{figure}[H]
	\centering
		\includegraphics[width=8.0cm]{F3.jpg}
\end{figure}
\end{frame}

\section{Methods}
\begin{frame}\frametitle{Methods}
\begin{minipage}[0.4\textheight]{\textwidth}
\begin{columns}[T]
\begin{column}{0.5\textwidth}
\begin{figure}[H]
	\centering
		\includegraphics[width=5.0cm]{F2.jpg}
\end{figure}
\end{column}
\begin{column}{0.5\textwidth}
Methods used in the current work include following steps:
\begin{itemize}
	\item Data capture
	\item Color composition from 3 bands
	\item Defining Region of Interest (ROI)
	\item Selection of 3 different regions within the city area Image classification
	\item Combining classes and re-classification Post-processing
	\item Spatial analysis for 3 different areas of Taipei city
\end{itemize}
\end{column}
\end{columns}
\end{minipage}
\end{frame}

\subsection{K-means Algorithm}
\begin{frame}\frametitle{K-means Algorithm}
K-means algorithm procedure has 3 steps:
\begin{itemize}
	\item Choosing the initial \alert{centroids} for defined number of land classes
	\item \alert{Assignation} of each pixel to the nearest centroid which represent certain land cover class
	\item Creating new centroids by taking the mean value of all pixels assigned to each previous land cover \alert{class} (i.e. centroid).
\end{itemize}
\alert{Looping} between the other steps 2 and 3: \\
The difference between the old and the new centroids is '\alert{inertia}'. \\
The algorithm repeats (loops) steps until the 'inertia' is less than a \alert{threshold}.\\
It \alert{smoothes iteratively} the allocation of pixels until the centroids do not move significantly.
\end{frame}
 
\subsection{K-Means Clustering}
\begin{frame}\frametitle{K-Means Clustering}
 K-Means classification: general idea.
\begin{minipage}[0.4\textheight]{\textwidth}
\begin{columns}[T]
\begin{column}{0.5\textwidth}
 \begin{figure}[H]
	\centering
		\includegraphics[width=5.0cm]{F6.jpg}
\end{figure}
K-Means: a mathematical approach:\\\vspace{1em}

$J(X,C)=\sum_{i=0}^n\frac{min}{\mu_{j}\in C} (||x_j-\mu_i||^{2})$

\end{column}
\begin{column}{0.5\textwidth}
\begin{itemize}
	\item  K-means is a flat clustering algorithm which is often used in classification techniques.
	\item The objective of K- means: to minimize the average squared Euclidean distance between the cluster centers (the means).
	\item K-means analysis allocates pixels into various clusters by defining the mathematical centroids of all clusters - groups of pixel with similar values of spectral reflectance, or digital number (DNs)
	\item K-means separates raster pixels in n clusters (groups of equal variance) by minimizing the ‘inertia’ criterion.
\end{itemize}
\end{column}
\end{columns}
\end{minipage}
\end{frame}

\subsection{K-means Classification: ENVI GIS}
\begin{frame}\frametitle{K-means Classification by Means of ENVI GIS}
\begin{figure}[H]
	\centering
		\includegraphics[width=11.0cm]{F8.jpg}
\end{figure}
\end{frame}

\subsection{Land Cover Types}
\begin{frame}\frametitle{Land Cover Types}
Classification of the land cover types in the city:
\begin{itemize}
	\item Forests
	\item Urban areas – 2 (roads)
	\item Grasslands
	\item Open fields (little or no vegetation)
	\item Water areas
	\item Urban vegetation – 1 (bushes)
	\item Cultivated lands
	\item Agricultural facilities
	\item Urban vegetation –2 (parks and squares)
	\item Urban areas -2 (built-up surfaces)
\end{itemize}

\begin{figure}[H]
	\centering
		\includegraphics[width=8.0cm]{F9.jpg}
\end{figure}
\end{frame}


\subsection{ENVI GIS Classification by K-means Method}
\begin{frame}\frametitle{ENVI GIS Classification by K-means Method}
\small{1990 Landsat TM scene classification (fragment). Random color visualization}
\begin{figure}[H]
	\centering
		\includegraphics[width=7.0cm]{F10.jpg}
\end{figure}
\small{2005 Landsat TM scene classification (fragment). Random color visualization}
\begin{figure}[H]
	\centering
		\includegraphics[width=7.0cm]{F11.jpg}
\end{figure}
\end{frame}

\subsection{Combining Classes}
\begin{frame}\frametitle{Combining Classes}
\begin{figure}[H]
	\centering
		\includegraphics[width=11.0cm]{F12.jpg}
\end{figure}
\end{frame}

\subsection{Creating Region Of Interest (ROI)}
\begin{frame}\frametitle{Creating Region Of Interest (ROI)}
\begin{figure}[H]
	\centering
		\includegraphics[width=11.0cm]{F13.jpg}
\end{figure}
\end{frame}

\subsection{Subset Data via ROI}
\begin{frame}\frametitle{ Subset Data via ROI}
\begin{figure}[H]
	\centering
		\includegraphics[width=11.0cm]{F14.jpg}
\end{figure}
\end{frame}

\subsection{Regional Division of Taipei}
\begin{frame}\frametitle{Regional Division of Taipei City}
\begin{figure}[H]
	\centering
		\includegraphics[width=10.0cm]{F15.jpg}
\end{figure}
\end{frame}

\subsection{Statistics of Change Detection}
\begin{frame}\frametitle{Statistics of Change Detection: Process}
\begin{minipage}[0.4\textheight]{\textwidth}
\begin{columns}[T]
\begin{column}{0.7\textwidth}
\begin{figure}[H]
	\centering
		\includegraphics[width=8.0cm]{F16.jpg}
\end{figure}
\end{column}
\begin{column}{0.3\textwidth}
Process workflow:
\begin{itemize}
	\item Change detection statistics
	\item Setting options of process
	\item Choice of 'Initial Stage' image
	\item Choice of 'Final Stage' Image
	\item Defining equivalent classes
	\item Defining pixel size for area statistics
	\item Change detection statistics output
\end{itemize}
\end{column}
\end{columns}
\end{minipage}
\end{frame}

\subsection{Post-Classification Data Processing}
\begin{frame}\frametitle{ Post-Classification Data Processing}
\begin{figure}[H]
	\centering
		\includegraphics[width=11.0cm]{F17.jpg}
\end{figure}
\end{frame}

\subsection{Statistical Processing}
\begin{frame}\frametitle{Statistical Processing}
\begin{figure}[H]
	\centering
		\includegraphics[width=9.0cm]{F18.jpg}
\end{figure}
\begin{itemize}
	\item Left: \alert{Getis Ord G statistics} - measures concentration of the emergency values (highest or lowest) for the land cover classes
	\item Center: \alert{Geary statistics} - measures spatial dependance and autocorrelation of pixels.
	\item Right: \alert{Moran statistics} - measures spatial features similarity values based on pixels location and values
\end{itemize}
\end{frame}

\subsection{Computed Areas of the Land Cover Types}
\begin{frame}\frametitle{Computed Areas of the Land Cover Types}
\begin{minipage}[0.5\textheight]{\textwidth}
\begin{columns}[T]
\begin{column}{0.6\textwidth}
Results: ROI has changes in land cover classes: 
\begin{figure}[H]
	\centering
		\includegraphics[width=7.0cm]{F19.jpg}
\end{figure}
\end{column}
\begin{column}{0.4\textwidth}

\begin{enumerate}
	\item class “urban areas” increased from 16.9\% in 1990 to 21.8\% in 2005
	\item class “urban vegetation” decreased from 3.1\% in 1990 to 2.9\% in 2005
	\item class “forests” decreased from 62.4\% in 1990 to 60.8\% in 2005
	\item class “grasslands” decreased from 15.6\% in 1990 to 10.3\% in 2005
\end{enumerate}
\end{column}
\end{columns}
\end{minipage}
\end{frame}

\section{Results}
\begin{frame}\frametitle{Results: Regions I, II and III}
Three regions of Taipei have differences in the land cover types:
\begin{figure}[H]
	\centering
		\includegraphics[width=8.0cm]{F20.jpg}
\end{figure}
\end{frame}

\section{Conclusions}
\begin{frame}\frametitle{Conclusions}
\begin{minipage}[0.4\textheight]{\textwidth}
\begin{columns}[T]
\begin{column}{0.5\textwidth}
\vspace{2em}
\begin{figure}[H]
	\centering
		\includegraphics[width=5.0cm]{F21.jpg}
\end{figure}
\end{column}
\begin{column}{0.5\textwidth}
\vspace{2em}
Comparison of regions I, II and III:
\begin{itemize}
	\item Region I: located on the left bank of Tamsui river, agricultural area. It has significant changes in the land cover types since 1990s. 
	\item Region II is the core, old city area is the most stable region. It has the least changes: this area is already industrialized for a long time. 
	\item Region III is an area located southwards from the core city. It has undergone notable changes caused by intensive relocation of the population to the Taipei area after 1980s. This is regions is dynamically developing since 1990s.
\end{itemize}
\end{column}
\end{columns}
\end{minipage}
\end{frame}

\section{Discussion}
\begin{frame}\frametitle{Discussion}
\begin{itemize}
	\item Spatial analysis performed by ENVI GIS enabled to process satellite images for urban studies. 
	\item Spatio-temporal analysis was applied to Landsat TM images taken at 1990 and 2005: 
	\item Built-in functions of the mathematical K-means algorithm enabled to classify Landsat TM images and to derive information on land cover types.
	\item Image classification was used to analyze land cover changes in Taipei which includes built-up areas and natural green areas.
	\item Results of image processing and spatial analysis show changes in structure, shape and configuration of urban landscapes in Taipei since 1990
	\item Areas occupied by human activities increased, while natural landscapes undergone modifications.
	\item Changes in urban landscapes of Taipei are caused by the increased relocation of population, urbanization and occupied lands for urban needs.
\end{itemize}
\end{frame}

\section{Thanks}
\begin{frame}{Thanks}
  	\centering \LARGE 
	\emph{Thank you for attention !}\\
	\vspace{5em}
\normalsize
Acknowledgement: \\
Current work has been supported by the \\
Taiwan Ministry of Education Short Term Research Award (STRA) \\
for author's 2-month research stay (April-May 2013) at\\
National Taiwan University (NTU), \\
Department of Geography.
\end{frame}

%%%%%%%%%%% Bibliography %%%%%%%
\section{Bibliography}
\Large{Bibliography}
\nocite{*}
\printbibliography[heading=none]
	
%%%%%%%%%%% Bibliography %%%%%%%	

\end{document}
%Changing the font size locally (from biggest to smallest):	
%\Huge
%\huge
%\LARGE
%\Large
%\large
%\normalsize (default)
%\small
%\footnotesize
%\scriptsize
%\tiny

\end{document}