\documentclass[pdflatex,compress,10pt,
	xcolor={dvipsnames,dvipsnames,svgnames,x11names,table},
	hyperref={
	colorlinks = true,
	breaklinks = true, 
	citecolor=NavyBlue, 
	urlcolor = blue, 
%	urlbordercolor=magenta,
	filecolor=magenta} 
]{beamer}	
\usetheme{JuanLesPins}
\usecolortheme[named=Turquoise1]{structure}

\usepackage[super]{nth}
\usepackage{amsmath}
% Путь к файлам с иллюстрациями
\graphicspath{{fig/}} % path to folder with Figures

% ----------------------------------------------------------------------------
% *** START BIBLIOGRAPHY <<<
% ----------------------------------------------------------------------------
\usepackage[
backend=biber, 
	style = phys,
	maxbibnames=99,
	citestyle=numeric,
	giveninits=true,
	isbn=true,
	url=true,
	natbib=true,
	sorting=ndymdt,
	bibencoding=utf8,
	useprefix=false,
	language=auto, 
	autolang=other,
	backref=true,
	backrefstyle=none,
	indexing=cite,]{biblatex}
\DeclareSortingTemplate{ndymdt}{
  \sort{
    \field{presort}
  }
  \sort[final]{
    \field{sortkey}
  }
  \sort{
    \field{sortname}
    \field{author}
    \field{editor}
    \field{translator}
    \field{sorttitle}
    \field{title}
  }
  \sort[direction=descending]{
    \field{sortyear}
    \field{year}
    \literal{9999}
  }
  \sort[direction=descending]{
    \field[padside=left,padwidth=2,padchar=0]{month}
    \literal{99}
  }
  \sort[direction=descending]{
    \field[padside=left,padwidth=2,padchar=0]{day}
    \literal{99}
  }
  \sort{
    \field{sorttitle}
  }
  \sort[direction=descending]{
    \field[padside=left,padwidth=4,padchar=0]{volume}
    \literal{9999}
  }
}

\addbibresource{Groningen.bib}%   \tiny \scriptsize \footnotesize \normalsize
\renewcommand*{\bibfont}{\scriptsize} % 

\setbeamertemplate{bibliography item}{\insertbiblabel}

% ----------------------------------------------------------------------------
% *** END BIBLIOGRAPHY <<<
% ----------------------------------------------------------------------------

% -------------------- FOOTNOTE *** START------------------------
% \title[Short Title]{Long Title}
\makeatletter
\setbeamertemplate{footline}{%
\leavevmode%
\hbox{\begin{beamercolorbox}[wd=.24 \paperwidth,ht=2.5ex,dp=3.0ex,leftskip=.01cm plus1fill,rightskip=.05cm]{author in head/foot}%
\usebeamerfont{title in head/foot}\insertshortauthor
    \end{beamercolorbox}%
    \begin{beamercolorbox}[wd=.76\paperwidth,ht=2.5ex,dp=1.125ex,leftskip=.05cm,rightskip=.15cm plus1fil]{title in head/foot}%
        \usebeamerfont{title in head/foot}\insertshorttitle{}
        \insertframenumber{} / \inserttotalframenumber \ \hspace*{2ex} 
    \end{beamercolorbox}}%
    \vskip-4pt%
}
\makeatother

% -------------------- FOOTNOTE *** END------------------------

% --------------------- TOC *** START -----------------------------------------
\setcounter{tocdepth}{3}
\setcounter{secnumdepth}{3}

\setbeamertemplate{section in toc}{%
  {\color{magenta!70!black}\inserttocsectionnumber.}~\inserttocsection}
\setbeamercolor{subsection in toc}{bg=white,fg=structure}
\setbeamertemplate{subsection in toc}{%
  \hspace{1.2em}{\color{Green1}\rule[0.3ex]{3pt}{3pt}}~\inserttocsubsection\par}
  
% --------------------- TOC *** END -----------------------------------------


\date{24-06-2013}
\title[\textcolor{white}{How could obligation chain be structured along cross-border gas supply ? Groningen, Netherlands, 24-06-2013}]{\textcolor{DodgerBlue4}{How could obligation chain be structured along cross-border gas supply for gas security ? \\3-tier legal interactions}}

\subtitle{\textcolor{Cyan4}{Presented at: \\
\large{\emph{Groningen Energy Summer School \\
'A multi-disciplinary approach to energy transition,\\ from policy to physics'}}
\normalsize{together with Huong Hoang. \\
Rijksuniversiteit Groningen\\
Groningen, Netherlands}}}

\author[\textcolor{Sienna3}{Polina Lemenkova}]{\large{Polina Lemenkova}}

% ----------------------------------------------------------------------------

\begin{document}
\begin{frame}
           \titlepage
\end{frame}

\section*{Table of Content}
\begin{frame}{Table of Content}
    \begin{columns}[onlytextwidth,T]
        \begin{column}{.5\textwidth}
            \tiny{\tableofcontents[sections=1-8]}
        \end{column}
        \begin{column}{.5\textwidth}
            \tiny{\tableofcontents[sections=9-14]}
        \end{column}
    \end{columns}
\end{frame}


\section{Summary}
\subsection{Research Aims}
\begin{frame}\frametitle{Research Aims}

\begin{alertblock}{Measure}
to measure components and linkages of legal obligations undertaken by the actors involving cross-border gas supply chain
\end{alertblock}

\begin{block}{Investigate}
to investigate possibility to establish a legal structure for promoting security of gas supply chain
\end{block}

\begin{block}{Examine}
to examine consequences of gas supply chain for government and companies
\end{block}

\begin{examples}{Analyze:}
legal structures (international-domestic-contract law): entitlement vs. state responsibility as requirements for functioning/enforcing obligation chain
\end{examples}

\end{frame}

\subsection{Research Problems}
\begin{frame}\frametitle{Research Problems}
\small{
\begin{alertblock}{Forecast}
Forecast: natural gas will increase demand at global average of nearly 2\% per year by 2035:
\begin{itemize}
	\item [$\Longrightarrow$] convenience in consumption
	\item [$\Longrightarrow$] economical and environmental efficiency
	\item [$\Longrightarrow$] natural gas remains a preferred choice in many applications with high consumption
\end{itemize}
\end{alertblock}

\begin{block}{Gas Reserve}
Gas reserve is highly unevenly distributed: 
\begin{itemize}
	\item [$\longrightarrow$] ca 0.75\% deposited: Middle East \& Eurasia; 
	\item [$\longrightarrow$]  54\% of total estimated: Russia, Iran \& Qatar
\end{itemize}
\end{block}

\begin{examples}{Large consumers:}
in Asia \& Europe are insufficient or lack of domestic gas to meet their demand
\end{examples}
}
\end{frame}

\subsection{Gas Market Situation: Pros \& Cons}
\begin{frame}\frametitle{Gas Market Situation: Pros \& Cons}
\begin{alertblock}{Supply-Demand}
Cross-border trade plays important role in 'supply-demand' relationships of unevenly distributed resources
\end{alertblock}

\begin{alertblock}{Long-Term Contracts}
Long-term contracts remain important for viable investment of a gas value chain
\end{alertblock}

\begin{block}{Supply Security}
Cross-border trade doesn't always guarantee the security of supply (adequate volume when and where it is needed at affordable price)
\end{block}

\begin{block}{Uncommitted Volume}
Long-term contracts limit the availability of uncommitted volume for market to balance demand-supply
\end{block}
\end{frame}

\section{Problem Statement}
\begin{frame}\frametitle{Problem Statement}

\begin{alertblock}{Scheme Drawbacks}
however, this scheme has serious drawbacks:
\begin{itemize}
	\item [$\rightarrow$] Current mechanism of O.G.S. has limitations in guaranteeing supply security for importing country.
	\item [$\rightarrow$] Gas market is constrained in delivering adequate volume at affordable price to the right place on time
	\item [$\rightarrow$] Impossibility of the importing government in directly controlling overseas gas supply source due to sovereignty of the producing country
	\item [$\rightarrow$] Ineffective managing/sustaining cross-border gas supply flow
\end{itemize}
\end{alertblock}
\end{frame}

\section{Hypothesis}
\begin{frame}\frametitle{Hypothesis}

\begin{alertblock}{Hypothesis}
\begin{itemize}
	\item [$\longleftrightarrow$] Targets of the test: dynamic dimensions of availability and affordability.
	\item [$\longleftrightarrow$] The scenarios of excess and shortage of gas supply (volume and infrastructure capacity) are applied for testing rights and obligations of the actors involved.
	\item [$\longleftrightarrow$] Producing/exporting country covers expensive supply and cheap supply (due to geological conditions and technology requirements).
	\item [$\longleftrightarrow$] 2 types of consuming/importing countries: 'rich consumer with little supply shortage; and 'poor consumer in high demand for gas' (levels of economic development \& social welfare).
	\item [$\longleftrightarrow$] Measure: gas volume, infrastructure capacity, distance, timing, price \& company’s flexibility.
\end{itemize}
\end{alertblock}

\end{frame}

\section{Research Methodology}
\begin{frame}\frametitle{Research Methodology}

\begin{alertblock}{Conceptual Analysis}
 to look into the elements and interactions of legal obligations in\\
  \alert{three tiers} of the relationship:
 \begin{itemize}
	\item [$\leftharpoondown$] government vs government in international law
	\item [$\leftharpoondown$] government vs company in domestic law
	\item [$\leftharpoondown$] company vs company in contract law
\end{itemize}
\end{alertblock}

\end{frame}

\section{Research Techniques}
\begin{frame}\frametitle{Research Techniques}
\small{
\begin{alertblock}{Assuming}
Performing the obligations implies aspect of actors’ expectations.
\end{alertblock}

\begin{block}{Examining}
Examining consequences for the government and the company.
\end{block}

\begin{alertblock}{Clarifying}
Clarifying the concepts
\end{alertblock}

\begin{block}{Investigating}
Investigating legal structure (international-domestic-contract law): entitlement \emph{vs} state responsibility as requirements for functioning/enforcing obligation chain
\end{block}

\begin{alertblock}{Exploring}
Exploring documentary database 
\end{alertblock}
}
\end{frame}

\section{Demand-Supply Imbalance: Triggers and Consequences}
\begin{frame}\frametitle{ Demand-Supply Imbalance: Triggers and Consequences}

\begin{alertblock}{Rigidity}
Rigidity of dedicated and network bound \alert{infrastructures}\\
$\Downarrow$\\
Limited \alert{swing capacity}\\
$\Downarrow$\\
The market responsiveness becomes \alert{inhibited (less flexible)}
\end{alertblock}

\begin{examples}{Imbalance:}
demand-supply relationships, e.g. by tight supply: 
\begin{itemize}
	\item [$\searrow$]market price are driven upward (when affordability aspect of supply security is ignored)
	\item [$\searrow$]market price cannot still attract demanded gas volume (because of relative concentration of the producers/suppliers)
\end{itemize}
\end{examples}
\end{frame}

\section{Government’s Role}
\subsection{Sustainable Gas Supply Flow}
\begin{frame}\frametitle{Government’s Role: Sustainable Gas Supply Flow}

\begin{alertblock}{Involvement}
Government’s role in gas sector for sustainable supply flow. The government's involvement helps to regulate uncertainties and difficulties that go beyond the guarantee of gas market:
\begin{itemize}
	\item [$\hookleftarrow$] Requirement of timely supply-demand balancing (e.g. in short notice):
		\begin{itemize}
			\item [$\longmapsto$] distance between production and consumption locations of the infrastructures.
			\item [$\longmapsto$] infrastructure capacity and constraints
		\end{itemize}
	\item [$\hookleftarrow$] Uncertainties related to volume availability
	\item[$\hookleftarrow$]  Affordable price
\end{itemize}
\end{alertblock}
\end{frame}

\subsection{Gas Sector: Domestic Affairs}
\begin{frame}\frametitle{Government's Role in Gas Sector: Domestic Affairs}

\begin{alertblock}{Domestic Affairs}
\begin{itemize}
	\item[$\mapsto$] formulating \& monitoring domestic framework with clarified objectives and responsibilities of the market players
	\item[$\mapsto$] licensing scheme \& legal framework under law rules 
	\item[$\mapsto$] controlling operations of the companies and local
market
	\item[$\mapsto$] facilitating gas inflow
	\item[$\mapsto$] influencing company’s operations through obligations or restrictions
	\item[$\mapsto$] stimulating companies to promote company’s presence and performance of particular business activities
\end{itemize}
\end{alertblock}
\end{frame}

\subsection{International Affairs}
\begin{frame}\frametitle{Government's Role in Gas Sector: International Affairs}

\begin{alertblock}{International Affairs}
\begin{itemize}
	\item [$\rightharpoonup$] aims at general strategies/policies of energy security securing supply of overseas gas
	\item [$\rightharpoonup$] indirect involvement of the governments: ‘many large gas contracts have been closed in the context of bilateral meetings between government leaders’
	\item [$\rightharpoonup$] creating multilateral agreements mitigating trade barriers,
	\item [$\rightharpoonup$] supporting cross-border activities among participating countries (e.g. stipulation of free transit principle in GATT/WTO
	\item [$\rightharpoonup$] mitigating trade barriers,
	\item [$\rightharpoonup$] agreement or in Energy Charter Treaty)
\end{itemize}
\end{alertblock}
 
\end{frame}

\subsection{Diplomatic Affairs}
\begin{frame}\frametitle{Government's Role in Gas Sector: Diplomatic Affairs}

\begin{alertblock}{Diplomatic Affairs}
\begin{itemize}
	\item [$\rightharpoondown$] interference in market \& cooperation relation with other countries within bilateral \& multilateral
agreements
	\item [$\rightharpoondown$] strengthening bilateral relationships 
	\item [$\rightharpoondown$] participating in multilateral agreements 
	\item [$\rightharpoondown$] engaging in diplomatic agreements
in cooperative spirit
	\item [$\rightharpoondown$] obtaining certainties of the supply source
	\item [$\rightharpoondown$] facilitating operations of the companies
\end{itemize}
\end{alertblock}

\end{frame}

\section{Advantages and Disadvantages}
\subsection{Possible Negative Consequences of Governmental Participation}
\begin{frame}\frametitle{Possible Disadvantages and Negative Consequences of Governmental Participation}

\begin{alertblock}{Dependance}
Cooperation agreement depend on diplomatic attitude and goodwill of the governments
\end{alertblock}

\begin{block}{Instabilities}
Cooperation agreements have no obligatory effect
\end{block}

\begin{block}{Uncertainties}
It may contain uncertainties
\end{block}

\begin{examples}{Unpredictability:}
It may be waived or failure in achieving expectations or practical yields
\end{examples}
\end{frame}


\subsection{Multilateral Agreements: Advantages \& Drawbacks}
\begin{frame}\frametitle{Multilateral Agreements: Advantages \& Drawbacks}

\begin{alertblock}{Energy Charter Treaty}
Multilateral agreements (M.A.), e.g. Energy Charter Treaty, provide general cooperation frameworks with more favorable conditions for participating members.
\end{alertblock}

\begin{block}{Entities or Trade}
However, M.A. could not enforce the entities or trade activities and may only be considered as ‘facilitation of trade’.
\end{block}

\begin{examples}{Scarcity or Shortage:}
Overseas gas supplies may be constrained to destine timely to the right location at affordable price with adequate volume, especially in case of scarcity or shortage.
\end{examples}

\end{frame}

\subsection{Government's Participation in Gas Sector: Drawbacks}
\begin{frame}\frametitle{Government's Participation in Gas Sector: Drawbacks}
\begin{alertblock}{Drawbacks}
The government’s capability of enforcing an independent company to perform any obligations at any time remains questionable:\\
$\Updownarrow$\\
company’s nature of for-profits aim needs for affordable gas supply for economic development and social welfare at macroeconomic level $\Leftrightarrow$\\
$\Updownarrow$\\
practice of freedom of contract in pursuing business
\end{alertblock}
\end{frame}

\section{Current mechanism of Overseas Gas Supply (O.G.S.)}
\begin{frame}\frametitle{Current mechanism of Overseas Gas Supply (O.G.S.)}

\begin{alertblock}{O.G.S.}
Current mechanism of Overseas Gas Supply consists in an operational frame with combination of:
\begin{itemize}
	\item [$\Rightarrow$] contractual obligations functioning in the market
	\item [$\Rightarrow$] government’s obligations originated from inter-government agreement
	\item [$\Rightarrow$] obligations imposed on gas undertakings by domestic regulation
\end{itemize}
\end{alertblock}

\end{frame}

\section{Limitations and Challenges}
\begin{frame}\frametitle{Limitations and Challenges}

\begin{examples}{Limitations and Challenges:}
Managing/sustaining cross-border gas supply flow has difficulties for the importing country. These include:
\begin{itemize}
	\item [$\rightleftharpoons$] lack of enforcing power over independent company’s operation
	\item [$\rightleftharpoons$] Impossibility in directly controlling overseas gas supply source, the explanation of which is attached to sovereignty of the producing country
	\item [$\rightleftharpoons$] no direct control of access to infrastructure capacity along gas chain
\end{itemize}
\end{examples}

\end{frame}

\section{Long-Term Goal}
\begin{frame}\frametitle{Long-Term Goal}

\begin{alertblock}{Responsibility}
The government is ultimately responsible for gas supply security of the country
Therefore, the main question is:
\end{alertblock}

$\downarrow$

\begin{examples}{Obligation Chain}
\begin{itemize}
	\item [$\Leftarrow$] satisfying both availability and affordability
dimensions in securing overseas gas flow, which is currently not easy for importing government
	\item [$\Leftarrow$] creating legal obligation chain along cross-border countries that guarantees availability and affordability of gas supply
\end{itemize}
\end{examples}
\end{frame}

\section{Thanks}
\begin{frame}{Thanks}
  	\centering \LARGE 
	\emph{Thank you for attention !}\\
	\vspace{5em}
\small
Acknowledgement: \\
Current work has been supported by the \\
Rijksuniversiteit Groningen Travel Grant (No. GESP 2013/355)\\
for author's 2-week participation at \\
Groningen Energy Summer School (June 2013),\\
Groningen, Netherlands.
\end{frame}

%%%%%%%%%%% Bibliography %%%%%%%
\section{Bibliography}
\Large{Bibliography}\\
\footnotesize{Author's publications on Geography, Environment, GIS and Landscape Studies:}
\vspace{1em}
\nocite{*}
\printbibliography[heading=none]

\end{document}

\end{document}

%Changing the font size locally (from biggest to smallest):	
%\Huge
%\huge
%\LARGE
%\Large
%\large
%\normalsize (default)
%\small
%\footnotesize
%\scriptsize
%\tiny

\end{document}