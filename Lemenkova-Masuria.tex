\documentclass[pdflatex,compress,8pt,
	xcolor={dvipsnames,dvipsnames,svgnames,x11names,table},
	hyperref={colorlinks = true,
	breaklinks = true, urlcolor = NavyBlue, breaklinks = true}]{beamer}	
\usetheme{Berlin}
\usecolortheme[named=Salmon2]{structure}

% ----------------------------------------------------------------------------
% *** START BIBLIOGRAPHY <<<
% ----------------------------------------------------------------------------
\usepackage[
	backend=biber, 
	style = numeric,
	maxbibnames=99,
	citestyle=numeric,
	giveninits=true,
	isbn=true,
	url=true,
	natbib=true,
	sorting=ndymdt,
	bibencoding=utf8,
	useprefix=false,
	language=auto, 
	autolang=other,
	backref=true,
	backrefstyle=none,
	indexing=cite,
]{biblatex}
\DeclareSortingTemplate{ndymdt}{
  \sort{
    \field{presort}
  }
  \sort[final]{
    \field{sortkey}
  }
  \sort{
    \field{sortname}
    \field{author}
    \field{editor}
    \field{translator}
    \field{sorttitle}
    \field{title}
  }
  \sort[direction=descending]{
    \field{sortyear}
    \field{year}
    \literal{9999}
  }
  \sort[direction=descending]{
    \field[padside=left,padwidth=2,padchar=0]{month}
    \literal{99}
  }
  \sort[direction=descending]{
    \field[padside=left,padwidth=2,padchar=0]{day}
    \literal{99}
  }
  \sort{
    \field{sorttitle}
  }
  \sort[direction=descending]{
    \field[padside=left,padwidth=4,padchar=0]{volume}
    \literal{9999}
  }
}

\addbibresource{Masuria.bib}%  \scriptsize \footnotesize
\renewcommand*{\bibfont}{\tiny} % 

\setbeamertemplate{bibliography item}{\insertbiblabel}

% ----------------------------------------------------------------------------
% *** END BIBLIOGRAPHY <<<
% ----------------------------------------------------------------------------

\usepackage[super]{nth}
\usepackage{amsmath}

%%%%%%%%%%%%%%%%%%%%%%%%%%%%

\title{The Environment of the Masuria Lakeland and\\ Biebrza National Park, North-Eastern Poland}
\subtitle{University of Warsaw}
\institute{Presentation at GEM MSc Course, Erasmus Mundus Scholarship}

\author{Polina Lemenkova}
         \date{June 25, 2010}

\begin{document}
\begin{frame}
           \titlepage
\end{frame}

\section*{Outline}
        \begin{frame}
           \tableofcontents
         \end{frame}

\section{Introduction}
\begin{frame}\frametitle{Introduction}
\begin{minipage}[0.4\textheight]{\textwidth}
\begin{columns}[T]
\begin{column}{0.5\textwidth}
\vspace{2em}
\begin{figure}[H]
	\centering
		\includegraphics[width=4.5cm]{F2.jpg}
\end{figure}
\footnotesize{Great Masurian Lakes (Wielkie Jeziora Mazurskie)}
\end{column}
\begin{column}{0.6\textwidth}
\vspace{2em} 
\begin{itemize}
	\item Masurian Lakeland and Biebrza National Park: wetland ecosystems of the North-Eastern Poland
	\item Uniqueness of the biodiversity if the area 
	\begin{itemize}
		\item richness of species diversity
		\item rare types of species
	\end{itemize}
	\item Beavers – example of rare species of the European environment
	\item Environmental problems: changes of land cover types (encroachment of reeds, willows and birch), habitat drain\end{itemize}
	\begin{figure}[H]
		\includegraphics[width=2.7cm]{F1.jpg}
	\end{figure}
	\footnotesize{Wilkasy, Gizhicko district}
\end{column}
\end{columns}
\end{minipage}
\end{frame}

\section{Lakeland} 
\begin{frame}\frametitle{Masuria Lakelands}
\begin{minipage}[0.4\textheight]{\textwidth}
\begin{columns}[T]
\begin{column}{0.5\textwidth}
\vspace{2em}
\begin{figure}[H]
	\centering
		\includegraphics[width=4.0cm]{F3.jpg}
\end{figure}
\footnotesize{Landscape near lake Niegocin}
\begin{figure}[H]
	\centering
		\includegraphics[width=4.0cm]{F4.jpg}
\end{figure}
Mixed forest
\end{column}
\begin{column}{0.5\textwidth}
\vspace{3ex} 
\begin{itemize}
	\item Masuria region: north-eastern Poland, has the \nth{2} World's density of lakes
	\item Masuria Lakeland – an area of over 2000 lakes, connected through a series of channels \& rivers,
	\item Gizhicko Gmina – study area within Masuria: Wilkasy, Gizhicko district
\end{itemize}
\vspace{1em}
\begin{figure}[H]
	\centering
		\includegraphics[width=4.0cm]{F5.jpg}
\end{figure}
\footnotesize{Pine forest}
\end{column}
\end{columns}
\end{minipage}
\end{frame}

\section{Landscapes}
\begin{frame}\frametitle{Landscapes of the Mazuria Lakes and surroundings}
\begin{minipage}[0.4\textheight]{\textwidth}
\begin{columns}[T]
\begin{column}{0.4\textwidth}
\vspace{2em}
\begin{figure}[H]
	\centering
		\includegraphics[width=4.0cm]{F7.jpg}
\end{figure}
\footnotesize{Wetlands in Masuria. Photo: author}
\end{column}
\begin{column}{0.6\textwidth}
\vspace{2em} 
Some large lakes:
\begin{itemize}
	\item Jezioro Sniardwy,
	\item Jezioro Niegocin,
	\item Reservat Mokre - Jezioro Dargin,
	\item Reservat Jezioro Dobskie - Jezioro Dobskie,
	\item Reservat Wyspy - Jezioro Mamry
\end{itemize}
\begin{figure}[H]
	\centering
		\includegraphics[width=3.0cm]{F6.jpg}
\end{figure}
Landscapes of the Mazuria are diverse: Pine forests, lakes, wetlands, mixed forests etc with various types of vegetation. Photo: author.
\end{column}
\end{columns}
\end{minipage}
\end{frame}

\section{Wetlands}
\begin{frame}\frametitle{Wetland landscapes in Mazuria}
\begin{minipage}[0.4\textheight]{\textwidth}
\begin{columns}[T]
\begin{column}{0.5\textwidth}
\vspace{2em}
\begin{figure}[H]
	\centering
		\includegraphics[width=5.0cm]{F8.jpg}
\end{figure}
\footnotesize{Landscapes in Masuria. Photo: author}
\end{column}
\begin{column}{0.5\textwidth}
\vspace{2em} 
Masurian wetlands: precious ecosystems
\begin{itemize}
            \item Wetlands – specific type of ecosystems
            \item Wetlands are typical for the Mazuria area in North-Eastern Poland
            \item Diversified landscape is a result of glacier activity Northern part of the region: moraine hills, hollows.
            \item Southern part – sand plains.
            \item The most valuable element – groups of lakes linked by a net of rivers
\end{itemize}
Masurian wetlands are rich in biodiversity. Some species:
\begin{itemize}
	\item mammals (fox, wolf etc)
	\item elk, bison, deer
	\item primeval forest species of trees, bushes and other plants
\end{itemize}
\end{column}
\end{columns}
\end{minipage}
\end{frame}

\section{Beavers}
\begin{frame}\frametitle{Beavers}
\begin{minipage}[0.4\textheight]{\textwidth}
\begin{columns}[T]
\begin{column}{0.3\textwidth}
\vspace{2em}
\begin{figure}[H]
	\centering
		\includegraphics[width=4.0cm]{F11.jpg}
\end{figure}
\footnotesize{European Beaver. Photo: author.}
\end{column}
\begin{column}{0.7\textwidth}
\begin{itemize}\small
            \item European Beaver (\emph{Castor fiber}) - one of the particular examples of the Masurian ecosystems. 
            \item Beavers have natural trait for building dams and canals for their dwelling (“lodges”). 
            \item Beavers are second-largest rodent in the world \\(after the capybara). 
            \item Beavers' colonies create one or more dams to provide still, deep water to protect against predators, and to float food and building material.
            \item Beavers can reach up to 30 kg weight.
            \item Masurian lakes play an important role in the restitution of the lowland European beaver population. 
            \item Beavers were hunted almost to extinction in Europe.
            \item Besides Poland there are Beaver populations in Germany, Finland, Russia, Czech Republic and Slovakia.
\end{itemize}
\end{column}
\end{columns}
\end{minipage}
\end{frame}

\section{Environment}
\begin{frame}\frametitle{Environmental Problems}
\begin{minipage}[0.4\textheight]{\textwidth}
\begin{columns}[T]
\begin{column}{0.5\textwidth}
\vspace{2em}
\begin{figure}[H]
	\centering
		\includegraphics[width=4.7cm]{F12.jpg}
\end{figure}
\footnotesize{European Beaver. Photo: author.}
\end{column}
\begin{column}{0.5\textwidth}
\vspace{2em} 
Some environmental problems in the region: 
\begin{itemize}
	\item encroachment of reeds, willows and birches
	\item endanger of wetland habitats by eutrophication and drying up
	\item (in past years) hunting 
	\item land use changing
\end{itemize}
\begin{figure}[H]
	\centering
		\includegraphics[width=3.0cm]{F13.jpg}
\end{figure}
\footnotesize{European Beaver. Photo: author.}
\end{column}
\end{columns}
\end{minipage}
\end{frame}

\section{BNP}
\begin{frame}\frametitle{Biebrza National Park (BNP) and Museum}
\begin{minipage}[0.4\textheight]{\textwidth}
\begin{columns}[T]
\begin{column}{0.4\textwidth}
\vspace{2em}
\begin{figure}[H]
	\centering
		\includegraphics[width=4.0cm]{F16.jpg}
\end{figure}
\end{column}
\begin{column}{0.6\textwidth}
\vspace{2em} 
\begin{itemize}
            \item BNP is named after 'Biebrza' river (the major river in the area, 145 km long);
            \item The name has linguistic origin from “beaver”. Founded as a National Park in 1993.
            \item Located in the river basin of Narew, the Biebrza and their tributaries
            \item BNP - the  biggest area of swamps and wetlands in Europe;
            \item BNP has most diverse flora and fauna comparing to similar wetland ecosystems 
            \item BNP is the largest National Park in Poland
\end{itemize}
\end{column}
\end{columns}
\end{minipage}
\end{frame}

\section{Museum}
\begin{frame}\frametitle{in BNP Museum...}
\begin{minipage}[0.4\textheight]{\textwidth}
\begin{columns}[T]
\begin{column}{0.5\textwidth}
\begin{figure}[H]
	\centering
		\includegraphics[width=3.7cm]{F14.jpg}
\end{figure}
\begin{figure}[H]
	\centering
		\includegraphics[width=3.7cm]{F15.jpg}
\end{figure}
\footnotesize{In the BNP Museum. Photo: author}
\end{column}
\begin{column}{0.5\textwidth}
\vspace{2em} 
Typical parts and elements of Biebrza landscapes:
\begin{itemize}
	\item Swamps, peat bog, mixed forests
	\item Trees: birches, olsa
	\item Animals: elks, wolves and beavers
	\item Birds: stork, black-tailed godwit, sandpipers
\end{itemize}
Methods of the environmental protection:
\begin{itemize}
	\item understanding environmental relationships and structure of the ecosystems
	\item restricted or prohibited human activities in the area
	\item cultivation and restitution of the population of rare species \\(beavers, bisons, etc)
\end{itemize}
\end{column}
\end{columns}
\end{minipage}
\end{frame}

\begin{frame}\frametitle{Attractiveness}
\begin{minipage}[0.4\textheight]{\textwidth}
\begin{columns}[T]
\begin{column}{0.5\textwidth}
\vspace{2em}
\begin{figure}[H]
	\centering
		\includegraphics[width=3.0cm]{F9.jpg}
\end{figure}
\begin{figure}[H]
	\centering
		\includegraphics[width=3.0cm]{F10.jpg}
\end{figure}
\footnotesize{In the BNP Museum. Photo: author}
\end{column}
\begin{column}{0.5\textwidth}
\vspace{2em} 
Main factors of Biebrza attractiveness \& uniqueness:
\begin{itemize}
	\item ecological value, nature
	\item picturesque value of landscapes
	\item variety of flora and fauna
\end{itemize}
\begin{figure}[H]
	\centering
		\includegraphics[width=3.7cm]{F17.jpg}
\end{figure}
\footnotesize{European stork. Photo: author}
\end{column}
\end{columns}
\end{minipage}
\end{frame}

\section{Nature}
\begin{frame}\frametitle{Nature of Biebrza National Park}
Biebrza National Park gives habitat for many rare fauna species near extinction due to the reasons: 
\begin{itemize}
	\item Climatic and ecological characteristics
	\item Water environment with high level of moisture, necessary for some species, birds
	\item Vast area (59,233 ha)
	\item Few human activities in the area of wetlands 
\end{itemize}
Just some examples of species:
\begin{itemize}
	\item Birds - Spotted Eagle, Black Grouse, Great Snipe, 
	\item Animals – Beaver, Otter, Wolf
	\item Plants – Fen Orchid, Lady’s Slipper
\end{itemize}
\end{frame}

\section{Statistics}
\begin{frame}\frametitle{Biebrza: Statistics}
\begin{minipage}[0.4\textheight]{\textwidth}
\begin{columns}[T]
\begin{column}{0.4\textwidth}
\vspace{2em}
\begin{figure}[H]
	\centering
		\includegraphics[width=4.0cm]{F18.jpg}
\end{figure}
\begin{figure}[H]
	\centering
		\includegraphics[width=4.0cm]{F19.jpg}
\end{figure}
\end{column}
\begin{column}{0.6\textwidth}
\vspace{2em} 
Biebrza National Park gives habitat to
\begin{itemize}
	\item 280 bird species: it is the place of nesting and living for lots of water birds which only exist in such areas
	\item 178 among them breed in the Biebrza wetlands
	\item 48 mammals
	\item The biggest population of Elks (ca 600 individuals)
	\item more tan 1000 different types of vascular plants species (including rare orchids Lady’s Slipper)
	\item BNP encompasses the largest area of marshes, peat bogs and grassland in Europe, mainly of primordial origin
\end{itemize}
\end{column}
\end{columns}
\end{minipage}
\end{frame}

\section{Conclusion}
\begin{frame}\frametitle{Conclusion: Environmental Protection}
\begin{itemize}
	\item From 1995 BNP is included in the list of RAMSAR regions – i.e. it is an ecological area of international importance (wetland).
	\item The BNP is included into the European Ecological NATURA 2000 network \& protecting system. 
	\item For many of bird, mammals and vegetation species BNP is one of the last places for living that still exist in Europe
	\item BNP includes Red Bog project – one of the oldest and most protected area in Poland, within BNP. 
	\item The Red Bog is the \nth{2} biggest raised  bog in Poland
	\item The Masuria Lakeland and BNP are the areas especially rich in species and biodiversity
	\item The areas of Masuria Lakeland and BNP must be strictly protected because of their unique characteristics and importance for the European environment
	\item There are some environmental problems in the area: encroachment of reeds, willows, birches, peatbog fires
	\item There are already some environmental activities in the area and protecting projects funded by the EU
\end{itemize}
\end{frame}

\section{Thanks}
\begin{frame}{Thanks}
  	\centering \LARGE 
  	\emph{Thank you for attention !}\\
\end{frame}

%%%%%%%%%%% Bibliography %%%%%%%

\section{Bibliography}
\begin{frame}{Bibliography}
\scriptsize{Author's publications focused on environment: \cite{Lemenkova2006e}, \cite{Lemenkova2006b}, \cite{Lemenkova2006a}, \cite{Lemenkova2007b}, \cite{Lemenkova2004a}, \cite{Lemenkova2008b},  \cite{Lemenkova2005b1}, \cite{Lemenkova2005a}.}

\nocite{*}
\printbibliography[heading=none]

\end{frame}
%%%%%%%%%%% Bibliography %%%%%%%	

%Changing the font size locally (from biggest to smallest):	
%\Huge
%\huge
%\LARGE
%\Large
%\large
%\normalsize (default)
%\small
%\footnotesize
%\scriptsize
%\tiny


\end{document}