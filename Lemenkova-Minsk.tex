\documentclass[pdflatex,compress,9pt,
	xcolor={dvipsnames,dvipsnames,svgnames,x11names,table},
	hyperref={colorlinks = true,breaklinks = true, urlcolor = NavyBlue, breaklinks = true}]{beamer}
	
\usepackage[utf8]{inputenc}
\usepackage[T2A,T1]{fontenc}
%\usefonttheme{structuresmallcapsserif}
\usepackage{bookman}

\usetheme{Bergen}
%\usecolortheme{seahorse}Teal DarkCyan
%\usecolortheme{wolverine} 
\usecolortheme[named=Crimson]{structure}

%%%%%%%%%%%%%%%%%%%%%%%%%%%%%%%%%%%

\usepackage{gensymb} % degree symbol
\usepackage[super]{nth}
%%%%%%%%%%%%%%%%%%%%%%%%%%%%%%%%%%%
% ----------------------------------------------------------------------------
% *** START BIBLIOGRAPHY <<<
% ----------------------------------------------------------------------------
\usepackage[
	backend=biber, 
%	style = numeric,
	style=apa,
	maxbibnames=99,
%	citestyle=authoryear,
	citestyle=numeric,
	giveninits=true,
	isbn=true,
	url=true,
	natbib=true,
	sorting=ndymdt,
	bibencoding=utf8,
	useprefix=false,
	language=auto, 
	autolang=other,
	backref=true,
	backrefstyle=none,
	indexing=cite,
]{biblatex}
\DeclareSortingTemplate{ndymdt}{
  \sort{
    \field{presort}
  }
  \sort[final]{
    \field{sortkey}
  }
  \sort{
    \field{sortname}
    \field{author}
    \field{editor}
    \field{translator}
    \field{sorttitle}
    \field{title}
  }
  \sort[direction=descending]{
    \field{sortyear}
    \field{year}
    \literal{9999}
  }
  \sort[direction=descending]{
    \field[padside=left,padwidth=2,padchar=0]{month}
    \literal{99}
  }
  \sort[direction=descending]{
    \field[padside=left,padwidth=2,padchar=0]{day}
    \literal{99}
  }
  \sort{
    \field{sorttitle}
  }
  \sort[direction=descending]{
    \field[padside=left,padwidth=4,padchar=0]{volume}
    \literal{9999}
  }
}

\addbibresource{Minsk.bib}
\renewcommand*{\bibfont}{\tiny} % \tiny \scriptsize \footnotesize

\setbeamertemplate{bibliography item}{\insertbiblabel}

% ----------------------------------------------------------------------------
% *** END BIBLIOGRAPHY <<<
% ----------------------------------------------------------------------------
% Путь к файлам с иллюстрациями
\graphicspath{{fig/}} % path to folder with Figures

\title{Cost-Effective Raster Image Processing\\
 for Geoecological Analysis \\
 Using “ISOCLUST” Classifier:\\
 a Case Study of Estonian Landscapes}\vspace{2em}
 
\subtitle{\vspace*{0.5cm}Presented at \nth{5} International Conference \\
'Modern Problems of Geoecology and Landscapes Studies'\\
Belarus State University (BSU)\\
Minsk, Belarus}

\author{Polina Lemenkova}
\date{October 14, 2014}

\begin{document}
\begin{frame}
           \titlepage
\end{frame}

\section*{Outline}
\begin{frame}
  \small{         \tableofcontents}
\end{frame}

\section{Introduction}

\subsection{Study Area}
\begin{frame}{Study Area: P\"{a}rnu Region, Estonia}
\begin{minipage}[0.4\textheight]{\textwidth}
\begin{columns}[T]
\begin{column}{0.5\textwidth}
\begin{figure}[H]
	\centering
		\includegraphics[width=4.2cm]{F1.jpg}
\end{figure}
\begin{block}{Study Area}
The research region encompasses coastal area of Baltic Sea: south-western Estonia
\end{block}
 
\begin{alertblock}{Spatial extent}
Spatial extent of the study area is limited to the surroundings of P\"{a}rnu County.
\end{alertblock}

\end{column}
\begin{column}{0.5\textwidth}
\begin{figure}[H]
	\centering
		\includegraphics[width=4.5cm]{F2.jpg}
\end{figure}
\end{column}
\end{columns}
\end{minipage}
\end{frame}

\subsection{Research Aim}
\begin{frame}\frametitle{Research Aim}
The purpose of this study is following \alert{two aims}:
\begin{alertblock}{GIS Analysis}
first, a geographic (GIS) analysis of land cover types in the coastal landscapes of western Estonia, P\"{a}rnu surroundings at two various temporal dates (\alert{1992} and \alert{2006})
\end{alertblock}

\begin{block}{IDRISI GIS}
second, an overview of the technical methods enabling image processing by different tools of IDRISI GIS software"
\end{block}

\begin{block}{Landsat TM Images}
Hence, the main research methods consists in processing and classification of satellite remote sensing data Landsat TM aimed at land cover types recognition and thematic mapping.
\end{block}

\end{frame}

\subsection{Landscapes}
\begin{frame}\frametitle{Landscapes}

\begin{block}{South-West Estonia: Unique Environment}
South-west Estonia is known for unique environmental settings: mild maritime climate, broad beaches, coniferous pine forests on the coastal zone
\end{block}

\begin{alertblock}{Landscapes}
Landscapes here are rich and world-known for their diversity, variability, unique composition structure and high esthetic value
\end{alertblock}

\begin{examples}{Types of Landscapes}
Landscape types include, for example, mixed and broadleaved forests, traditional agricultural semi-natural landscapes, wooded meadows, plant communities, heathland, bogs and moors, complex anthropogenic areas with different land use structure, shrubland, grasslands, birch-dominating coastal areas and flooded meadows
\end{examples}

\end{frame}

\section{Examples of Landscapes}
\subsection{Baltic Sea Coasts}
\begin{frame}\frametitle{Examples of Landscapes: Baltic Sea Coasts}
\begin{examples}{Landscapes:}
Baltic Sea Coasts. Photos: author.
\end{examples}
\begin{figure}[H]
	\centering
		\includegraphics[width=9.0cm]{F3.jpg}
\end{figure}
\end{frame}

\subsection{Marine Landscapes: P\"{a}rnu}
\begin{frame}\frametitle{Marine Landscapes: P\"{a}rnu Coasts}
\begin{examples}{Marine Landscapes:}
 P\"{a}rnu Coasts. Photos: author.
\end{examples}
\begin{figure}[H]
	\centering
		\includegraphics[width=9.0cm]{F4.jpg}
\end{figure}
\end{frame}

\subsection{Lacustrine landscapes: Luitemaaa}
\begin{frame}\frametitle{Lacustrine landscapes: Luitemaa Nature Conservation Area}
\begin{examples}{Lacustrine landscapes:}
Luitemaa Nature Conservation Area. Photos: author.
\end{examples}
\begin{figure}[H]
	\centering
		\includegraphics[width=9.0cm]{F5.jpg}
\end{figure}
\end{frame}

\subsection{Forest landscapes: Luitemaa}
\begin{frame}\frametitle{Forest landscapes: Luitemaa Nature Conservation Area}
\begin{examples}{Forest Landscapes:}
Luitemaa Nature Conservation Area. Photos: author.
\end{examples}
\begin{figure}[H]
	\centering
		\includegraphics[width=9.0cm]{F6.jpg}
\end{figure}
\end{frame}

\section{Anthropogenic Impacts}
\subsection{Facts}
\begin{frame}\frametitle{Anthropogenic Impacts}

\begin{block}{Tourism Activities}
P\"{a}rnu region is traditionally popular as a \alert{tourism destination} due to favorable combination of factors:
\begin{itemize}
	\item geographic value: advantageous location on the coasts of Baltic Sea
	\item social value: good facilities for the tourism and tourism reputation
	\item environmental value: unique nature (marine landscapes, pine forests)
\end{itemize}
\end{block}

\begin{block}{Agricultural Activities}
P\"{a}rnu region is also known for traditional \alert{agricultural activities} (field crops cultivation, intensive planting, etc), as well as extensive \alert{housing development} in the rural area.
\end{block}

\begin{alertblock}{Human Pressure}
All these factors create additional human pressure on the local ecosystems and may lead to fragmentation of the landscape structure.
\end{alertblock}
\end{frame}

%%%%%%%%%%
\subsection{Ecohouses Construction}
\begin{frame}\frametitle{Ecohouses Construction}
\begin{examples}{Ecohouses Construction}
Photos: author.
\end{examples}
\begin{figure}[H]
	\centering
		\includegraphics[width=9.0cm]{F7.jpg}
\end{figure}
\end{frame}

\begin{frame}\frametitle{Examples: Eco-housing}
\begin{examples}{Eco-housing}
Photos: author.
\end{examples}
\begin{figure}[H]
	\centering
		\includegraphics[width=9.0cm]{F12.jpg}
\end{figure}
\end{frame}

\section{Data}
\begin{frame}\frametitle{Data}
The research data used in this project include vector and raster types of data:

\begin{block}{Raster data:}
Thematic raster layers (GeoTIFF) Landsat TM including scenes taken on \alert{18 June 2006} and \alert{03 June 1992}. \\
Both images cover summer months, thus enabling vegetation coverage to be easily recognized. \\
The images were downloaded from the Earth Science Data Interface, Global Land Cover Facility.
\end{block}

\begin{block}{Vector data:}
\alert{CORINE} vector layers (abbreviation from \alert{Coordination of Information on the Environment}), developed by the European Environmental Agency, EU Commission). \\
The CORINE data were stored in ESRI format shape-files. They contain information on land use types provided by the Estonian Land Board and available at the University of Tartu
\end{block}
\end{frame}

\section{Methods}
\begin{frame}\frametitle{Methods}

\begin{block}{GIS Projects}
The GIS projects has been organized and executed in two different software: \alert{Arc GIS 10.0} and \alert{IDRISI GIS Andes 15.0}.
\end{block}

\begin{block}{Landsat TM}
The raster processing GIS approach and classification was applied in the current work towards Landsat TM two images."
\end{block}

\begin{block}{ISOCLUST}
The method is based on the ISOCLUST unsupervised classification executed by means of IDRISI GIS.
\end{block}

\begin{block}{Machine Learning}
The ISOCLUST available in IDRISI GIS performs the most of the image processing workflow in semi-automatically regime
\end{block}

\begin{block}{Land Cover Classes}
It results in a map with pre-defined number of 16 land class categories which enable to compare two different stages of landscape development: \alert{“earlier”} and \alert{“now”}.
\end{block}

\begin{block}{Objectivity}
The ISOCLUST method was chosen, since it enables to avoid subjectivity in classification.
\end{block}

\end{frame}

\begin{frame}\frametitle{Workflow Step-1}

\begin{block}{IMPORT}
Initially, the data of Landsat TM were imported to IDRISI Andes GIS from \alert{GeoTIFF} format to IDRISI specific format \alert{.rst}, through \alert{Data Provider Format} import.\\ As each Landsat TM scene is a multispectral image with several spectral bands, each band was displayed and visualized as a separate image.
\end{block}

\begin{block}{COLOR COMPOSITE}
Afterwords, the images were composed using \alert{Color Composite} function. \\
The combination of three bands was made as a single color composite image (bands 2-3-4).\\
 This composition displays urban areas distinctively, which enables to clearly recognize them
\end{block}

\begin{block}{PROJECT}
Then data were organized in a created project in \alert{IDRISI GIS}.
\end{block}

\end{frame}

\begin{frame}\frametitle{Workflow Step-1}
\begin{examples}{Image Processing}
\end{examples}
\begin{figure}[H]
	\centering
		\includegraphics[width=9.0cm]{F8.jpg}
\end{figure}
\end{frame}

\begin{frame}\frametitle{Workflow Step-2}

\begin{block}{CLASSIFICATION}
The next step includes application of chosen classification method of ISOCLUST approach towards images processing. ISOCLUST classifier technique is based on the histogram peak selection technique"
\end{block}

\begin{block}{Analysis of Spectral Signatures}
The ground principle of the classification consists in the analysis of spectral signatures that are individual for each land cover class.
\end{block}

\begin{block}{Analysis of Spectral Reflections}
The analysis of spectral reflections strongly depends on the local surface features: texture, structure, color, etc.
\end{block}

\begin{block}{Spectral Signatures}
Information on spectral signatures is received by the satellite sensors and recorded on the images (in this case, Landsat TM). This information is used for the image classification.
\end{block}

\begin{examples}{Information Extraction:}
Using individual characteristics of objects, derived from the multispectral Landsat TM bands, information from the images was extracted, analyzed and used for land classification
\end{examples}

\end{frame}

\begin{frame}\frametitle{Workflow Step-3}
\begin{figure}[H]
	\centering
		\includegraphics[width=6.0cm]{F9.jpg}
\end{figure}
\begin{block}{Image Comparison}
Changes in land cover types in selected Estonian landscapes are shown on the histograms on 1992 and 2006.
\end{block}

\begin{block}{2006 vs 1992}
 In 2006 the urban area became larger than in 1992 (land cover class "3" on the histogram. This can be explained by various reasons.
\end{block}

\begin{block}{Impact Factors}
The most important reason is intense suburbanization $=>$ the major process in the current urban dynamics of modern Estonia: intensive construction of summer homes and cottages in the coastal area.
\end{block}

\begin{block}{Buildings}
New buildings and houses created along the P\"{a}rnu Bay, $=>$ increased area of urban areas.
\end{block}
\end{frame}

\section{Results}
\begin{frame}\frametitle{Image Processing}

\begin{block}{ISOCLUST}
ISOCLUST classification of the images enabled to create thematic maps of the same study areas
\end{block}

\begin{block}{CORINE}
According to CORINE, there are 16 land cover types typical for the study area.
\end{block}

\begin{figure}[H]
	\centering
		\includegraphics[width=9.0cm]{F10.jpg}
\end{figure}
\end{frame}

\begin{frame}\frametitle{Land Cover Classes}
\begin{minipage}[0.4\textheight]{\textwidth}
\begin{columns}[T]
\begin{column}{0.5\textwidth}
\small{\begin{itemize}
	\item discontinuous urban fabric
	\item industrial or commercial units
	\item green urban areas
	\item pastures
	\item complex cultivation patterns
	\item agriculture lands with grass
	\item broad-leaved forest
	\item coniferous forest
	\item mixed forest
	\item natural grassland
	\item moors and heathlands
	\item transitional woodland
	\item beaches, dunes, sand
	\item island marshes
	\item water bodies
	\item sea and ocean
\end{itemize}}
\end{column}
\begin{column}{0.5\textwidth}
\begin{figure}[H]
	\centering
		\includegraphics[width=4.5cm]{F11.jpg}
\end{figure}
\end{column}
\end{columns}
\end{minipage}
\end{frame}

\section{Thanks}
\begin{frame}{Thanks}
\emph{Thank you for attention !}\\
\begin{figure}[H]
	\centering
		\includegraphics[width=8.0cm]{F13.jpg}
\end{figure}
\small{Photo: author.}\\
\normalsize
Acknowledgement: \\
The research has been done at the University of Tartu \\
under support of DoRa grant (ESF, Estonia).\\
The University of Tartu provided \alert{data} and \alert{software}:\\
 \alert{CORINE} vector layers, \alert{IDRISI GIS} 15.0 and \alert{ArcGIS} 10.0.
\end{frame}

%%%%%%%%%%% Bibliography %%%%%%%
\section{Bibliography}
\Large{Bibliography}
\nocite{*}
\printbibliography[heading=none]
	
%%%%%%%%%%% Bibliography %%%%%%%	

\end{document}
%Changing the font size locally (from biggest to smallest):	
%\Huge
%\huge
%\LARGE
%\Large
%\large
%\normalsize (default)
%\small
%\footnotesize
%\scriptsize
%\tiny

\end{document}