%\documentclass{beamer}
\documentclass[pdflatex,compress,8pt,
	xcolor={dvipsnames,dvipsnames,svgnames,x11names,table},
	hyperref={	 
	pdfauthor={Lemenkova Polina}, 
	pdfsubject={Preentation}, 
	pdfcreator={Lemenkova Polina}, 
	pdfproducer={Lemenkova Polina}, 
	colorlinks=true,
	linkcolor=Red3, 
	citecolor=NavyBlue, 
%	urlbordercolor=cyan,
	urlcolor = NavyBlue, 
	breaklinks = true}]{beamer}
\usetheme{Warsaw}
\usecolortheme{rose} %lily %orchid
\usecolortheme{seahorse}%whale %  seahorse % dolphin
\usecolortheme[named=SpringGreen1]{structure}% 

%\usepackage[T2A,T1]{fontenc}
\usepackage[T2A]{fontenc}
\usepackage[utf8]{inputenc}
\usepackage{fontspec}
\usepackage[english,russian]{babel}
\usepackage{hyphenat} %переносы
\usepackage{ragged2e}

%%%%%%%%%%%%% рус: начало %%%%%%%%%%%%
\usepackage{polyglossia} % Поддержка многоязычности (fontspec подгружается автоматически)
%\usepackage{fontspec}
\setmainlanguage{english} % Язык по-умолчанию английский  (в американской вариации по-умолчанию)
\setotherlanguage{russian} % Дополнительный язык = русский
%\setmonofont{CMU Typewriter Text} % моноширинный шрифт
\setmonofont{Times New Roman}
\newfontfamily\cyrillicfonttt{CMU Typewriter Text} % моноширинный шрифт для кириллицы
\defaultfontfeatures{Ligatures=TeX} % стандартные лигатуры TeX, замены нескольких дефисов на тире и т. п. Настройки моноширинного шрифта должны идти до этой строки, чтобы при врезках кода программ в коде не применялись лигатуры и замены дефисов
\setmainfont{CMU Serif} % Шрифт с засечками
\newfontfamily\cyrillicfont{CMU Serif} % Шрифт с засечками для кириллицы
\setsansfont{CMU Sans Serif} % Шрифт без засечек
\newfontfamily\cyrillicfontsf{CMU Sans Serif}

%%%%%%%%%%%%% рус: конец %%%%%%%%%%%%

% ----------------------------------------------------------------------------
% *** START BIBLIOGRAPHY <<<
% ----------------------------------------------------------------------------
\usepackage[
	backend=biber, 
	style = phys,
	maxbibnames=99,
	citestyle=numeric,
	giveninits=true,
	isbn=true,
	url=true,
	natbib=true,
	sorting=ndymdt,
	bibencoding=utf8,
	useprefix=false,
	language=auto, 
	autolang=other,
	backref=true,
	backrefstyle=none,
	indexing=cite,
]{biblatex}
\DeclareSortingTemplate{ndymdt}{
  \sort{
    \field{presort}
  }
  \sort[final]{
    \field{sortkey}
  }
  \sort{
    \field{sortname}
    \field{author}
    \field{editor}
    \field{translator}
    \field{sorttitle}
    \field{title}
  }
  \sort[direction=descending]{
    \field{sortyear}
    \field{year}
    \literal{9999}
  }
  \sort[direction=descending]{
    \field[padside=left,padwidth=2,padchar=0]{month}
    \literal{99}
  }
  \sort[direction=descending]{
    \field[padside=left,padwidth=2,padchar=0]{day}
    \literal{99}
  }
  \sort{
    \field{sorttitle}
  }
  \sort[direction=descending]{
    \field[padside=left,padwidth=4,padchar=0]{volume}
    \literal{9999}
  }
}

\addbibresource{Moscow.bib}%  \scriptsize \footnotesize
\renewcommand*{\bibfont}{\tiny} % 

\setbeamertemplate{bibliography item}{\insertbiblabel}

% Путь к файлам с иллюстрациями
\graphicspath{{fig/}} % path to folder with Figures

\usepackage{gensymb} % degree symbol
\usepackage[super]{nth}
\usepackage{amsmath}
\usepackage{subfig}
\usepackage{multicol}
\usepackage[T1]{fontenc}
\usepackage[utf8]{inputenc}
\usepackage{palatino}
\usepackage{multicol} % to split itemization

%%%%%%%%%%%%%%%%%%%%%%%%%%%%

% ----------------------------------------------------------------------------
% *** END BIBLIOGRAPHY <<<
% ----------------------------------------------------------------------------

% -------------------- FOOTNOTE *** START------------------------
% \title[Short Title]{Long Title}
\makeatletter
\setbeamertemplate{footline}{%
\leavevmode%
\hbox{\begin{beamercolorbox}[wd=.24 \paperwidth,ht=2.5ex,dp=3.0ex,leftskip=.01cm plus1fill,rightskip=.05cm]{author in head/foot}%
\usebeamerfont{title in head/foot}\insertshortauthor
    \end{beamercolorbox}%
    \begin{beamercolorbox}[wd=.76\paperwidth,ht=2.5ex,dp=1.125ex,leftskip=.05cm,rightskip=.15cm plus1fil]{title in head/foot}%
        \usebeamerfont{title in head/foot}\insertshorttitle{}
        \insertframenumber{} / \inserttotalframenumber \ \hspace*{2ex} 
    \end{beamercolorbox}}%
    \vskip-4pt%
}
\makeatother

% -------------------- FOOTNOTE *** END------------------------

% --------------------- TOC *** START -----------------------------------------
\setcounter{tocdepth}{3}
\setcounter{secnumdepth}{3}

\setbeamertemplate{section in toc}{%
  {\color{magenta!70!black}\inserttocsectionnumber.}~\inserttocsection}
\setbeamercolor{subsection in toc}{bg=white,fg=structure}
\setbeamertemplate{subsection in toc}{%
  \hspace{1.2em}{\color{Green1}\rule[0.3ex]{3pt}{3pt}}~\inserttocsubsection\par}
  
% --------------------- TOC *** END -----------------------------------------

\title[Using ArcGIS in Teaching Geosciences. Lomonosov Moscow State University, Moscow, Russia, 2007-06-05.]{Использование геоинформационных технологий в преподавании дисциплин в Высшей школе}

\subtitle{Аттестационная работа \\
на получение дополнительного к Высшему образованию\\
Московский Государственный Университет им. М. В. Ломоносова\\
Факультет глобальных процессов\\
Отделение педагогического образования\\
Специализация 'Преподаватель Высшей школы'
}
\author[Polina Lemenkova]{Полина Алексеевна Леменкова}

\date{2007-06-05}

\begin{document}
\begin{frame}
           \titlepage
\end{frame}

\section*{Table of Contents}
\begin{frame}{Table of Contents}
    \begin{columns}[onlytextwidth,T]
        \begin{column}{.30\textwidth}
            \footnotesize{\tableofcontents[sections=1-5]}
        \end{column}
        \begin{column}{.30\textwidth}
            \footnotesize{\tableofcontents[sections=6-6]}
        \end{column}
        \begin{column}{.30\textwidth}
            \footnotesize{\tableofcontents[sections=7-11]}
        \end{column}
    \end{columns}
\end{frame}


\section{Introduction}
\begin{frame}\frametitle{Introduction}

\small{
\begin{block}{}
Выявление путей повышения эффективности преподавания географических дисциплин в ВУЗах благодаря использованию ГИС и геоинформационных технологий.
\end{block}

\begin{alertblock}{}
\begin{itemize}
	\item Проанализировать психолого-педагогическую и методическую литературу, теоретические вопросы преподавания географических дисциплин в ВУЗе.
	\item Определить структуру и функции ГИС-программ, изучить один из ГИС для дальнейшего преподавания (выыбран Arc GIS 9.1).
	\item Изучить практические возможности использования ГИС в учебном процессе преподавания географии в ВУЗах в российской и международной практике
	\item Исследовать эффективность использования геоинформационных технологий использования и ГИС-программ в образовательном процессе Высшей школы.
	\item Разработать схему занятий (поурочный план) со студентами по изучению программы ArcGIS 9.1
\end{itemize}
\end{alertblock}

\begin{alertblock}{}
В настоящее время необходимы разработки новых методик преподавания на географическом факультете с опорой на ГИС-технологии как информационно-технический базис изучаемых дисциплин.
\end{alertblock}

\begin{block}{}
Содержание и методика ГИС-преподавания на лабораторных занятиях и семинарах 
\end{block}
}
\end{frame}

\section{GIS: Background}
\begin{frame}\frametitle{GIS: Background}
ГИС – сокращ. Географическая Информационная Система \\
(от англ. 'Geographic Information System', GIS)  –  это интегрированные в единой информационной среде компоненты:
\begin{itemize}
	\item базы данных
	\item атрибутивная информация
	\item пространственная информация
	\item электронные пространственные изображения (карты, схемы, планы) 
	\item доп. информация (рисунки, иллюстрации, фотографии, тексты, интернет-ссылки)
\end{itemize}

\begin{minipage}[0.4\textheight]{\textwidth}
		\begin{columns}[T]
			\begin{column}{0.5\textwidth}
%				\vspace{4em}
				\begin{figure}[H]
					\centering
					\includegraphics[width=5.0cm]{F1.jpg}
				\end{figure}
			\end{column}
			\begin{column}{0.5\textwidth}
\vspace{2em}
Схема состава ГИС:
\begin{itemize}
	\item компьютеры + периферийное оборудование
	\item программное обеспечение ГИС
	\item данные (карты, базы данных и т.д.)
	\item методы обработки данных
	\item люди (пользователи ГИС)
\end{itemize}
			\end{column}
		\end{columns}
\end{minipage}

\end{frame}

\section{GIS}
\subsection{GIS: Advantages}
\begin{frame}\frametitle{GIS: Advantages}
Отличие ГИС от традиционной картографии:

\begin{minipage}[0.4\textheight]{\textwidth}
		\begin{columns}[T]
			\begin{column}{0.5\textwidth}
				\begin{figure}[H]
					\centering
					\includegraphics[width=5.0cm]{F2.jpg}
				\end{figure}
			\end{column}
			\begin{column}{0.5\textwidth}
\begin{alertblock}{}
ГИС  представляет собой не просто  набор нарисованных карт, а  информационную систему, 
обеспечивающую:
\begin{itemize}
	\item сбор, хранение, обработку, отображение данных (графическая визуализация)
	\item анализ пространственных данных (геокодирование, моделирование)
\end{itemize}
\end{alertblock}
			\end{column}
		\end{columns}
	\end{minipage}

\begin{block}{}
ГИС - современная технология картографирования и анализа  объектов и событий реального мира.
\end{block}

\begin{alertblock}{ГИС}
ГИС используется для: картопроизводства с нуля редактирования имеющихся карт оцифровки старых бумажных карт создания картографических баз данных
\end{alertblock}

\end{frame}

\subsection{Arc GIS: Specifics}
\begin{frame}\frametitle{Arc GIS: Specifics}
	\begin{minipage}[0.4\textheight]{\textwidth}
		\begin{columns}[T]
			\begin{column}{0.5\textwidth}
	%			\vspace{4em}
				\begin{figure}[H]
					\centering
					\includegraphics[width=5.0cm]{F3.jpg}
				\end{figure}
			\end{column}
			\begin{column}{0.5\textwidth}
			\vspace{2em}
Преимущества Arc GIS. \\
В качестве рабочей программы для данной работы выбрана программа \alert{Arc GIS} (производство компании ESRI). \\
Arc GIS 9.1 в настоящее время безусловный мировой лидер среди ГИС многофункционального назначения. 
			\end{column}
		\end{columns}
	\end{minipage}
	
Преимущества Arc GIS 9.1 перед другими ГИС:

\begin{itemize}
	\item удобство и простота использования программы
	\item наличие множества функций для всевозможных видов обработки и анализа геоданных
	\item возможность обработки как растровых так и векторных данных
	\item улучшенное создание и распечатка карт
	\item встроенный язык программирования Avenue 
	\item совместимость: проекты, созданные в других ГИС можно конвертировать в ArcGIS, т.к. его форматы совместимы с другими программами.
\end{itemize}
   
\end{frame}

\subsection{Arc GIS: Structure}
\begin{frame}\frametitle{Arc GIS: Structure}
Модульная архитектура ArcGIS состоит из отдельных блоков, главные из которых следующие:
\begin{itemize}
	\item Arc Catalog – база данных ArcGIS
	\item Arc Map – главный блок для картографирования
	\item Arc Globe – для 3-мерной визуализации
	\item Arc Reader – для просмотра готовых карт
	\item Arc Tools – главный инструментарий программы, где создаются запросы
	\item ArcScan – для создания векторных карт на основе сканированных
	\item Arc Info - главное информационное ядро программы
\end{itemize}
\begin{figure}[H]
	\centering
		\includegraphics[width=8.0cm]{F4.jpg}
\end{figure}

\end{frame}

\section{Arc GIS: Functionality}
\subsection{GIS: Data Visualization}
\begin{frame}\frametitle{GIS: Data Visualization}

	\begin{minipage}[0.4\textheight]{\textwidth}
		\begin{columns}[T]
			\begin{column}{0.3\textwidth}
				\begin{figure}[H]
					\centering
					\subfloat {\includegraphics[width=3.0cm]{F17.jpg}}
						\hspace{1mm}
					\subfloat {\includegraphics[width=3.0cm]{F18.jpg}}
				\end{figure}
			\end{column}
			\begin{column}{0.7\textwidth}	
				\small{
				\begin{alertblock}{}
Преимущества использования геоинформационных технологий студентами в учебном процессе обусловлены следующими функциями ГИС вообще и Arc GIS в частности:
				\end{alertblock}

				\begin{block}{}
Функция наглядности ГИС обогащает круг географических представлений студентов, делает обучение более доступным,  развивает наблюдательность, их мышление и познавательные способности, помогает более глубокому и прочному усвоению учебного материала.  
				\end{block}
				
				\begin{alertblock}{}
Познавательная функция. При этом происходит опора на чувственную форму познания, в основе которой лежат ощущения - восприятия - представления
				\end{alertblock}
				
				\begin{block}{}
ГИС помогает сформировать у студентов гармоничный взгляд на мир, обеспечивающий его комплексное восприятие и лучшее понимание взаимосвязей между его составляющими.
				\end{block}

				\begin{block}{}
Результат: усвоение студентами материала при использовании ГИС происходит эффективно и результативно
				\end{block}
				}
			\end{column}
		\end{columns}
	\end{minipage}

\end{frame}

\subsection{GIS: Education}
\begin{frame}\frametitle{GIS: Education}

\begin{alertblock}{}
Воспитательная функция ГИС: работе с ГИС свойственны эмоциональность, воздействие на чувства студентов, интерес к работе. Объяснение и демонстрирование студентам функций и приемов работы с ГИС в классе (каждый работает на своем мониторе) увеличивает эффективность и стимулирует интерес
\end{alertblock}

\begin{figure}[H]
	\centering
		\subfloat {\includegraphics[width=3.6cm]{F15.jpg}}
			\hspace{1mm}
		\subfloat {\includegraphics[width=3.3cm]{F5.jpg}}
			\hspace{1mm}
		\subfloat {\includegraphics[width=3.3cm]{F6.jpg}}
\end{figure}

\begin{block}{}
Развитие культуры: решение разнообразных задач с ГИС способствует развитию общегеографической культуры, профессиональным навыкам, развивает кругозор и эрудированность студентов. Результат: обучение с ГИС активизирует творческие способности
\end{block}

\end{frame}

\subsection{GIS: Development}
\begin{frame}\frametitle{GIS: Development}

	\begin{minipage}[0.4\textheight]{\textwidth}
		\begin{columns}[T]
			\begin{column}{0.4\textwidth}
	%			\vspace{4em}
				\begin{figure}[H]
					\centering
					\includegraphics[width=3.5cm]{F7.jpg}
				\end{figure}
			\end{column}
			\begin{column}{0.6\textwidth}
\begin{block}{}
Развивающая функция ГИС. При cистематическом целенаправленном использовании ГИС студентами наблюдается ряд положительных изменений: ГИС способствует умственному развитию студентов. В процессе работы происходит активное обучение приемам наблюдения,  анализа и синтеза наблюдаемого, активизируется логическое мышление и анализ причинно-следственных связей, развивается способность принятия решений и построения выводов в процессе самостоятельной работы, улучшается и оттачивается общая компьютерная грамотность и навыки работы с разным ПО.
\end{block}
			\end{column}
		\end{columns}
	\end{minipage}

\begin{block}{}
\small{С помощью ГИС студенты учатся управлять информацией, 
проводить запросы по специальным базам данных (реляционные СУБД), 
получать ответы на вопросы аналитического характера: Кто владелец данного земельного участка? 
На каком расстоянии друг от друга расположены эти объекты?
Где расположена данная промзона?
Где есть места для строительства нового дома? 
Каков основный тип почв под еловыми лесами? 
Как повлияет на движение транспорта строительство новой дороги?}
\end{block}

\begin{block}{}
Результат: работа студентов с ГИС способствует их умственному развитию, активизирует логико-аналитическое мышление
\end{block}

\end{frame}

\subsection{GIS: Information}
\begin{frame}\frametitle{GIS: Information}

\begin{alertblock}{}
Информационная и экологичекая функция ГИС. ГИС несут значительную смысловую и информационную нагрузку, в том числе экологичесую, усвоение которых увеличивает кругозор и общую эрудицию студентов 
\end{alertblock}

\begin{block}{}
ГИС формируют у студентов пространственные представления и понятия о размещении природных и социально-экономических объектов и явлений
\end{block}

% 2 картинки
\begin{figure}[H]
	\centering
		\subfloat {\includegraphics[width=3.0cm]{F8.jpg}}
			\hspace{5mm}
		\subfloat {\includegraphics[width=3.0cm]{F9.jpg}}
\end{figure}

\begin{alertblock}{}
Работа с ГИС помогает студентам выявлять взаимосвязи между различными параметрами (например, почвами, климатом и урожайностью с/х культур), что важно при изучении географических дисциплин
\end{alertblock}

\begin{block}{}
Результат: работа студентов с ГИС повышает эрудицию и географическую и экологическую культуру, 
расширяют кругозор.
\end{block}

\end{frame}

\section{ArcGIS: Basics}
\begin{frame}\frametitle{ArcGIS: Basics}

	\begin{minipage}[0.4\textheight]{\textwidth}
		\begin{columns}[T]
			\begin{column}{0.5\textwidth}
				\vspace{2em}
				\begin{figure}[H]
					\centering
					\includegraphics[width=5.0cm]{F11.jpg}
				\end{figure}
			\end{column}
			\begin{column}{0.5\textwidth}
	%			\vspace{4em} 
\begin{alertblock}{}
Принципы работы в ArcGIS. Студенты при работе с ГИС оперируют с различными слоями тематической информации, мышью, выбирая объекты для пространственного анализа и обработки геоданных
\end{alertblock}

\begin{alertblock}{}
Информация в Arc GIS 9.1 хранится послойно, что позволяет студентам составлять всевозможные комбинации из тематических слоев; редактирование карт происходит 'на лету': можно составлять различные серийные тематические карты
\end{alertblock}
			\end{column}
		\end{columns}
	\end{minipage}

\begin{block}{}
ГИС полезно  представить студентам в виде функции; где область определения - база исходных данных (табличных, графических, текстовых), область существования -  графическое представление этих данных, а сама функциональная зависимость -  методика перевода одного в другое.
\end{block}

\end{frame}

\subsection{Arc GIS: Module Structure}
\begin{frame}\frametitle{Обучение модульной системе Arc GIS}

	\begin{minipage}[0.4\textheight]{\textwidth}
		\begin{columns}[T]
			\begin{column}{0.5\textwidth}		
	%			\vspace{4em}
				\begin{figure}[H]
					\centering
					\includegraphics[width=5.0cm]{F12.jpg}
				\end{figure}
			Обработка геоданных, знакомство с отдельными модулями Arc GIS. Основные этапы работы в обучении студентов должны включать следующие этапы: $\nearrow$
			\end{column}
			\begin{column}{0.5\textwidth}
\begin{alertblock}{}

\begin{itemize}
	\item выбор проекции карты
	\item работа с базами геоданных
	\item векторизация растровых данных
	\item (на примере оцифровки горизонталей рельефа) 
	\item изучение модуля Spatial Analyst ArcGIS (для построения цифровой модели рельефа)
	\item изучение модуля Arc Scene ArcGIS 9.1 (для построения 3-мерного изображения карты)
	\item создание таблиц в Arc GIS 9.1 
	\item моделирование и пространственный анализ в ArcGIS
	\item дизайн и художественное оформление карт
	\item создание тематических карт
\end{itemize}

\end{alertblock}
			\end{column}
		\end{columns}
	\end{minipage}
\end{frame}

\subsection{ArcGIS 9.1: Map Projections}
\begin{frame}\frametitle{ArcGIS 9.1: Map Projections}

\begin{minipage}[0.4\textheight]{\textwidth}
		\begin{columns}[T]
			\begin{column}{0.5\textwidth}

\small{\begin{alertblock}{}
Работа с картографическими проекциями полезна для студентов-географов, т.к. проекция карты - базовое понятие при работе с картами и обработке пространственной информации
\end{alertblock}

\begin{block}{}
Картографические проекции переводят данные из трехмерного пространства  на двухмерную поверхность. Выбор оптимальной проекции для пространственных данных требует от студентов рассмотрения всех типов данных и объектов, которые должна содержать карта, учета ограничений некоторых частных проекций назначения карты
\end{block}
}

\begin{alertblock}{}
ArcInfo и ArcView обеспечивают устойчивую поддержку многих проекций. При помощи ArcMap, модуля ArcInfo,
проекция может быть легко вызвана или изменена. 
\end{alertblock}

			\end{column}
			\begin{column}{0.5\textwidth}
	%			\vspace{4em}
				\begin{figure}[H]
					\centering
					\includegraphics[width=5.0cm]{F13.jpg}
				\end{figure}
\begin{block}{}
Щелчком правой кнопкой мыши на рамке изображения и последующим выбором строки Свойства вызываем таблицу Системы координат, которая показывает параметры текущей  координатной системы  и позволяет отобразить данные  в одной из более 60 заданных проекций или изменить параметры системы координат.
\end{block}
			\end{column}
		\end{columns}
	\end{minipage}
\end{frame}

\subsection{ArcGIS: Geomorphology}
\begin{frame}\frametitle{ArcGIS: Geomorphology}

\small{
\begin{alertblock}{}
Оцифровка рельефа. Студентам надо освоить векторизацию растровых данных на примере работы с рельефом, т.к. рельеф – основа и базис практически всех карт. Работа осуществляется с использованием основных модулей ArcGIS: ArcMap, ArcGlobe, ArcCatalog. 
\end{alertblock}

\begin{block}{}
На отдельных этапах работы используются также другие программы: Easy Trace или Autotrace (для ускоренной оцифровки), Surfer (для построения сетки рельефа и как аналог ArcGlobe).
\end{block}
}

\begin{figure}[H]
	\centering
		\subfloat {\includegraphics[width=3.5cm]{F14.jpg}}
			\hspace{1mm}
		\subfloat {\includegraphics[width=3.2cm]{F16.jpg}}
			\hspace{1mm}
		\subfloat {\includegraphics[width=3.5cm]{F19.jpg}}
\end{figure}

\small{
\begin{alertblock}{}
Основные этапы работы включают: оцифровку горизонталей по отсканированной растровой топокарте
занесение атрибутивных данных и информации о высотах в ArcCatalog, построение сетки высот и модели местности ЦМР (DEM), применение послойной окраски и утолщенных горизонталей в оформлении карты 3D- визуализация полученной карты с помощью модуля ArcGlobe использование метода отмывки рельефа.
\end{alertblock}
}

\end{frame}

\subsection{ArcGIS: Spatial Analyst. DEM}
\begin{frame}\frametitle{ArcGIS: Spatial Analyst. DEM}

\begin{alertblock}{}
Изучение модуля Spatial Analyst ArcGIS для построение цифровой модели рельефа. Логическим продолжением предыдущей темы является обучение посторению ЦМР с помощью модуля Spatial Analyst ArcGIS
\end{alertblock}

\begin{block}{}
На первом этапе работы активизируем модуль 3D Analyst ArcGIS. Открываем меню 3D Analyst и выбираем Create/ Modify TIN. Появляется диалоговое окно, в котором слева предлагается выбор слоев, на основе которых может быть построена TIN - модель 
\end{block}

\begin{figure}[H]
	\centering
		\subfloat {\includegraphics[width=4.3cm]{F20.jpg}}
			\hspace{1mm}
		\subfloat {\includegraphics[width=5.2cm]{F28.jpg}}
\end{figure}

\end{frame}

\subsection{ArcGIS: Spatial Analyst. TIN}
\begin{frame}\frametitle{ArcGIS: Spatial Analyst. TIN}

\begin{block}{}
Отмечаем галочкой слои, содержащие информацию о высоте в нашем случае поле 'AHEIGHT'. Можно отредактировать легенду TIN двойным щелчком мыши. В ArcMap есть широкие возможности для раскраски рельефа цветовыми шкалами
\end{block}

\begin{alertblock}{}
Затем выбираем способ классификации для построения легенды слоя абсолютных высот. Можно задать любое количество классов и любые интервалы между ними, цветовые шкалы так же можно редактировать 
\end{alertblock}

\begin{figure}[H]
	\centering
		\subfloat {\includegraphics[width=3.2cm]{F22.jpg}}
			\hspace{1mm}
		\subfloat {\includegraphics[width=3.4cm]{F23.jpg}}
			\hspace{1mm}
		\subfloat {\includegraphics[width=3.2cm]{F21.jpg}}
\end{figure}

\end{frame}

\subsection{ArcGIS: ArcScene}
\begin{frame}\frametitle{ArcGIS: ArcScene}
\small{
\begin{alertblock}{}
Изучение модуля ArcScene ArcGIS для построения 3D карты. Делаем слой оттенения (3D Analyst- Surface Analysis Hillshade), в окне задаем направление освещения (по умолчанию СЗ) и Altitude угол высоты солнца над горизонтом Azimuth (чем он ниже - тем длиннее тени).
\end{alertblock}
}
\begin{figure}[H]
	\centering
		\subfloat {\includegraphics[width=3.0cm]{F25.jpg}}
			\hspace{1mm}
		\subfloat {\includegraphics[width=3.0cm]{F26.jpg}}
\end{figure}

\begin{figure}[H]
	\centering
		\subfloat {\includegraphics[width=3.0cm]{F24.jpg}}
			\hspace{1mm}
		\subfloat {\includegraphics[width=3.0cm]{F27.jpg}}
\end{figure}

\small{
\begin{alertblock}{}
Экспорт в ArcScene: загрузить слой поверхности/ После добавления слоя в проект ArcScene подгружаем легенду .lyr, сохраненную в ArcMap.
\end{alertblock}
}
\end{frame}

\subsection{ArcSDE (1)}
\begin{frame}\frametitle{ArcSDE (1)}
	\begin{minipage}[0.4\textheight]{\textwidth}
		\begin{columns}[T]
			\begin{column}{0.5\textwidth}
	%			\vspace{4em}
				\begin{figure}[H]
					\centering
					\includegraphics[width=5.0cm]{F29.jpg}
				\end{figure}
\small{				
				\begin{block}{}
Топология позволяет моделировать в ГИС пространственные отношения: связность (связаны ли между собой линии дорожной сети?) и смежность (существует ли расстояние между двумя полигонами участков?). Знать принципы топологии важно.
				\end{block}

				\begin{block}{}
В ArcGIS есть возможность автономного редактирования баз геоданных: откреплять геообъекты из многопользовательской базы геоданных для работы с ними в полевых условиях
				\end{block}
}
			\end{column}
			\begin{column}{0.5\textwidth}
		\small{
			\begin{alertblock}{}
Обучение студентов работе с базами геоданных в ArcGIS 9.1 База геоданных – важнейшая компонента и информационное ядро ГИС. База геоданных – это технология для хранения разнородных геоданных, позволяющая повысить эффективность их хранения и использования в сложных проектах и системах.
			\end{alertblock}

			\begin{block}{}
Работая с базами геоданных, студенты получают навыки по использованию баз данных, т.к. устройство баз геоданных аналогично обычным (MS Access), и по обработке геоданных, обеспеченной благодаря топологиям.
			\end{block}

			\begin{alertblock}{}
Фундаментальная особенность баз геоданных - поддержка топологий, обеспечивающая логико-пространственную взаимосвязь разных объектов на территории.
			\end{alertblock}

			\begin{alertblock}{}
Топология полезна для контроля целостности совпадающей геометрии у разных классов объектов 
(совпадает ли береговая линия с границей страны?)
			\end{alertblock}
		}
			\end{column}
		\end{columns}
	\end{minipage}
	
\end{frame}

\subsection{ArcSDE (2)}
\begin{frame}\frametitle{ArcSDE (2)}
	\begin{minipage}[0.4\textheight]{\textwidth}
		\begin{columns}[T]
			\begin{column}{0.5\textwidth}
\begin{figure}[H]
	\centering
		\subfloat {\includegraphics[width=4.0cm]{F30.jpg}}
			\vspace{2mm}
		\subfloat {\includegraphics[width=4.0cm]{F30a.jpg}}
\end{figure}
			\end{column}
			\begin{column}{0.5\textwidth}
\small{
Обучение студентов работе с базами геоданных в ArcGIS.

\begin{block}{}
Главное хранилище данных в ArcGIS - сервер пространственных данных ArcSDE, созданный 
в 1994 г. и поддерживаемый различные форматы как других приложений ArcGIS (ArcView, ArcEditor), 
так и других ГИС-программ: Erdas Imagine для обработки растра, MapInfo для работы с векторными картами. 
\end{block}

\begin{alertblock}{}
ArcSDE позволяет использовать общую базу данных одновременно нескольким пользователям, что актуально для студенческих групп 
\end{alertblock}

\begin{block}{}
В ArcSDE можно сохранять варианты данных на определенный временной отрезок и хранить т.о. резервные копии и при желании вернуться к ним.
\end{block}
}
			\end{column}
		\end{columns}
	\end{minipage}

\end{frame}

\subsection{ArcGIS: ModelBuilder}
\begin{frame}\frametitle{ArcGIS: ModelBuilder}
Обучение студентов моделированию и пространственному анализу в ArcGIS
\small{
\begin{alertblock}{}
Для решения сложных задач геообработки (c использованием нескольких инструментов) можно создать модель путем связывания отдельных процессов в ModelBuilder. При построении модели можно мышью перетаскивать инструменты из окна ArcToolbox или дерева ArcCatalog. Для задания параметров модели существует отдельное окно, где студенты могут задавать критерии моделирования.
\end{alertblock}

\begin{block}{}

Объясняем студентам назначение клавиш: Контейнер с набором инструментов. Инструмент: запускает какую-то функцию геообработки. Скрипт: может быть написан на любом компилируемом языке программирования: Python, Jscript или VBScript. В ArcGIS также есть встроенный язык Avenue. Редактировать модель можно в окне Model Builder.
\end{block}
}
\begin{figure}[H]
	\centering
		\subfloat {\includegraphics[width=5.1cm]{F31.jpg}}
			\hspace{1mm}
		\subfloat {\includegraphics[width=4.7cm]{F32.jpg}}
\end{figure}

\end{frame}

\subsection{ArcGIS: Spatial Analysis}
\begin{frame}\frametitle{ArcGIS: Spatial Analysis}

\begin{alertblock}{}
Примеры результатов пространственного анализа, проведенного студентами при работе с ArcGIS. Для выделения районов, находящихся в доступности m или n км студентам предлагается выполнить запрос на поиск территорий, удовлетворяющих условию 'расстояние любой точки на границе данного полигонального объекта до другого доkжно быть менее или равным X'. 
\end{alertblock}

\begin{figure}[H]
	\centering
		\subfloat {\includegraphics[width=4.4cm]{F33.jpg}}
			\hspace{1mm}
		\subfloat {\includegraphics[width=4.0cm]{F33a.jpg}}
\end{figure}

\begin{block}{}
Подобные запросы можно проводить при экологическом анализе (выделить территории, находящиеся на расстоянии Х или Y км от фабрики, очага выбросов вредных веществ или катастрофы и т.д) и при других видах 
территориальной оценки. Пример: Зоны доступности магазинов в 500 и 300 метров.
\end{block}

\end{frame}

\subsection{Export: MS Excel}
\begin{frame}\frametitle{Export: MS Excel}
Обучение студентов работе с таблицами в Arc GIS 9.1; экспорт из MS Excel
	
\begin{alertblock}{}
Вся атрибутивная информация в ArcGIS хранится в табличной форме в модуле ArcCatalog,
поэтому необходимо обучить студентов их использованию и редактированию и экспорту/импорту.
\end{alertblock}

\begin{figure}[H]
	\centering
		\subfloat {\includegraphics[width=3.4cm]{F34.jpg}}
			\hspace{1mm}
		\subfloat {\includegraphics[width=4.0cm]{F35.jpg}}
\end{figure}

\begin{block}{}
В ArcGIS 9-й версии возможен экспорт таблиц из Excel. Для переноса данных из таблицы Excel в ArcGIS, сохраняем их в текстовом формате с разделителями – табуляциями в Excel -  DBF 3 (dBase III) 
\end{block}

\end{frame}

\subsection{Export Data}
\begin{frame}\frametitle{Export Data}

\begin{alertblock}{}
Добавляем файл в ArcMap (File - Add Data). В Source указываем местонахождение и имя файла. Добавленная в ArcMap таблица отражается в иерархическом 'дереве' проекта, где находятся все созданные карты, слои, растровые изображенияи снимки, модели, созданные в проекте и таблицы с данными.
\end{alertblock}

\begin{figure}[H]
	\centering
		\subfloat {\includegraphics[width=3.5cm]{F36.jpg}}
			\hspace{1mm}
		\subfloat {\includegraphics[width=2.8cm]{F37.jpg}}
			\hspace{1mm}
		\subfloat {\includegraphics[width=3.5cm]{F38.jpg}}
\end{figure}

\begin{block}{}
Подгруженную таблицу можно просматривать, редактировать и изменять. На основе табличных данных 
создаем точечный слой и добавляем его в проект (в отдельных столбцах хранятся координаты и атрибуты).
\end{block}

\end{frame}

\subsection{ArcGIS: Design Solutions}
\begin{frame}\frametitle{ArcGIS: Design Solutions}
\small{
\begin{alertblock}{}
Изучение принципов оформления карт активизирует творческие способности студентов; как правило, оформление карт - одна из самых интересных для них тем
\end{alertblock}

\begin{block}{}
ArcGIS предоставляет большие возможности художественного оформления карт: цветовые палитры,  графические символы, штриховки, управление слоями (прозрачность: совмещать одни тематические слои на фоне других) цветовые градиенты 'оттенение' для отмывки рельефа.
\end{block}
}
\begin{figure}[H]
	\centering
		\subfloat {\includegraphics[width=3.8cm]{F39.jpg}}
			\hspace{1mm}
		\subfloat {\includegraphics[width=4.0cm]{F40.jpg}}
\end{figure}
\small{
\begin{alertblock}{}
На одном из семинаров можно предложить студентам задание по составлению в ArcGIS графиков и диаграмм, полезных для представления и анализа разновременных социально-экономических данных, где показано изменение статистической информации по годам: изменение численности / плотности населения, ВВП страны или региона по годам и т.д. На основе данных студенты создают графики, пользуясь инструментом Chart в основном меню.
\end{alertblock}
}
\end{frame}

\section{ArcGIS: Thematic Mapping}

\subsection{Hydrology}
\begin{frame}\frametitle{Hydrology}

	\begin{minipage}[0.4\textheight]{\textwidth}
		\begin{columns}[T]
			\begin{column}{0.4\textwidth}

\begin{itemize}
	\item чертежи профилей водохранилищ
	\item составление схем речной сети и бассейнов и распределение
	\item речного стока в пределах бассейна
	\item расчет характеристик и параметров гидрографической сети
	\item моделирование подъема и спада воды в периоды 
	\item моделирование половодий и паводков
	\item анализ водопотребления 
	\item в районе, освоение человеком водоохранных зон
\end{itemize}
			\end{column}
			\begin{column}{0.6\textwidth}

			\begin{figure}[H]
					\centering
					\includegraphics[width=5.5cm]{F42.jpg}
			\end{figure}
$\Longleftarrow$ Используя ArcGIS студенты-гидрологи могут выполнять следующие задания.
\footnotesize{
\begin{block}{}
Территории в зонах затопления на ортофотоплане в ArcGIS. Разработка комплексных ГИС гидрологического назначения позволяет собрать воедино разобщенную гидрологическую информацию, и на основе фактических и прогнозных данных оперативно представлять сведения для работы паводковых комиссий. Поэтому для будущих гидрологов важно знать инструментарий и возможности ГИС в гидрологии.
\end{block}
}
			\end{column}
		\end{columns}
	\end{minipage}
\end{frame}

\subsection{Economic Geography}
\begin{frame}\frametitle{Economic Geography}

	\begin{minipage}[0.4\textheight]{\textwidth}
		\begin{columns}[T]
			\begin{column}{0.5\textwidth}
				\begin{figure}[H]
					\centering
					\includegraphics[width=4.0cm]{F41.jpg}
				\end{figure}
\small{	
\begin{alertblock}{}
Использование ArcGIS студентами экономико-географических специальностей. Основные направления применения ГИС: транспортная логистика, размещение объектов торговли и услуг, услуги, связанные с местоположением клиента (сфера телекоммуникаций и сотовых связей)
\end{alertblock}
}
			\end{column}
			\begin{column}{0.5\textwidth}
\small{
\begin{block}{}
Актуальная задача для студентов экономико-географов - размещение объектов и логистика, 
включающая:
\begin{itemize}
	\item многофакторный анализ территории, 
	\item транспортная инфраструктура, 
	\item планирование перевозок
	\item ценовые характеристики территории
	\item поиск оптимальных положений для новых объектов 
	\item отображение объектов: магазины, театры, кафе, дома, склады
	\item режимы землепользования
	\item оперативное управление парком транспортных средств
	\item демография (потенциальный спрос).
	\item оценка обеспеченности районов магазинами 
\end{itemize}
Для решения этих задач используем модули пространственного анализа ArcGIS: ArcGIS Spatial Analyst (на сплошных поверхностях). ArcGIS Network Analyst (на сетевых структурах).
\end{block}
}
			\end{column}
		\end{columns}
	\end{minipage}

\end{frame}

\subsection{Glaciology}
\begin{frame}\frametitle{Glaciology}
\small{
Использование ArcGIS в области гляциологии. Задачи, которые студенты могут решать, используя модули ArcGIS 9.1:
\begin{itemize}
	\item деградация ледников Арктики и Антарктики, высокогорных районов 
	\item криогенез в пределах ледникового бассейна
	\item расчет характеристик и параметров оледенения (толщина, цвет, плотность льда) 
	\item моделирование и прогнозирование изменения ледников
	\item анализ сокращения площади ледников
	\item изучение скорости сокращения ледников по данным космических снимков и карт за к.-л.
 промежутки времени (такие задачи можно решать при написании курсовых).
\end{itemize}
}
\begin{figure}[H]
	\centering
		\includegraphics[width=8.0cm]{F45.jpg}
\end{figure}
\small{
Ледниковый купол: изображение на космическом снимке и на карте. Совместное использование карт и снимков в ArcGIS позволяет студентам вычислять сокращение площадей ледников.
}
\end{frame}

\subsection{Photogrammetry and RS}
\begin{frame}\frametitle{Photogrammetry and RS}

\begin{alertblock}{}
Тематическое картографирование при изучении студентами дисциплины 'Фотограмметрия и основы дистанционного зондирования Земли'. \\
Использование современных средств ArcGIS 9.1 вместо громоздких стереокомпараторов намного повышает эффективность работы при обработке снимков:
\end{alertblock}

\begin{block}{}
С помощью модуля Stereo Analyst ArcGIS 9.1, кроме обычного двухмерного изображения возможен просмотр трехмерных данных.\\
Stereo Analyst ArcGIS 9.1 позволяет: проводить частичную обработку космических снимков обработывать два снимка, используя эффект стереопары
\end{block}

\begin{figure}[H]
	\centering
		\subfloat {\includegraphics[width=3.2cm]{F46.jpg}}
			\hspace{1mm}
		\subfloat {\includegraphics[width=3.2cm]{F47.jpg}}
			\hspace{1mm}
		\subfloat {\includegraphics[width=3.2cm]{F48.jpg}}
\end{figure}

\begin{alertblock}{}
Модуль ArcScene ArcGIS 9.1 дает возможность 3-мерного, объемного отображения карты, что выглядит очень эффектно и интересно для студентов
\end{alertblock}

\end{frame}

\subsection{Cadaster and Land Use}
\begin{frame}\frametitle{Cadaster and Land Use Planning}

\begin{alertblock}{}
Тематическое картографирование студентами географического факультета при изучении  дисциплины 'Кадастр и землеустройство'.
\end{alertblock}

\begin{block}{}
Использование ПО ArcGIS 9.1 при изучении  дисциплины 'Кадастр и землеустройство' или студентами архитектурных специальностей и землеустроительных факультетов помогает решать задачи градостроительного кадастра.
\end{block}

\begin{alertblock}{}
Наличие сведений о зданиях, сооружениях и других элементах градостроительной инфраструктуры позволит решать разнообразные аналитические задачи: функциональное зонирование, моделирование развития территорий, анализ ограничений с учетом трехмерных характеристик объектов, (этажность), плотность населения по районам
\end{alertblock}

\begin{block}{}
На основе ArcGIS студенты могут имитировать принятие управленческих решений:
например, проанализоровать эффективность постройки школ (на основе имеющейся в базе данных численности населения) проанализировать эффективности расположения избирательных участков выделить объекты, требующих капитального ремонта (на основе периода постройки зданий) 
\end{block}

\end{frame}

\subsection{Biogeography}
\begin{frame}\frametitle{Biogeography}
\small{
Тематическое картографирование для студентов кафедры 'Биогеография'.

\begin{alertblock}{}
Работа с интерактивными акртами способствует развитию экологического воспитания студентов, пробуждает интерес к теме. Разные масштабы и цели оценки состояния лесов требуют учета многих аспектов и параметров биоразнообразия: это реализуемо средствами ГИС (наложение слоев, анализ информации, моделирование)
\end{alertblock}

\begin{block}{}
Классические средства картографии в полной мере содержатся в  интерфейсе ArcGIS: художественно-оформительские средства, разнообразные и перекрестные штриховок. \\
Используя ArcGIS студенты могут: 
\begin{itemize}
	\item дешифрировать породы деревьев 
	\item составлять карты лесной таксации
	\item определять экосистемное биоразнообразие лесов	
	\item дешифрировать контуры лесных выделов по космоснимкам
	\item устанавливать текущее состояние лесных территорий (лесхозов)
	\item прогнозировать развитие лесов с учетом их естественной динамики,
	\item оценивать интенсивность лесопользования и сохранения биоразнообразия
	\item проводить расчет и визуализацию основных параметров/индексов биоразнообразия
\end{itemize}
Наполнение базы геоданных ГИС: рельеф, инфраструктура, ДДЗ, снимки, гидросеть, таблицы.
\end{block}
}
\end{frame}

\subsection{Landscape Studies}
\begin{frame}\frametitle{Landscape Studies}

\begin{alertblock}{}
Использование ArcGIS студентами кафедры 'Ландшафтоведение' для решения задач по экологичекому менеджменту и комплексной оценки территорий. Используя инструментарий ArcGIS студенты старших курсов кафедры 'Ландшафтоведение' могут проводить комплексные сложные исследования состояния ландшафтов территорий.
	\begin{itemize}
		\item создание комплексной базы данных, объединяющей сведения о компонентах геосистем
		\item создание электронной ландшафтной карты
		\item оценка устойчивости геосистем (и их отдельных компонентов) к различным видам антропогенного воздействия на основе интегральных балльных оценок по факторам устойчивости интеграция карт устойчивости ландшафтов к техногенной нагрузке с картами объектов 
		\item обустройство и выделение потенциально опасных для хозяйственного освоения участков территории (оценка экологического риска) 
		\item выбор оптимальной стратегии при проектировании с учетом экономической и экологической составляющих базы данных
		\item поддержка принятия управленческих решений организация на базе ArcGIS системы экологического мониторинга с использованием материалов наземных (полевых) наблюдений и ДДЗ, включая космические снимки высокого разрешения
	\end{itemize}
\end{alertblock}
\end{frame}

\subsection{Environmental Mapping}
\begin{frame}\frametitle{Environmental Mapping}

\begin{alertblock}{}
Модуль Spatial Analyst ArcView 3.2a из линейки ArcGIS 9.1 позволяет студентам выполнить ряд практических занятий по комплексной оценки геоэкологического состояния территорий и районирвания, которые могут быть тренировочными и полезными к написанию курсовых:
\begin{itemize}
	\item анализ пространственного распространения загрязнителей в почве или воде 
	\item анализ взаимного влияния типов данных характеристики геологической среды, гидросеть и гидрологические характеристики, 
	\item расстояние до населенных кварталов, 
	\item рельеф и геоморфологические характеристики,
	\item содержание тяжелых металлов в почвах и водах.
\end{itemize}
\end{alertblock}

\begin{block}{}
Основные этапы таких работ следующие: Особенность организации ландшафтной ГИС: комплексности и множество данных и материалов, которые используются для проведения экспертных оценок и принятия управленческих решений экологического менеджмента. Для анализа пространственных связей между элементами природных экосистем полезно устанавливать горячие связи HotLinks.
\end{block}

\end{frame}

\section{Conclusion}
\begin{frame}\frametitle{Conclusion}
\begin{itemize}
	\item В настоящее время использование ГИС  практически полностью вытеснило методы 
традиционной картографии (рисовка от руки, изготовление твердых фотокопий карт)
	\item Использование ГИС актуально в различных сферах геонаук: геология, геоморфология, гидрология, 
 океанология, геодезия, землепользование и кадастр, картография, экология, ландшафтоведение и др.
	\item На предприятиях среди выпускников востребованы владеющие ГИС-пакетами и новыми IT
	\item Среди различных ГИС-программ ArcGIS является наиболее оптимальным, популярным и
многофункциональным, что позволяет использовать его в различных сферах деятельности.
	\item Обучение использованию ГИС - актуально как для студентов-картографов, так и для других
специальностей и кафедр
	\item Изучение ПО ArcGIS активизирует интерес студентов к предмету, позволяет решать пространственно-аналитические задачи, развивает кругозор, укрепляет навыки работы с базами данных.
	\item Обучение студентов ГИС (ПО ArcGIS)  является в настоящее время актуальной и необходимой задачей
\end{itemize}
\end{frame}

\section{Thanks}
\begin{frame}{Thanks}
  	\centering \LARGE 
	\emph{Спасибо за внимание !}
\end{frame}

\section{Literature}
\begin{frame}\frametitle{Literature}
\begin{figure}[H]
	\centering
		\includegraphics[width=11.5cm]{liter.jpg}
\end{figure}
\end{frame}

\begin{frame}\frametitle{Bibliography}
%\footnotesize{Author's publications on Cartography, Mapping, Geography, Environment, GIS and Landscape Studies:}
	\nocite{*}
	\printbibliography[heading=none]
\end{frame}

%%%%%%%%%%% Bibliography %%%%%%%	

%Changing the font size locally (from biggest to smallest):	
%\Huge
%\huge
%\LARGE
%\Large
%\large
%\normalsize (default)
%\small
%\footnotesize
%\scriptsize
%\tiny

\end{document}