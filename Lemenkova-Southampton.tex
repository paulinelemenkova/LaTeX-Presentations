\documentclass[pdflatex,compress,8pt,
	xcolor={dvipsnames,dvipsnames,svgnames,x11names,table},
	hyperref={colorlinks = true,breaklinks = true, urlcolor = NavyBlue, breaklinks = true}]{beamer}
\usetheme{Marburg}

\usepackage[super]{nth}
\usepackage{amsmath}
\usepackage{subfig}

% ----------------------------------------------------------------------------
% *** START BIBLIOGRAPHY <<<
% ----------------------------------------------------------------------------
\usepackage[
	backend=biber, 
	style = numeric,
%	style=phys, % без doi
	maxbibnames=99,
	citestyle=numeric,
	giveninits=true,
	isbn=true,
	url=true,
	natbib=true,
	sorting=ndymdt,
	bibencoding=utf8,
	useprefix=false,
	language=auto, 
	autolang=other,
	backref=true,
	backrefstyle=none,
	indexing=cite,
]{biblatex}
\DeclareSortingTemplate{ndymdt}{
  \sort{
    \field{presort}
  }
  \sort[final]{
    \field{sortkey}
  }
  \sort{
    \field{sortname}
    \field{author}
    \field{editor}
    \field{translator}
    \field{sorttitle}
    \field{title}
  }
  \sort[direction=descending]{
    \field{sortyear}
    \field{year}
    \literal{9999}
  }
  \sort[direction=descending]{
    \field[padside=left,padwidth=2,padchar=0]{month}
    \literal{99}
  }
  \sort[direction=descending]{
    \field[padside=left,padwidth=2,padchar=0]{day}
    \literal{99}
  }
  \sort{
    \field{sorttitle}
  }
  \sort[direction=descending]{
    \field[padside=left,padwidth=4,padchar=0]{volume}
    \literal{9999}
  }
}

\addbibresource{Southampton.bib}%   \tiny  \scriptsize \footnotesize \normalsize
\renewcommand*{\bibfont}{\scriptsize} % 

\setbeamertemplate{bibliography item}{\insertbiblabel}

% ----------------------------------------------------------------------------
% *** END BIBLIOGRAPHY <<<
% ----------------------------------------------------------------------------

%%%%%%%%%%%%%%%%%%%%%%%%%%%%%%%%%

\title{Flood Hazards: A Case Study of the Floods in Bangladesh, Asia}
\author{Polina Lemenkova}
\subtitle{Presented at the Course GEOG6023\\ 'Physical Geography in Environmental Management'}
\institute{University of Southampton, School of Geography, \\
MSc GEM Studies, Erasmus Mundus Scholarship \\
Southampton, England (UK)}
\date{November 30, 2009}

\begin{document}
\begin{frame}
           \titlepage
\end{frame}

\section*{Outline}
\begin{frame}
           \tableofcontents
\end{frame}

\section{Introduction}
\subsection{Floods in the World}
\begin{frame}\frametitle{Floods in the world}
\begin{figure}[H]
	\centering
		\includegraphics[width=9.0cm]{F1.jpg}
\end{figure}
\end{frame}

\subsection{Floods in Asia}
\begin{frame}\frametitle{Floods in Asia}
\begin{figure}[H]
	\centering
		\includegraphics[width=9.0cm]{F2.jpg}
\end{figure}
\end{frame}

\section{Bangladesh}
\begin{frame}\frametitle{Bangladesh}
\begin{minipage}[0.4\textheight]{\textwidth}
\begin{columns}[T]
\begin{column}{0.5\textwidth}
\vspace{2em}
\begin{figure}[H]
	\centering
		\includegraphics[width=5.0cm]{F3.jpg}
\end{figure}
\footnotesize{Bangladesh: areas vulnerable to flooding}
\end{column}
\begin{column}{0.5\textwidth}
\vspace{2em} 
Key points on Bangladesh: 
\footnotesize{
\begin{itemize}
	\item Bangladesh is among the countries that are the most affected by climate change
	\item Frequent natural disasters, loss of life, damage to infrastructure and economic assets, impacts on lives and livelihoods
	\item Floods, tropical cyclones, storm surges and droughts are likely to become more frequent and severe in the coming years
	\item Lies in the delta of three of the largest rivers in the world – the Brahmaputra, the Ganges and the Meghna
	\item Mostly low and flat topography, susceptible to river and rainwater flooding
	\item More than 50 million of people still live in poverty
	\item Many people live in remote or ecologically fragile parts of the country, such as river islands
	\item Approximately one quarter of the country inundated each year
	\item The poorest and most vulnerable living in the exposed areas suffer the most
\end{itemize}
}
\end{column}
\end{columns}
\end{minipage}
\end{frame}

\subsection{The Flood Plain Area of the Meghna River}
\begin{frame}\frametitle{Meghna River}
\begin{minipage}[0.4\textheight]{\textwidth}
\begin{columns}[T]
\begin{column}{0.5\textwidth}
\vspace{2em}
\begin{figure}[H]
	\centering
		\includegraphics[width=4.7cm]{F4.jpg}
\end{figure}
\footnotesize{Meghna river flood plain area}
\end{column}
\begin{column}{0.5\textwidth}
\vspace{2em} 
Meghna river flood plain area: 
\small{\begin{itemize}
	\item Ideal case for examining relationship between the climate change, hazard and socio-economic vulnerability
	\item One of the highest flood hazard area in the world
	\item One of the poorest and the most flood-prone areas of Bangladesh
	\item Inhabited with more than four hundred thousand people
	\item Residents mostly farmers (farming rice, wheat, vegetables, oil seeds)
	\item Heavy monsoon rainfall generates excessive flows in the rivers and causes floods almost every year
	\item Suffered from devastating floods over the past 20 years in 1988, 1996, 1998 and 2004
\end{itemize}}
\end{column}
\end{columns}
\end{minipage}
\end{frame}
	
\subsection{Socio-Economic Vulnerability}
\begin{frame}\frametitle{Bangladesh: Socio-Economic Vulnerability}
\begin{itemize}
	\item Poorer segments of society live closer to the river
	\item Inundation levels significantly higher for poorer households
	\item Impoverished face a higher risk of flooding and are more vulnerable
	\item The higher the exposure level and income inequality, the less access to a natural resourses
	\item Average damage costs and coping capacity higher for wealthier households
	\item Households with higher income take preventive measures and have significantly lower damage costs
	\item Unequal income distributions in villages with higher risk exposure
	\item The least well prepared both in terms of household-level and community level face the highest risk of flooding
\end{itemize}
\end{frame}

\section{Impacts of the Flood Hazards in Bangladesh}
\begin{frame}\frametitle{Impacts of the Flood Hazards in Bangladesh}
\small{\textcolor{red}{Impact of Event = $\sum$ Intensity of Event * $\sum$ Baseline Conditions * $\sum$ Adaptive Capacity}}\\
In the coastal areas of Bangladesh communities are impacted by the following factors:
\begin{figure}[H]
	\centering
		\includegraphics[width=9.0cm]{F5.jpg}
\end{figure}
\end{frame}

\subsection{Impacts of the Major Floods in Bangladesh}
\begin{frame}\frametitle{Impacts of the Major Floods in Bangladesh}
Impacts of Major Floods in Bangladesh during the last 50 years
\begin{figure}[H]
	\centering
		\includegraphics[width=9.0cm]{F6.jpg}
\end{figure}
\end{frame}

\subsection{Vulnerability Factors}
\begin{frame}\frametitle{Vulnerability Factors}
Factors of vulnerability in Bangladesh which make impacts of flood hazards sensible:
\begin{itemize}
	\item Geographic location
	\begin{itemize}
		\item long coast line
		\item effect of saline water intrusion in the estuaries
		\item vast low-lying landmass
		\item extremely dynamic coastal geomorphological processes
	\end{itemize}
	\item Socio-demographic and economic features
	\begin{itemize}
		\item high population density
		\item nature-dependant traditional agricultural practices
	\end{itemize}
\end{itemize}
\begin{figure}[H]
	\centering
		\includegraphics[width=9.0cm]{F7.jpg}
\end{figure}
\end{frame} 

\subsection{Impacts of the Flood Hazards}
\begin{frame}\frametitle{Impacts of the Flood Hazards}
General impacts of flood hazards on the population:
\begin{itemize}
	\item initial economic conditions (poor or non-poor)
	\item location (coastal or non-coastal, rural or urban)
	\item gender and general health condition (capacity to resist illnesses, stresses)
	\item reduced livelihood options (loss of agriculture, restriction of movement)
\end{itemize}
Particular impacts of flood hazards on human health:
\begin{itemize}
	\item temperature rise
	\item degrading water quality as well as shortage leads to illnesses: cholera, diarrhea, dysentery, malaria and typhoid
	\item widespread malnutrition (due to shortage of water)
	\item women: increased risks of the involuntary foetus abortion in the coastal areas due to rising salinity leading to hypertension
\end{itemize}
\begin{figure}[H]
	\centering
		\subfloat {\includegraphics[width=6.0cm]{F8.jpg}}
			\hspace{5mm}
		\subfloat {\includegraphics[width=2.5cm]{F9.jpg}}
\end{figure}
\end{frame} 

\section{Conclusion}
\begin{frame}\frametitle{Conclusion}
Table 2. Causes of Impacts, Vulnerable Areas and Impacted Sectors in Bangladesh:
\begin{figure}[H]
	\centering
		\includegraphics[width=9.0cm]{F10.jpg}
\end{figure}


During the flood hazard 
\begin{itemize}
	\item the poor, 
	\item poor-healthy, 
	\item women will suffer much more disproportionately than the well-being and healthy men, more so in the coastal and rural areas than elsewhere.
\end{itemize}
\end{frame}

\section{Thanks}
\begin{frame}{Thanks}
  	\centering \LARGE 
  	\emph{Thank you for attention !}\\
\end{frame}

%%%%%%%%%%% Bibliography %%%%%%%

\section{Bibliography}
\Large{Bibliography}\ %\\vspace{1em}
\nocite{*}
\printbibliography[heading=none]

\end{document}

%Changing the font size locally (from biggest to smallest):	
%\Huge
%\huge
%\LARGE
%\Large
%\large
%\normalsize (default)
%\small
%\footnotesize
%\scriptsize
%\tiny

\end{document}