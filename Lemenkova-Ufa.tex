% All rights to the corporate identity of University of Amsterdam and its elements
% belong to the University of Amsterdam.
% Please refer to the following page for further information:
% https://extranet.uva.nl/en/a-z/content/corporate-communication/corporate-communication.html
%
% Copyright (c) 2015 Franz Geiger
%
% Permission is hereby granted, free of charge, to any person obtaining a copy
% of this software and associated documentation files (the "Software"), to deal
% in the Software without restriction, including without limitation the rights
% to use, copy, modify, merge, publish, distribute, sublicense, and/or sell
% copies of the Software, and to permit persons to whom the Software is
% furnished to do so, subject to the following conditions:
%
% The above copyright notice and this permission notice shall be included in
% all copies or substantial portions of the Software.
%
% THE SOFTWARE IS PROVIDED "AS IS", WITHOUT WARRANTY OF ANY KIND, EXPRESS OR
% IMPLIED, INCLUDING BUT NOT LIMITED TO THE WARRANTIES OF MERCHANTABILITY,
% FITNESS FOR A PARTICULAR PURPOSE AND NONINFRINGEMENT.  IN NO EVENT SHALL THE
% AUTHORS OR COPYRIGHT HOLDERS BE LIABLE FOR ANY CLAIM, DAMAGES OR OTHER
% LIABILITY, WHETHER IN AN ACTION OF CONTRACT, TORT OR OTHERWISE, ARISING FROM,
% OUT OF OR IN CONNECTION WITH THE SOFTWARE OR THE USE OR OTHER DEALINGS IN
% THE SOFTWARE.
%
% Additionally, this example document may be used with the template without
% above license notice.

\documentclass[pdflatex,compress,8pt,
	xcolor={dvipsnames,dvipsnames,svgnames,x11names,table},
	hyperref={colorlinks = true,
	breaklinks = true, 
	urlcolor = NavyBlue, 
	breaklinks = true}]{beamer}
% Load University of Amsterdam theme
% Options:
% 	logo: Select whether to use the Dutch or English logo in header.
% 		Either 'NL' or 'ENG'. Default: ENG
% 	maincolor: Select which main color to use for title page and as highlight color.
%		Possible values are: Any color definition. Default: uva-rood
% 		This theme pre-defines colors from the University of Amsterdam corporate identity:
% 			uva-zwart	(general UvA Black)
% 			uva-rood	(general UvA Red)
% 			uva-feb-groen	(Green for Faculty of Economics and Business)
% 			uva-fgw-blauw	(Blue for Faculty of Humanities)
% 			uva-fdr-donker-rood	(Dark Red for Faculty of Law)
% 			uva-gen-groen	(Green for Faculty of Medicine (AMC))
% 			uva-fnwi-paars	(Purple for Faculty of Science)
% 			uva-fmg-oranje	(Orange for Faculty of Social and Behavioural Sciences)
% 			uva-iis-licht-blauw	(Light Blue for Institute of Interdisciplinary Studies)
% 			uva-ilo-rood	(Red for Interfacultaire Lerarenopleidingen)
% 		Please refer to the following page for usage instructions regarding the
% 		corporate identity colors: https://extranet.uva.nl/en/a-z/content/corporate-communication/corporate-identity-elements/colour/colour-kopie.html
%\usetheme[logo=ENG, maincolor=uva-rood]{UniversityOfAmsterdam}
\usetheme[logo=ENG, maincolor=uva-rood]{UniversityOfAmsterdam}

\usepackage[utf8]{inputenc}
\usepackage{hyperref}
\usepackage[english]{babel}
\usepackage[T1]{fontenc}
\usepackage{gensymb} % degree symbol
\usepackage[super]{nth}
\usepackage{amsmath}
\usepackage{subfig}
\usepackage{csquotes}
\usepackage{graphicx} % to insert figures

\setcounter{tocdepth}{3}
\setcounter{secnumdepth}{3}

\setbeamertemplate{section in toc}{%
  {\color{uva-rood}\inserttocsectionnumber.}~\inserttocsection}
\setbeamercolor{subsection in toc}{bg=white,fg=structure}
\setbeamertemplate{subsection in toc}{%
  \hspace{1.2em}{\color{uva-rood}\rule[0.3ex]{3pt}{3pt}}~\inserttocsubsection\par}

%%%%%%%%%%%%%%%%%%%%%%%%%%%%

% Путь к файлам с иллюстрациями
\graphicspath{{fig/}} % path to folder with Figures

% ----------------------------------------------------------------------------
% *** START BIBLIOGRAPHY <<<
% ----------------------------------------------------------------------------
\usepackage[
backend=biber, 
%	style = numeric,
%	style=ieee,
%	style=nature,
%	style=science,
%	style=apa,
%	style=mla,
	style = phys,
	maxbibnames=99,
	citestyle=numeric,
	giveninits=true,
	isbn=true,
	url=true,
	natbib=true,
	sorting=ndymdt,
	bibencoding=utf8,
	useprefix=false,
	language=auto, 
	autolang=other,
	backref=true,
	backrefstyle=none,
	indexing=cite,]{biblatex}
\DeclareSortingTemplate{ndymdt}{
  \sort{
    \field{presort}
  }
  \sort[final]{
    \field{sortkey}
  }
  \sort{
    \field{sortname}
    \field{author}
    \field{editor}
    \field{translator}
    \field{sorttitle}
    \field{title}
  }
  \sort[direction=descending]{
    \field{sortyear}
    \field{year}
    \literal{9999}
  }
  \sort[direction=descending]{
    \field[padside=left,padwidth=2,padchar=0]{month}
    \literal{99}
  }
  \sort[direction=descending]{
    \field[padside=left,padwidth=2,padchar=0]{day}
    \literal{99}
  }
  \sort{
    \field{sorttitle}
  }
  \sort[direction=descending]{
    \field[padside=left,padwidth=4,padchar=0]{volume}
    \literal{9999}
  }
}

\addbibresource{Ufa.bib}%   \tiny \scriptsize \footnotesize \normalsize
\renewcommand*{\bibfont}{\scriptsize} % 

\setbeamertemplate{bibliography item}{\insertbiblabel}

% ----------------------------------------------------------------------------
% делать footnote \title[Short Title]{Long Title}
\makeatletter
\setbeamertemplate{footline}{%
\leavevmode%
\hbox{\begin{beamercolorbox}[wd=.24 \paperwidth,ht=2.5ex,dp=1.125ex,leftskip=.01cm plus1fill,rightskip=.05cm]{author in head/foot}%
            \usebeamerfont{title in head/foot}\insertshortauthor
    \end{beamercolorbox}%
    \begin{beamercolorbox}[wd=.76\paperwidth,ht=2.5ex,dp=1.125ex,leftskip=.05cm,rightskip=.15cm plus1fil]{title in head/foot}%
        \usebeamerfont{title in head/foot}\insertshorttitle{}
        \insertframenumber{} / \inserttotalframenumber \ \hspace*{2ex} 
    \end{beamercolorbox}}%
    \vskip0pt%
}
\makeatother

%-------------------------------------------------------
% ----------------------------------------------------------------------------
% *** END BIBLIOGRAPHY <<<
% ----------------------------------------------------------------------------

%%%%%%%%%%%%%%%%%%%%%%%%%%%%%%%%%%%%%%%

\author{Polina Lemenkova}
\title[Modelling Landscape Changes and Detecting Land Cover Types by Means of the RS Data and ILWIS GIS]{\Large{Modelling Landscape Changes and Detecting Land Cover Types by Means of the Remote Sensing Data and ILWIS GIS}}
\subtitle{\small{Presented at \nth{3} International Conference \emph{Information Technologies. Problems and Solutions}\\
Venue: Ufa State Petroleum Technological University. \\
Location: Ufa, Bashkortostan, Russia}}
\institute{Arctic Center, University of Lapland}
\date{May 20, 2015}

\begin{document}

% Title page (plain in order to hide regular background and footline)
\frame[plain]{\titlepage}

\section*{Table of Contents}
\begin{frame}{Table of Contents}
    \begin{columns}[onlytextwidth,T]
        \begin{column}{.45\textwidth}
            \small{\tableofcontents[sections=1-4]}
        \end{column}
        \begin{column}{.45\textwidth}
            \small{\tableofcontents[sections=5-15]}
        \end{column}
    \end{columns}
\end{frame}

\section{Introduction}
\subsection{Research Summary}
\begin{frame}\frametitle{Research Summary}

\begin{alertblock}{Application}
GIS and RS application for environmental studies of Yamal Peninsula, Russian North
\end{alertblock}

\begin{block}{Data}
Landsat TM scenes for 1988, 2001 and 2011
\end{block}

\begin{alertblock}{Aim}
Analysis of the land cover change dynamics in the past two decades (1988-2011). Mapping vegetation in tundra landscapes
\end{alertblock}

\begin{block}{ILWIS GIS}
GIS spatial analysis tools and Landsat imagery, calculation of NDVI by ILWIS GIS
\end{block}

\end{frame}

\subsection{Study Area}
\begin{frame}\frametitle{Study Area}
\begin{minipage}[0.4\textheight]{\textwidth}
\begin{columns}[T]
\begin{column}{0.5\textwidth}
\begin{figure}[H]
	\centering
		\includegraphics[width=4.0cm]{f02.jpg}
\end{figure}
Source: B. C. Forbes
\end{column}
\begin{column}{0.5\textwidth}
\begin{figure}[H]
	\centering
		\includegraphics[width=4.0cm]{f03.jpg}
\end{figure}

\begin{alertblock}{Location}
Yamal Peninsula, north Russia. \\
Map source: Web.
\end{alertblock}

\end{column}
\end{columns}
\end{minipage}
\end{frame}

\section{Yamal Peninsula: Geographic Settings}
\subsection{Climate and Environment}
\begin{frame}\frametitle{Climate and Environment}
\begin{minipage}[0.4\textheight]{\textwidth}
\begin{columns}[T]
\begin{column}{0.5\textwidth}
Yamal Peninsula: geomorphology : flat geomorphology, elevations lower than 90 m
Processes:
 \begin{itemize}
        	\item seasonal flooding,
	\item active erosion processing,
	\item permafrost distribution,
	\item cryogenic landslides formation
\end{itemize}
Landslides affect local ecosystem structure. \\
Landslides change vegetation types recovering after the disaster.
\end{column}
\begin{column}{0.5\textwidth}
Landscapes of Yamal. 
\begin{figure}[H]
	\centering
		\includegraphics[width=5.0cm]{f04.jpg}
\end{figure}
Source: http://pixtale.net/
\end{column}
\end{columns}
\end{minipage}
\end{frame}

\subsection{Landscapes}
\begin{frame}\frametitle{Landscapes of the Yamal Peninsula - I}
\begin{figure}[H]
	\centering
		\includegraphics[width=9.5cm]{f05.jpg}
\end{figure}
\end{frame}

\begin{frame}\frametitle{Landscapes of the Yamal Peninsula - II}
\begin{figure}[H]
	\centering
		\includegraphics[width=11.0cm]{f06.jpg}
\end{figure}
Dry grass heath tundra (left). Sedge grass tundra (center). Dry short shrub tundra (right)
\begin{figure}[H]
	\centering
		\includegraphics[width=7.0cm]{f07.jpg}
\end{figure}
Landscapes of Yamal (left). Sphagnum moss (right)
\end{frame}

\begin{frame}\frametitle{Landscapes of the Yamal Peninsula - III}
\begin{figure}[H]
	\centering
		\includegraphics[width=9.5cm]{f08.jpg}
\end{figure}
Dry short shrub sedge tundra (left). Wetlands (right)
\begin{figure}[H]
	\centering
		\includegraphics[width=7.0cm]{f09.jpg}
\end{figure}
Short shrub tundra
\end{frame}

\section{Methodology}
\begin{frame}\frametitle{Methodology: ILWIS GIS}
\begin{minipage}[0.4\textheight]{\textwidth}
\begin{columns}[T]
\begin{column}{0.5\textwidth}
Technical tools: The RS data processing was performed in ILWIS GIS software.
Research Methods: 
\begin{itemize}
	\item Image interpretation (Landsat TM scenes).
	\item Supervised classification
\end{itemize}
Following working steps summarize research scheme used in this research:
\begin{itemize}
	\item Data pre-processing
	\item Creation of image composites of several bands
	\item Supervised classification using various classifiers 
	\item Spatial analysis and interpretation of the results
	\item Time series analysis for detecting changes
	\item Final GIS mapping
\end{itemize}
\end{column}
\begin{column}{0.5\textwidth}
\begin{figure}[H]
	\centering
		\includegraphics[width=5.0cm]{f10.jpg}
\end{figure}
ILWIS GIS
\end{column}
\end{columns}
\end{minipage}
\end{frame}

\subsection{Minimum Distance Method}
\begin{frame}\frametitle{Minimum Distance Method}

\begin{alertblock}{NDVI}
Calculation of vegetation indices, e.g. Normalized Difference Vegetation Index (NDVI), is one of the most useful attempts in biogeographical research methods.
\end{alertblock}

\begin{examples}{Disadvantages:}
Technical details in modeling approach of image recognition: errors in pixels classification due to the ambiguity
\end{examples}

\begin{block}{Concept}
The principle of Minimum Distance method used for classification is based on calculating the shortest straight-line distance in the Euclidian coordinate system from each pixel’s DN to the pattern pixels of the land cover classes.
\end{block}

\begin{examples}{Misclassification:}
The misclassification by the Minimum Distance method may occur: erroneous recognition of some pixels, insufficient representation of classes.
\end{examples}

\end{frame}

\subsection{Workflow}
\begin{frame}\frametitle{Workflow}
Data pre-processing
\begin{itemize}
         \item import .img into ASCII raster format (GDAL). After converting, each image contained collection of 7 raster bands 
	\item Pre-processing (visual color and contrast enhancement) 
	\item Geographic referencing of Landsat scenes, initially based on  WGS 1984 datum: UTM (Universal Transverse Mercator) Projection, Eastern Zone 42, Northern Zone W, (Georeference Corner Editor)
	\item Crop of study area: the area of interest (AOI) was identified and cropped on the raw images. This area shows Bovanenkovo region in a large scale and best represents typical tundra landscapes.
	\item Supervsised Classification vi.GIS visualization and mapping
\end{itemize}
\end{frame}

\subsection{Data Import and Conversion}
\begin{frame}\frametitle{Data Import and Conversion}
\begin{itemize}
        \item Test area selection (Mask): 67\degree 00' - 72\degree 00' E - 70\degree 00' - 71\degree 00' N. 
	\item 3 selected  Landsat TM satellite images show Yamal region in 1988, 2001, 2011.
	\item Time span: 23 years (1988, 2001, 2011). 
	\item Summer months selected for vegetation assessment. 
	\item Data conversion / original images in format .TIFF converted to Erdas Imagine .img.
\end{itemize}
\begin{figure}[H]
	\centering
		\includegraphics[width=10.0cm]{f11.jpg}
\end{figure}
Initial remote sensing data, left to right: Landsat TM 1988, bands 7-3-1; Landsat TM 2011, pseudo natural colors composite; Landsat ETM + 2001 bands 6-3-1.
\end{frame}

\subsection{Data Georeferencing}
\begin{frame}\frametitle{Georeferencing: Google Earth}
\begin{figure}[H]
	\centering
		\includegraphics[width=10.0cm]{f19.jpg}
\end{figure}
\end{frame}

\subsection{Supervised Classification}
\begin{frame}\frametitle{Supervised Classification of the Landsat TM Image}
\begin{figure}[H]
	\centering
		\includegraphics[width=10.0cm]{N.jpg}
\end{figure}
\end{frame}

\section{Results}
\subsection{Computations}
\begin{frame}\frametitle{Computing Pixels on Various Land Cover Classes}
\begin{figure}[H]
	\centering
		\includegraphics[width=9.5cm]{T1.jpg}
\end{figure}
\end{frame}

\subsection{Mapping}
\begin{frame}\frametitle{Land Cover Classes}
\begin{figure}[H]
	\centering
		\subfloat {\includegraphics[width=3.5cm]{F22.jpg}}
		\subfloat {\includegraphics[width=3.6cm]{F23.jpg}}
		\subfloat {\includegraphics[width=3.6cm]{F24.jpg}}
\end{figure}
Results show classified maps of the selected region of Bovanenkovo on 1988, 2001 and 2011 yr (from left to right).
GIS mapping is performed using image classification. Results of the supervised classification show 3 maps of the vegetation distribution (above). 
\end{frame}

\subsection{Comments}
\begin{frame}\frametitle{Comments}
Classification is based on the relationship between the spectral signatures and object variables, i.e. vegetation types. Water areas are defined as “no vegetation” class.

\begin{block}{1988}
For year 1988 “forest” class covered 8,188,926 pixels, which is 11,32\% from the total amount.
\end{block}

\begin{examples}{1988:}
Maximal area, except for water, is covered by the shrubland (15,29\% from the total).
\end{examples}

\begin{block}{2011}
For year 2011, the percentage of the shrubland decreased down to 6,26\%, while area of forests increased from 11,32 to 15,97\%.
\end{block}

\begin{examples}{2011:}
The area of grass remained relatively stable with values slightly increasing to about 2\%
\end{examples}

\end{frame}

\subsection{Statistical Histograms}
\begin{frame}\frametitle{Statistical Histograms: 1988 and 2011}
\begin{figure}[H]
	\centering
		\subfloat {\includegraphics[width=6.5cm]{F25.jpg}}
			\vspace{2mm}
		\subfloat {\includegraphics[width=6.5cm]{F26.jpg}}
\end{figure}
\end{frame}

\section{Discussion}
\begin{frame}\frametitle{Discussion}
\begin{itemize}
        \item This research presented GIS based studies of the environment of Yamal Peninsula
        \item The study is technically based on ILWIS GIS, effective tool for spatial analysis
        \item The results of the spatial analysis are presented as 3 GIS maps illustrating changes in vegetation based on the image analysis using Landsat TM.
        \item Calculated land cover changes indicated vegetation dynamics in years 1988, 2001 and 2011.
        \item Application of the RS data is especially important for studies of the northern ecosystems, because it enables studying remotely located areas of Arctic
	\item GIS-based processing of the RS data (Landsat TM) improves technical aspects of the landscape studies and monitoring
	\item The results show successful use of ILWIS GIS software for spatio-temporal classification of the satellite images aimed at ecological mapping.
	\end{itemize}
\end{frame}

\section{Conclusion}
\begin{frame}\frametitle{Conclusion}

\begin{alertblock}{Time-Series Analysis}
Remote sensing plays important role in land use studies and serves as a valuable source of spatial information for the time series analysis.
\end{alertblock}

\begin{block}{ILWIS GIS}
Using enhanced ILWIS GIS tools to analyze and process satellite imagery contributes to the environmental analysis of the land cover changes. The classification used in the current work is pixel-based aimed to allocate and categorize pixels on the image to the created classes.
\end{block}

\begin{examples}{Remote Sensing:}
While traditional methods for vegetation monitoring are fieldwork and ground surveys, usually performed in large-scale areas, the use of remote sensing techniques enables to monitor extended areas in a small scale, as well as to assess temporal changes.
\end{examples}

\end{frame}

\section{Thanks}
\begin{frame}{Thanks}
  	\centering \LARGE 
  	\emph{Thank you for attention !}\\
	\vspace{5em}
\normalsize
Acknowledgement: \\
Current research has been funded by the \\
\emph{Finnish Centre for International Mobility (CIMO)} \\
Grant No. TM-10-7124, for author's research stay at \\
Arctic Center, University of Lapland (July 1 - December 31, 2011), \\
Rovaniemi, Finland.
\end{frame}

%%%%%%%%%%% Bibliography %%%%%%%

\section{Bibliography}
\begin{frame}[allowframebreaks]\frametitle{Bibliography}
	\nocite{*}
	\printbibliography[heading=none]
\end{frame}

\end{document}

%Changing the font size locally (from biggest to smallest):	
%\Huge
%\huge
%\LARGE
%\Large
%\large
%\normalsize (default)
%\small
%\footnotesize
%\scriptsize
%\tiny

\end{document}