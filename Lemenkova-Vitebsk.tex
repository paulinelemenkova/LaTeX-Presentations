\documentclass[pdflatex,compress,8pt,
	xcolor={dvipsnames,dvipsnames,svgnames,x11names,table},
	hyperref={colorlinks = true,breaklinks = true, urlcolor = NavyBlue, breaklinks = true}]{beamer}
\usetheme[
%%% option passed to the outer theme
%    progressstyle=fixedCircCnt,   % fixedCircCnt, movingCircCnt (moving is deault)
  ]{Feather}
  
% If you want to change the colors of the various elements in the theme, edit and uncomment the following lines

% Change the bar colors:
%\setbeamercolor{Feather}{fg=red!20,bg=red}

% Change the color of the structural elements:
%\setbeamercolor{structure}{fg=red}

% Change the frame title text color:
%\setbeamercolor{frametitle}{fg=blue}

% Change the normal text color background:
%\setbeamercolor{normal text}{fg=black,bg=gray!10}

%-------------------------------------------------------
% INCLUDE PACKAGES
%-------------------------------------------------------

\usepackage[utf8]{inputenc}
\usepackage[english]{babel}
\usepackage[T1]{fontenc}
\usepackage{helvet}
\usepackage{gensymb} % degree symbol
\usepackage[super]{nth}
\usepackage{amsmath}
\usepackage{subfig}
\usepackage{csquotes}

% Путь к файлам с иллюстрациями
\graphicspath{{fig/}} % path to folder with Figures

% ----------------------------------------------------------------------------
% *** START BIBLIOGRAPHY <<<
% ----------------------------------------------------------------------------
\usepackage[
	backend=biber, 
%	style = numeric,
%	style=ieee,
%	style=nature,
%	style=science,
%	style=apa,
%	style=mla,
	style = phys,
	maxbibnames=99,
	citestyle=numeric,
	giveninits=true,
	isbn=true,
	url=true,
	natbib=true,
	sorting=ndymdt,
	bibencoding=utf8,
	useprefix=false,
	language=auto, 
	autolang=other,
	backref=true,
	backrefstyle=none,
	indexing=cite,
]{biblatex}
\DeclareSortingTemplate{ndymdt}{
  \sort{
    \field{presort}
  }
  \sort[final]{
    \field{sortkey}
  }
  \sort{
    \field{sortname}
    \field{author}
    \field{editor}
    \field{translator}
    \field{sorttitle}
    \field{title}
  }
  \sort[direction=descending]{
    \field{sortyear}
    \field{year}
    \literal{9999}
  }
  \sort[direction=descending]{
    \field[padside=left,padwidth=2,padchar=0]{month}
    \literal{99}
  }
  \sort[direction=descending]{
    \field[padside=left,padwidth=2,padchar=0]{day}
    \literal{99}
  }
  \sort{
    \field{sorttitle}
  }
  \sort[direction=descending]{
    \field[padside=left,padwidth=4,padchar=0]{volume}
    \literal{9999}
  }
}

\addbibresource{Vitebsk.bib}
\renewcommand*{\bibfont}{\scriptsize} %\footnotesize

\setbeamertemplate{bibliography item}{\insertbiblabel}

% ----------------------------------------------------------------------------
% *** END BIBLIOGRAPHY <<<
% ----------------------------------------------------------------------------

% ----------------------------------------------------------------------------
% делать footnote \title[Short Title]{Long Title}
\makeatletter
\setbeamertemplate{footline}{%
\leavevmode%
\hbox{\begin{beamercolorbox}[wd=.24 \paperwidth,ht=2.5ex,dp=1.125ex,leftskip=.01cm plus1fill,rightskip=.05cm]{author in head/foot}%
            \usebeamerfont{title in head/foot}\insertshortauthor
    \end{beamercolorbox}%
    \begin{beamercolorbox}[wd=.76\paperwidth,ht=2.5ex,dp=1.125ex,leftskip=.05cm,rightskip=.15cm plus1fil]{title in head/foot}%
        \usebeamerfont{title in head/foot}\insertshorttitle{}
        \insertframenumber{} / \inserttotalframenumber \ \hspace*{2ex} 
    \end{beamercolorbox}}%
    \vskip0pt%
}
\makeatother

% ----------------------------------------------------------------------------

%%%%%%%%%%%%%%%%%%%%%%%%%%%%%%%%%%%%%%%

%-------------------------------------------------------
% DEFFINING AND REDEFINING COMMANDS
%-------------------------------------------------------

% colored hyperlinks
\newcommand{\chref}[2]{
  \href{#1}{{\usebeamercolor[bg]{Feather}#2}}
}

%-------------------------------------------------------
% INFORMATION IN THE TITLE PAGE
%-------------------------------------------------------

\title[Polina Lemenkova, 03/06/2015: To the question of the environmental education: how Landsat TM, ETM+ and MSS] 
{ \textbf{To the question of the environmental education:\\
 how Landsat TM, ETM+ and MSS images\\ 
 can be processed by GIS-techniques\\
 for geospatial research}}

\subtitle[Trends and Perspectives in the Creating Regional Systems]{
\small{Trends and Perspectives in Creating \\
Regional Systems of the Additional Education of Adults}}

\author[Polina Lemenkova]{Polina Lemenkova}

\institute[Vitebsk State Technological University VGTU]
{
      Vitebsk State Technological University (VGTU), Vitebsk, Belarus
  
}

\date{June 3, 2015}

%-------------------------------------------------------
% THE BODY OF THE PRESENTATION
%-------------------------------------------------------

\begin{document}

%-------------------------------------------------------
% THE TITLEPAGE
%-------------------------------------------------------

{\1% % this is the name of the PDF file for the background
\begin{frame}[plain,noframenumbering] % the plain option removes the header from the title page, noframenumbering removes the numbering of this frame only
  \titlepage % call the title page information from above
\end{frame}}

\begin{frame}{Content}
	\tableofcontents
\end{frame}

\section{Introduction}
\subsection{Research Summary}
\begin{frame}\frametitle{Research Summary}
 \begin{itemize}
        	\item GIS and RS application for environmental studies of Yamal
	\item Calculation of NDVI 
	\item Monitoring vegetation changes in tundra landscapes
	\item Analysis of the vegetation dynamics in the past two decades (1988-2011).
	\item Data: Landsat TM scenes for 1988, 2001 and 2011
	\item Originality: Application of ILWIS
	\item GIS spatial analysis tools and Landsat imagery 
	\item Area: Bovanenkovo region in Yamal Peninsula, Russian Extreme North
\end{itemize}
\end{frame}

\subsection{Study Area}
\begin{frame}\frametitle{Study Area}
\begin{minipage}[0.4\textheight]{\textwidth}
\begin{columns}[T]
\begin{column}{0.5\textwidth}
\vspace{2em}
\begin{figure}[H]
	\centering
		\includegraphics[width=4.0cm]{f02.jpg}
\end{figure}
Source: B. Forbes
\end{column}
\begin{column}{0.5\textwidth}
\vspace{2em}
\begin{figure}[H]
	\centering
		\includegraphics[width=4.0cm]{f03.jpg}
\end{figure}
Geographic location: Yamal Peninsula, north Russia.
(a) Geographic location of Yamal
(b) Location of the study Peninsula Map source: google.com area on Yamal (western coast). 
\end{column}
\end{columns}
\end{minipage}
\end{frame}

\section{Yamal Peninsula: Geographic Settings}
\subsection{Climate and Environment}
\begin{frame}\frametitle{Climate and Environment}
\begin{minipage}[0.4\textheight]{\textwidth}
\begin{columns}[T]
\begin{column}{0.5\textwidth}
Yamal Peninsula: geomorphology : flat geomorphology, elevations lower than 90 m
Processes:
 \begin{itemize}
        	\item seasonal flooding,
	\item active erosion processing,
	\item permafrost distribution,
	\item cryogenic landslides formation
\end{itemize}
Landslides affect local ecosystem structure. \\
Landslides change vegetation types recovering after the disaster.
\end{column}
\begin{column}{0.5\textwidth}
Landscapes of Yamal. 
\begin{figure}[H]
	\centering
		\includegraphics[width=5.0cm]{f04.jpg}
\end{figure}
Source: http://pixtale.net/
\end{column}
\end{columns}
\end{minipage}
\end{frame}

\subsection{Landscapes}
\begin{frame}\frametitle{Landscapes of the Yamal Peninsula - I}
\begin{figure}[H]
	\centering
		\includegraphics[width=10cm]{f05.jpg}
\end{figure}
\end{frame}

\begin{frame}\frametitle{Landscapes of the Yamal Peninsula - II}
\begin{figure}[H]
	\centering
		\includegraphics[width=11.0cm]{f06.jpg}
\end{figure}
Dry grass heath tundra (left). Sedge grass tundra (center). Dry short shrub tundra (right)
\begin{figure}[H]
	\centering
		\includegraphics[width=7.0cm]{f07.jpg}
\end{figure}
Landscapes of Yamal (left). Sphagnum moss (right)
\end{frame}

\begin{frame}\frametitle{Landscapes of the Yamal Peninsula - III}
\begin{figure}[H]
	\centering
		\includegraphics[width=11.0cm]{f08.jpg}
\end{figure}
Dry short shrub sedge tundra (left). Wetlands (right)
\begin{figure}[H]
	\centering
		\includegraphics[width=7.0cm]{f09.jpg}
\end{figure}
Short shrub tundra
\end{frame}

\section{Methodology}
\begin{frame}\frametitle{Methodology: ILWIS GIS}
\begin{minipage}[0.4\textheight]{\textwidth}
\begin{columns}[T]
\begin{column}{0.5\textwidth}
Technical tools: The RS data processing was performed in ILWIS GIS software.
Research Methods: 
\begin{itemize}
	\item Image interpretation (Landsat TM scenes).
	\item Supervised classification
\end{itemize}
Following working steps summarize research scheme used in this research:
\begin{itemize}
	\item Data pre-processing
	\item Creation of image composites of several bands
	\item Supervised classification using various classifiers 
	\item Spatial analysis and interpretation of the results
	\item Time series analysis for detecting changes
	\item Final GIS mapping
\end{itemize}
\end{column}
\begin{column}{0.5\textwidth}
\begin{figure}[H]
	\centering
		\includegraphics[width=5.0cm]{f10.jpg}
\end{figure}
ILWIS GIS: https://52north.org/software/software-projects/ilwis/
\end{column}
\end{columns}
\end{minipage}
\end{frame}

\subsection{Minimum Distance Method}
\begin{frame}\frametitle{Minimum Distance Method}
Calculation of vegetation indices, especially and in this case Normalized Difference Vegetation Index (NDVI), has become one of the most successful, popular and traditional attempts in biogeographical research methods.
\begin{itemize}
         \item The principle of Minimum Distance method used for classification is based on the calculating of the shortest straight-line distance in Euclidian coordinate system from each pixel’s DN to the pattern pixels of land cover classes.
	\item The main weakness of the supervised classification method is caused by modeling approach and technical details of image recognition, i.e. errors in pixels classification.
	\item The misclassification by the Minimum Distance method may occur due to the ambiguity and erroneous recognition of some of the pixels as well as insufficient representation of classes.
\end{itemize}
\end{frame}

\subsection{Workflow}
\begin{frame}\frametitle{Workflow}
Data pre-processing
\begin{itemize}
         \item import .img into ASCII raster format (GDAL). After converting, each image contained collection of 7 raster bands 
	\item Pre-processing (visual color and contrast enhancement) 
	\item Geographic referencing of Landsat scenes, initially based on  WGS 1984 datum: UTM (Universal Transverse Mercator) Projection, Eastern Zone 42, Northern Zone W, (Georeference Corner Editor)
	\item Crop of study area: the area of interest (AOI) was identified and cropped on the raw images. This area shows Bovanenkovo region in a large scale and best represents typical tundra landscapes.
	\item Supervsised Classification vi.GIS visualization and mapping
\end{itemize}
\end{frame}

\subsection{Data Import and Conversion}
\begin{frame}\frametitle{Data Import and Conversion}
\begin{itemize}
        \item Test area selection (Mask): 67\degree 00' - 72\degree 00' E - 70\degree 00' - 71\degree 00' N. 
	\item 3 selected  Landsat TM satellite images show Yamal region in 1988, 2001, 2011.
	\item Time span: 23 years (1988, 2001, 2011). 
	\item Summer months selected for vegetation assessment. 
	\item Data conversion / original images in format .TIFF converted to Erdas Imagine .img.
\end{itemize}
\begin{figure}[H]
	\centering
		\includegraphics[width=10.0cm]{f11.jpg}
\end{figure}
Initial remote sensing data, left to right: Landsat TM 1988, bands 7-3-1; Landsat TM 2011, pseudo natural colors composite; Landsat ETM + 2001 bands 6-3-1.
\end{frame}

\subsection{Data Georeferencing}
\begin{frame}\frametitle{Georeferencing: Google Earth}
\begin{figure}[H]
	\centering
		\includegraphics[width=10.0cm]{f19.jpg}
\end{figure}
\end{frame}

\subsection{Supervised Classification}
\begin{frame}\frametitle{Supervised Classification of the Landsat TM Image}
\begin{figure}[H]
	\centering
		\includegraphics[width=10.0cm]{N.jpg}
\end{figure}
\end{frame}

\section{Results}
\subsection{Computations}
\begin{frame}\frametitle{Computing Pixels on Various Land Cover Classes}
\begin{figure}[H]
	\centering
		\includegraphics[width=10.0cm]{T1.jpg}
\end{figure}
\end{frame}

\subsection{Mapping}
\begin{frame}\frametitle{Land Cover Classes}
\begin{figure}[H]
	\centering
		\subfloat {\includegraphics[width=3.5cm]{F22.jpg}}
		\subfloat {\includegraphics[width=3.5cm]{F23.jpg}}
		\subfloat {\includegraphics[width=3.5cm]{F24.jpg}}
\end{figure}
Results show classified maps of the selected region of Bovanenkovo on 1988, 2001 and 2011 yr (from left to right).
GIS mapping is performed using image classification. Results of the supervised classification show 3 maps of the vegetation distribution (above). 
\end{frame}

\subsection{Comments}
\begin{frame}\frametitle{Comments}
Classification is based on the relationship between the spectral signatures and object variables, i.e. vegetation types. Water areas are defined as “no vegetation” class.

\begin{block}{1988}
For year 1988 “forest” class covered 8,188,926 pixels, which is 11,32\% from the total amount.
\end{block}

\begin{examples}{1988:}
Maximal area, except for water, is covered by the shrubland (15,29\% from the total).
\end{examples}

\begin{block}{2011}
For year 2011, the percentage of the shrubland decreased down to 6,26\%, while area of forests increased from 11,32 to 15,97\%.
\end{block}

\begin{examples}{2011:}
The area of grass remained relatively stable with values slightly increasing to about 2\%
\end{examples}

\end{frame}

\subsection{Statistical Histograms}
\begin{frame}\frametitle{Statistical Histograms}
Histogram for supervised classification of Landsat TM image, 2011.
\begin{figure}[H]
	\centering
		\subfloat {\includegraphics[width=7.0cm]{F25.jpg}}
			\vspace{2mm}
		\subfloat {\includegraphics[width=7.0cm]{F26.jpg}}
\end{figure}
\end{frame}

\section{Discussion}
\begin{frame}\frametitle{Discussion}
\begin{itemize}
        \item This research presented GIS based studies of the environment of Yamal Peninsula
        \item The study is technically based on ILWIS GIS, effective tool for spatial analysis
        \item The results of the spatial analysis are presented as 3 GIS maps illustrating changes in vegetation based on the image analysis using Landsat TM.
        \item Calculated land cover changes indicated vegetation dynamics in years 1988, 2001 and 2011.
        \item Application of the RS data is especially important for studies of the northern ecosystems, because it enables studying remotely located areas of Arctic
	\item GIS-based processing of the RS data (Landsat TM) improves technical aspects of the landscape studies and monitoring
	\item The results show successful use of ILWIS GIS software for spatio-temporal classification of the satellite images aimed at ecological mapping.
	\end{itemize}
\end{frame}

\section{Conclusion}
\begin{frame}\frametitle{Conclusion}

\begin{alertblock}{Time-Series Analysis}
Remote sensing plays important role in land use studies and serves as a valuable source of spatial information for the time series analysis.
\end{alertblock}

\begin{block}{ILWIS GIS}
Using enhanced ILWIS GIS tools to analyze and process satellite imagery contributes to the environmental analysis of the land cover changes. The classification used in the current work is pixel-based aimed to allocate and categorize pixels on the image to the created classes.
\end{block}

\begin{examples}{Remote Sensing:}
While traditional methods for vegetation monitoring are fieldwork and ground surveys, usually performed in large-scale areas, the use of remote sensing techniques enables to monitor extended areas in a small scale, as well as to assess temporal changes.
\end{examples}

\end{frame}

\section{Thanks}
\begin{frame}{Thanks}
  	\centering \LARGE 
  	\emph{Thank you for attention !}\\
	\vspace{5em}
\normalsize
Acknowledgement: \\
Current research has been funded by the \\
Finnish Centre for International Mobility (CIMO) \\
Grant No. TM-10-7124, for author's research stay at \\
Arctic Center, University of Lapland (2012).
\end{frame}

\section{Literature}
\begin{frame}\frametitle{Literature}
\begin{figure}[H]
	\centering
		\includegraphics[width=10.0cm]{L.jpg}
\end{figure}

\end{frame}

%%%%%%%%%%% Bibliography %%%%%%%
\section{Bibliography}
%\vspace{2em}
\large{\textcolor{white}{Bibliography}}
\nocite{*}
\printbibliography

\end{document}