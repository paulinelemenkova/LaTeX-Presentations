\documentclass[pdflatex,compress,9pt,
	xcolor={dvipsnames,dvipsnames,svgnames,x11names,table},
	hyperref={colorlinks = true,breaklinks = true, urlcolor = NavyBlue, breaklinks = true}]{beamer}
\usetheme{Goettingen}

\usepackage[utf8]{inputenc}
\usepackage[T2A,T1]{fontenc}
\usepackage{gensymb} % degree symbol
\usepackage[super]{nth}


% ----------------------------------------------------------------------------
% *** START BIBLIOGRAPHY <<<
% ----------------------------------------------------------------------------
\usepackage[
	backend=biber, 
	style = numeric,
%	style=nature,
%	style=science,
%	style=apa,
%	style=mla,
%	style=phys, 
	maxbibnames=99,
%	citestyle=authoryear,
	citestyle=numeric,
	giveninits=true,
	isbn=true,
	url=true,
	natbib=true,
	sorting=ndymdt,
	bibencoding=utf8,
	useprefix=false,
	language=auto, 
	autolang=other,
	backref=true,
	backrefstyle=none,
	indexing=cite,
]{biblatex}
\DeclareSortingTemplate{ndymdt}{
  \sort{
    \field{presort}
  }
  \sort[final]{
    \field{sortkey}
  }
  \sort{
    \field{sortname}
    \field{author}
    \field{editor}
    \field{translator}
    \field{sorttitle}
    \field{title}
  }
  \sort[direction=descending]{
    \field{sortyear}
    \field{year}
    \literal{9999}
  }
  \sort[direction=descending]{
    \field[padside=left,padwidth=2,padchar=0]{month}
    \literal{99}
  }
  \sort[direction=descending]{
    \field[padside=left,padwidth=2,padchar=0]{day}
    \literal{99}
  }
  \sort{
    \field{sorttitle}
  }
  \sort[direction=descending]{
    \field[padside=left,padwidth=4,padchar=0]{volume}
    \literal{9999}
  }
}

\addbibresource{YO.bib}
\renewcommand*{\bibfont}{\tiny} %\scriptsize \footnotesize

\setbeamertemplate{bibliography item}{\insertbiblabel}

% ----------------------------------------------------------------------------
% *** END BIBLIOGRAPHY <<<
% ----------------------------------------------------------------------------

\title{Satellite Image Based Mapping of Wetland Tundra Landscapes Using ILWIS GIS}
\author{Polina Lemenkova}
         \date{March 19, 2015}

\begin{document}
\begin{frame}
           \titlepage
\end{frame}

\section*{Outline}
\begin{frame}
           \tableofcontents
\end{frame}

\section{Introduction}
\subsection{Research Goals}
\begin{frame}{Research Goals}
\begin{itemize}
            \item Distribution of different types of landscapes in the wetland tundra of the Yamal Peninsula
            \item Monitoring changes in the landscapes of tundra
            \item Analysis of the landscape dynamics for 2 decades (1988-2011).
            \item Data: Landsat TM satellite images for 1988 and 2011
            \item Application of ILWIS GIS for spatial analysis and data processing on the region of Bovanenkovo, Yamal.
            \item Technical approach: Remote sensing data processing by ILWIS GIS.
            \item Methods: Supervised classification of Landsat TM images
            \item Study area: tundra landscapes in the wetlands of the Yamal Peninsula in the Far North of Russia
\end{itemize}
\end{frame}

\section{Geographic Settings}

\subsection{Geomorphology of the Yamal Peninsula}
\begin{frame}{Geomorphology of the Yamal Peninsula}

\begin{minipage}[0.4\textheight]{\textwidth}
\begin{columns}[T]
\begin{column}{0.5\textwidth}
\begin{figure}[H]
	\centering
		\includegraphics[width=4.0cm]{F3.png}
\end{figure}
\begin{figure}[H]
	\centering
		\includegraphics[width=4.0cm]{F4.png}
\end{figure}
\end{column}
\begin{column}{0.5\textwidth}
\vspace{2em}
	Key points on the Yamal geomorphology: 
\begin{itemize}
       \item[-] Elevations almost flat, terrain less than 90 m.
       \item[-] Seasonal flooding
       \item[-] Active processes of erosion
       \item[-] Permafrost distribution
       \item[-] Local formation of ground cryogenic landslides
       \item[-] Specific ecological and climatic conditions (Arctic)
\end{itemize}
\end{column}
\end{columns}
\end{minipage}
\end{frame}

\subsection{Landscapes of the Yamal Peninsula} 
\begin{frame}\frametitle{Landscapes of the Yamal Peninsula}
\begin{minipage}[0.4\textheight]{\textwidth}
\begin{columns}[T]
\begin{column}{0.5\textwidth}
\begin{itemize}
       \item[*] Cryogenic landslides are formed as a result of the soil erosion are typical processes in the Yamal Peninsula
       \item[*] Soil erosion develop as a result of the soil subsidence and soil thawing
\end{itemize}
\begin{figure}[H]
	\centering
		\includegraphics[width=5.0cm]{F1.png}
\end{figure}
\end{column}
\begin{column}{0.5\textwidth}
\begin{itemize}
       \item[*] Cryogen landslides have a negative impact on the local ecosystems
       \item[*] Cryogen landslides disrupt the strata of the soil and slow down restoration of vegetation after the landslide
\end{itemize}
\begin{figure}[H]
	\centering
		\includegraphics[width=5.0cm]{F2.png}
\end{figure}
\end{column}
\end{columns}
\end{minipage}
\end{frame}

\subsection{Cryogenic Landslides on the Yamal Peninsula}
\begin{frame}{Cryogenic Landslides on the Yamal Peninsula}
\begin{itemize}
       \item[--] The negative effect of cryogenic landslides - changes in types of vegetation cover at the place of their formation.
       \item[--] For 10 years after active cryogenic landslide processes, the area of their occurrence remains uncovered.
       \item[--] Then, over the next few years, a process of slow restoration of the soil and vegetation cover takes place 
       \item[--] Vegetation succession: plant communities with dominant herbs, mosses, lichens and sedge, willow and meadows with short shrubs.
       \item[--] Vegetation in the early stages of restoration (mosses, lichens) indirectly indicates recent formation of the cryogenic landslides 
       \item[--] Meadows and willow shrub, on the contrary, indicate a relatively developed and restored plant community. 
       \item[--] Areas subjected to the formation of cryogenic landslides in past 2-3 decades are usually characterized by the spread of willow and shrubbery, an indirect indicator of these processes in the past.
\end{itemize}
\end{frame}

\section{Methods}
\subsection{Data Processing Algorithm} 
\begin{frame}\frametitle{Data Processing Algorithm}
\begin{minipage}[0.4\textheight]{\textwidth}
\begin{columns}[T]
\begin{column}{0.5\textwidth}
\vspace{2em}
\begin{figure}[H]
	\centering
		\includegraphics[width=3.5cm]{F5.png}
\end{figure}
\begin{figure}[H]
	\centering
		\includegraphics[width=3.5cm]{F6.png}
\end{figure}
Examples of various types of the vegetation typical for the Yamal tundra, Arctic.
\end{column}
\begin{column}{0.5\textwidth}
Algorithms of the data processing in ILWIS GIS: 
\begin{enumerate}[i]
            \item Data collection, import and conversion
            \item Data: 2 Landsat TM images, 1988 \& 2011
            \item Data pre-processing 
            \item Georeferencing: WGS 1984 ellipsoid to UTM, E42, NW
            \item 3 spectral channels for image processing: color composite\& multi-band layers
            \item Clustering segmentation and classification
            \item GIS mapping, spatial analysis
            \item Google Earth imagery verification
            \item Results interpretation
\end{enumerate}
\end{column}
\end{columns}
\end{minipage}
\end{frame}

\subsection{Research Questions and Aims} 
\begin{frame}\frametitle{Research Questions and Aims}
\begin{minipage}[0.4\textheight]{\textwidth}
\begin{columns}[T]
\begin{column}{0.5\textwidth}
\vspace{2em}
\begin{figure}[H]
	\centering
		\includegraphics[width=5.0cm]{F7.png}
\end{figure}
\end{column}
\begin{column}{0.5\textwidth}
\vspace{2em}
Research questions and aims: 
\begin{enumerate}[(I)]
            \item The aim of the work is the use of GIS and RS data (Landsat TM) for monitoring tundra land cover types
            \item Approaches: images classification, visualization and mapping
            \item Have landscapes within the test territory of the study region changed over the past 14 years (1988-2011)?
            \item What types of land cover types were dominating previously, and which ones are now ?
            \item Methodologically, how ILWIS GIS can be used to process RS data ?
\end{enumerate}
\end{column}
\end{columns}
\end{minipage}
\end{frame}

\subsection{Landsat TM images}
\begin{frame}\frametitle{Landsat TM images}
\begin{minipage}[0.4\textheight]{\textwidth}
\begin{columns}[T]
\begin{column}{0.5\textwidth}
\begin{itemize}
            \item AOI mask: 67\degree 00'-72\degree 00'E-70\degree 00'- 71\degree 00'N 
\end{itemize}
\begin{figure}[H]
	\centering
		\includegraphics[width=5.0cm]{F8.png}
\end{figure}
\end{column}
\begin{column}{0.5\textwidth}
\begin{itemize}
            \item Time span: 23 years (1988-2011)
            \item Images taken during June to assess vegetation
            \item Original Landsat TM images (.tiff) were converted to the Erdas Imagine .img format.
\end{itemize}
\begin{figure}[H]
	\centering
		\includegraphics[width=5.0cm]{F9.png}
\end{figure}
\end{column}
\end{columns}
\end{minipage}
\end{frame}

\subsection{Image Georeferencing}
\begin{frame}\frametitle{Image Georeferencing}
\begin{figure}[H]
	\centering
		\includegraphics[width=10.0cm]{F14.png}
\end{figure}
Georeference Corner Editor of ILWIS GIS
\end{frame}

\subsection{Spectral Reflectance}
\begin{frame}\frametitle{Spectral Reflectance (SR)}
\begin{enumerate}[SR i.]
	\item Image classification is grouping pixels into classes (merging pixels)
	\item Clusters correspond to the types of vegetation cover according to the AOI settings
	\item Classification is based on using spectral brightness of the image pixels
	\item Spectral and texture characteristics of various land cover types are displayed on the image as different spectral brightnesses of the pixels
	\item Spectral reflectances show spectral reflectivity of the land cover types (through pixels' spectral brightness) and individual properties of the vegetation objects detected on a raster image 
\end{enumerate}
\end{frame}

\subsection{Image Clustering}
\begin{frame}\frametitle{Image Clustering}
\begin{enumerate}[(a)]
            \item Cluster analysis is a statistical procedure for grouping objects (pixels on a raster image)
            \item Pixels are ordered into homogeneous thematic groups (clusters)
            \item Each digital pixel in the image is assigned to the corresponding land cover type group
            \item Grouping is based on the proximity of the spectral brightness value (Digital Number, DN) of the pixel to the centroid.
            \item The logical segmentation algorithm consists of grouping the pixels in the image (merging pixels) into clusters. 
            \item Grouping pixels occurs in semi-automatic mode based on the distinctness from neighboring (neighbor pixels).
            \item The process is repeated interactively until optimal values of the classes (and pixels attached to these classes) are reached.
\end{enumerate}
\end{frame}

\subsection{Image Classification}
\begin{frame}\frametitle{Image Classification (IC)}
\begin{minipage}[0.4\textheight]{\textwidth}
\begin{columns}[T]
\begin{column}{0.5\textwidth}
\vspace{2em}
\begin{figure}[H]
	\centering
		\includegraphics[width=4.0cm]{F10.png}
\end{figure}
\begin{figure}[H]
	\centering
		\includegraphics[width=4.0cm]{F11.png}
\end{figure}
\end{column}
\begin{column}{0.5\textwidth}
\vspace{2em} 
\begin{enumerate}[{IC}-1]
            \item Thematic mapping is based on the results of the classification of images
            \item Visualization of the landscapes' structure and vegetation types within the AOI.
            \item To classify land cover types, image pixels were identified for each category and grouped into different land categories.
            \item Land cover types were evaluated and identified with each land cover class
            \item Number of cluster groups is 13 representing vegetation land cover types of the Yamal tundra
\end{enumerate}
\end{column}
\end{columns}
\end{minipage}
\end{frame}

\section{Results}
\subsection{Mapping Results}
\begin{frame}\frametitle{Mapping Results}
\begin{minipage}[0.4\textheight]{\textwidth}
\begin{columns}[T]
\begin{column}{0.5\textwidth}
1988
\begin{figure}[H]
	\centering
		\includegraphics[width=5.0cm]{F12.png}
\end{figure}
\end{column}
\begin{column}{0.5\textwidth}
2011
\begin{figure}[H]
	\centering
		\includegraphics[width=5.0cm]{F13.png}
\end{figure}
\end{column}
\end{columns}
\end{minipage}
\end{frame}

\subsection{Results Interpretation}
\begin{frame}\frametitle{Results Interpretation}
\begin{itemize}
            \item Statistical results of calculations of types of vegetation cover were obtained in a semi-automatic mode in ILWIS GIS
            \item 1988 'willow shrubs' type covered 412,292 pixels from the total part of the AOI, and 'high willow' class is 823,430 pixels
            \item 2011: willow increased to 651427 pixels, ('willow shrubs'), and 893092 pixels ('high willows')
            \item Both combined classes of willows, typical for AOI with a high water content, cover total 1544519 pixels, which is 40.27 \%. 
            \item Area of grasses decreased compared to shrub and willow
            \item Max area covered by class 'heather and dry grass' is 933798 pixels
\end{itemize}
\end{frame}

\subsection{Google Earth Verification}
\begin{frame}\frametitle{Google Earth Verification}
\begin{minipage}[0.4\textheight]{\textwidth}
\begin{columns}[T]
\begin{column}{0.5\textwidth}
\vspace{2em}
\begin{figure}[H]
	\centering
		\includegraphics[width=5.0cm]{F15.png}
\end{figure}
The selected area represents one of the most diversified part of the tundra landscapes of Yamal
\end{column}
\begin{column}{0.5\textwidth}
\vspace{2em} 
\begin{itemize}
            \item AOI has a complex structure of boggy landscapes and unique types of vegetation
            \item Therefore, in order to control the most difficult areas, the images were verified by Google Earth
            \item Visualization of the same area in the satellite image and Google Earth image at the same time.
            \item This made it possible to visually check heterogeneous areas with mixed land cover types and landscapes
\end{itemize}
\end{column}
\end{columns}
\end{minipage}
\end{frame}

\section{Conclusions}
\begin{frame}{Conclusions}
\begin{enumerate}[I.]
            \item Monitoring landscape changes is an important tool for assessing the ecological stability of a region
            \item Spatial analysis of the multi-temporal satellite images by ILWIS GIS algorithms is an effective tool
            \item Research demonstrated how Yamal wetland tundra landscapes changed over a 23-year period of time
            \item Data included LandsatTM satellite imagery covering the Yamal Peninsula, Far North of Russia
            \item Image processing was done by classification methods.
            \item Results shown changes in the landscapes from 1988 to 2011
            \item Results confirm presence of the destructive processes caused changes in tundra boggy landscapes.
            \item Research demonstrated successful ILWIS GIS based of the RS data analysis, effective for tundra monitoring
\end{enumerate}
\end{frame}


\section{Thanks}
\begin{frame}{Thanks}
  	\centering \LARGE 
  	\emph{Thank you for attention !}\\
	\vspace{5em}
\normalsize
Acknowledgement: \\
Current research has been funded by the \\
Finnish Centre for International Mobility (CIMO) \\
Grant No. TM-10-7124, for author's research stay at \\
Arctic Center, University of Lapland (2012).
\end{frame}

%%%%%%%%%%% Bibliography %%%%%%%
\section{Bibliography}
\Large{Bibliography}
\nocite{*}
\printbibliography[heading=none]
	
%%%%%%%%%%% Bibliography %%%%%%%	

\end{document}
%Changing the font size locally (from biggest to smallest):	
%\Huge
%\huge
%\LARGE
%\Large
%\large
%\normalsize (default)
%\small
%\footnotesize
%\scriptsize
%\tiny

\end{document}